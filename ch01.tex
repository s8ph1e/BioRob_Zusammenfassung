\subsection{Bionik}

\begin{itemize}
	\item Biologisch inspirierte Forschung
	\item Systematisches Lernen von der Natur
	\item Bionik ist die Umsetzung von biologischen Problemlösungen in die Technik
	\item Anwendungsbereiche: Konstruktionen, Sensorik, Bewegung, Algorithmen...
\end{itemize}

Beispiele:
\begin{itemize}
	\item In der Luftfahrt: Inspiration Vogelflug
	\begin{itemize}
		\item Gezielte Luftverwirbelung am Ende der Tragfläche
		\item Treibstoffverbrauch um ca. $3-5\%$ reduziert
	\end{itemize}
	\item Im Alltag: Inspiration Lotusblume
	\begin{itemize}
		\item Schmutzabweisende Oberflächen durch Nanostrukturen (z.B. Fassadenfarben)
	\end{itemize}
	\item Im Alltag: Inspiration Klettfrüchte
	\begin{itemize}
		\item Textilen Klettverschluss basiert auf Mikrohaken
	\end{itemize}	
\end{itemize}

\subsection{Klassische Robotik}

Vier Robotergenerationen:
\begin{enumerate}
	\item Generation: (programmierbare Manipulatoren, $1960--1075$)
	\begin{itemize}
		\item geringe Rechenleistung, nur feste Haltepunkte (Punkt-zu-Punkt-Programmierung), kaum sensorielle Fähigkeiten (Pick-and-Place-Aktionen)
	\end{itemize}
	\item Generation: (adaptive Roboter, $1976--1982$)
	\begin{itemize}
		\item mehr Sensoren (z.B. Kameras), Anpassung an Umwelt, eigene Programmiersprache (z.B. VAL), geringe Roboter-Intelligenz (adaptive Aufgabendurchführung)
	\end{itemize}
	\item Generation: (autonome Roboer, ab $1982$)
	\begin{itemize}
		\item hohe Rechenleistung (Multiprozessorsysteme), Aufgabenorientierte Programmierung, Forderung nach (maschineller) Autonomie
	\end{itemize}
	\item Generation: (humanoide AI-Roboter)
	\begin{itemize}
		\item Flexibilität bzgl. Umwelt und Aufgabe, Lernfähigkeit, Selbstreflexion, Emotion
	\end{itemize}
\end{enumerate}
Anwendungsfelder: Industrie-Roboter, Service-Roboter, Personal-Roboter

\subsection{Bionik + Robotik}
\textbf{Systematisches Lernen von der Natur mit Anwendung in der Robotik}
\begin{itemize}
	\item Biologisch motivierte Roboter
	\item Umsetzung von biologischen Problemlösungen in der Robotik
	\item Konstruktion von biologisch inspirierten Robotern, Bewegungssteuerung, Evolutionäre Algorithmen, etc.\
\end{itemize}

\subsection{Motivation}
\textbf{Die Roboter-Akzeptanz hängt primär von dessen Natürlichkeit ab}: Begeisterung für natürliche, biologisch motivierte Systeme. \\

Vorteile:
\begin{itemize}
	\item leicht, robust, anpassungs- und lernfähig, energetisch optimiert, fehlertolerant, flexibel, flüssige Bewegungen (hardwareschonened)
	\item Integration in menschliches Umfeld
	\item Keine Roboter-Infrastruktur notwendig
\end{itemize}

\subsection{Begriffserklärung}
\begin{itemize}
	\item Biologisch motivierte Robotersysteme sollen primär durch ihre Lokomotionsformen definiert werden: \textbf{Schwimmen, Kriechen, Laufen, Hüpfen, Fliegen}
	\item Für Realisierung der Lokomotion werden Prinzipien aus der Natur übertragen: \textbf{Antrieb, Konstruktion, Umwelterfassung, Steuerung}
	\item Die \textbf{direkte Abbildung} aus der Natur ist \textbf{nicht notwendig} und meistens auch nicht sinnvoll/machbar
\end{itemize}

\subsection{Systemanforderungen BioBots}
\begin{itemize}
	\item Autonomie und Mobilität
	\item Energieautarkie
	\item Hohe Flexibilität
	\item Robustheit gegenüber unterschiedlichen Störungen
	\item Umfangreiche perzeptive Komponenten zur Erfassung interner Zustände und der Umwelt
	\item Adaptive Verhaltenssteuerung
	\item Erweiterbarkeit (perzeptive Komponenten, Skills)	
\end{itemize}

\begin{figure}
	\centering
	\includegraphics[width=0.7\textwidth]{figures/interdisziplinaer.png}
\end{figure}

\begin{figure}
	\centering
	\includegraphics[width=0.7\textwidth]{figures/einordnung.png}
\end{figure}

\subsection{Anwendungsbeispiel}
\begin{itemize}
	\item Applikation
	\begin{itemize}
		\item Leitzustand
		\item Benutzerinteraktion
		\item Missionsplanung
	\end{itemize}
	\item Navigation
	\begin{itemize}
		\item Lokalisation
		\item Umweltmodell
		\item Pfadplanung
	\end{itemize}
	\item Lokomotion
	\begin{itemize}
		\item Internes Modell
		\item Lokale Adaption
		\item Gangarten
		\item Reflexe
	\end{itemize}
	\item Basissteuerung
	\begin{itemize}
		\item Gelenke
		\item Beine/Körper
		\item Motorprimitive
	\end{itemize}
	\item Hardware
	\begin{itemize}
		\item Mechanik
		\item Aktuatoren
		\item Sensorik
		\item Elektronik
	\end{itemize}
\end{itemize}

\subsection{Anwendungsfelder}
\begin{itemize}
	\item Wartungs- und Inspektionsaufgaben
	\item Erkundung von rauem Gelände (Vulkane, Meeresboden, Planeten)
	\item Untersuchung von Meeresböden, Pipelines
	\item Forst- und Landwirtschaft
	\item Katastrophenschutz
	\item Prothetik
	\item Rehabilitation
	\item Assistenzaufgaben
	\item Entertainment
\end{itemize}
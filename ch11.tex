\subsection{Vererbung und Evolution in der Biologie}

\subsubsection{Die Mendelschen Gesetze}

\textbf{Nomenklatur}
\begin{itemize}
	\item Homozygotie: Reinerbigkeit im Bezug auf ein genetisches Merkmal
	\item Heterozygotie: Mischerbigkeit im Bezug auf ein genetisches Merkmal
	\item Allel: mögliche Ausprägung eines Gens, das sich an einem bestimmten Ort auf einem Chromosom befindet
	\item Gen: Abschnitt auf der DNA
	\item Chromosom: Strukturen, die Gene und damit Erbinformationen enthalten. Sie bestehen aus DNA, die mit vielen Proteinen verpackt ist
	\item Genom (Erbgut): Gesamtheit der vererbbaren Informationen einer Zelle oder die Gesamtheit der materiellen Träger der vererbbaren Informationen, also die Gesamtheit der Chromosomen, der Gene oder DNA
	\item Diploider Organismus: besitzt von jedem Gen, das z.B.\ die Blutgruppe oder Haarfarbe kodiert, zwei Kopien, im Normalfall eine von jedem Elternteil (Mensch ist diploid)
\end{itemize}

\begin{figure}
	\centering
	\includegraphics[width=.5\textwidth]{figures/nomenklatur.png}
	\caption{Vererbung Nomenklatur}
\end{figure}

\textbf{Vererbung von Merkmalen in der Natur -- Mendelsche Gesetzte}
\begin{enumerate}
	\item Uniformitätsgesetz:
	Kreuzt man zwei homozygote Linien, die sich in einem oder mehreren Allelpaaren unterscheiden, so sind alle F1-Hybriden uniform.
	\begin{itemize}
		\item Es werden zwei reinerbige (homozygote) Eltern verpaart
		\item Die Eltern unterscheiden sich
		\item Die Nachkommen der ersten Generation (F1-Hybriden) sind dann gleich (uniform) bezogen auf das untersuchte Merkmal
		\item Unterscheidung je nach Erbgang:
		\begin{itemize}
			\item dominant-rezessiv: ein Merkmal dominant, ein Merkmal rezessiv
			\item intermediär: Merkmale bilden Mischform
			\item Kodominant: beide Merkmale separat ausgebildet
		\end{itemize}
	\end{itemize}
	\item Spaltungsgesetz:
	Kreuzt man (F1-)Hybriden, die in einem Allelpaar heterozygot sind, so ist die F1-Generation nicht uniform.
	\begin{itemize}
		\item Es werden zwei mischerbige (heterozygote) Eltern verpaart
		\item Die Eltern besitzen beide die gleichen Merkmalsanlagen
		\item Die Nachkommen der ersten Generation (F1-Hybriden) sind dann nicht gleich bezogen auf das untersuchte Merkmal
		\item Auswirkung durch Erbgang: dominant-rezessive Vererbung:
		\begin{itemize}
			\item $\frac{1}{4}$ rein rezessive
			\item $\frac{2}{4}$ mischerbig
			\item $\frac{1}{4}$ rein dominant
		\end{itemize}
	\end{itemize}
	\item Unabhängigkeitsregel (gilt eingeschränkt):
	Kreuzt man zwei hohozygote Linien untereinander, die sich in zwei oder mehreren Allelpaaren voneinander unterscheiden, so werden die einzelnen Allele unabhängig voneinander, entsprechend den beiden ersten Mendelschen Gesetzen vererbt.
\end{enumerate}

\subsubsection{Zellteilung, Vererbung und Mutation}
\textbf{Zellteilung}
\begin{itemize}
	\item \textbf{Meiose} = Reifeteilung = Reduktionsteilung
	\begin{itemize}
		\item besondere Art der Zellkernteilung
		\item Zahl der Chromosomen halbiert sich
		\item vollzieht sich immer in zwei Teilungsschritten
		\item Rekombination der elterlichen Chromosomen
		\item Grundlage der geschlechtlichen Fortpflanzung (Säugetiere, Fische, Vögel..)
		\item Haploidie: Chromosomensatz einer Zelle nur einfach vorhanden, Zellkern enthält von allen Chromosomentypen nur ein einziges Exemplar.
		Typischerweise sind die Chromosomensätze der Eizellen und Spermien haploid. Ihre haploiden Chromosomensätze verschmelzen bei der Befruchtung zum doppelten Chromosomensatz einer diploiden Zelle.
		\item Homolog: zwei Chromosomen enthalten gleiche Gene.
	\end{itemize}
	\item \textbf{Mitose} = Karyokinese = indirekte Kernteilung
	\begin{itemize}
		\item gewöhnliche Kernteilung
		\item Keine Rekombination der Chromosomen
		\item Beispiele: Bakterien, manche Pflanzen, Gewebe, ...
	\end{itemize}
\end{itemize}
%
\begin{figure}
	\centering
	\begin{subfigure}{\textwidth}
	\centering
	\includegraphics[width=\textwidth]{figures/meiose.png}
	\end{subfigure}
	\begin{subfigure}{.6\textwidth}
	\centering
	\includegraphics[width=\textwidth]{figures/meiose_1.png}
	\end{subfigure}
	\caption{Meiose -- Zellteilung}
\end{figure}
%
\textbf{Mutationen}
\begin{itemize}
	\item Genommutationen (Monosomie, Trisomie)
	\item Chromosommutation (Verlust, Verdopplung, Hinzufügung, Drehung $180°$)
	\item Genmutationen (ähnlich wie Chromosommutation nur auf Genbasis)
\end{itemize}

\subsection{Genetische Algorithmen -- \glqq Evolution im Rechner\grqq}

\subsubsection{Genetische Algorithmen -- Motivation und Nomenklatur}
Biologisches Evolutionsmodell nach Darwin: \emph{survival of the fittest}, \emph{natural selection}; Selektion als treibende Kraft\\
Evolution als Optimierung komplexer künstlicher Systeme: Zufall und Selektion $\Rightarrow$ hoch komplexe und and die Arbeits- bzw.\ Lebensumgebung hervorragend angepasste Systeme erzeugen lassen \\\\
\textbf{Nomenklatur}
\begin{itemize}
	\item Fitnessfunktion $\triangleq$ zu optimierende Bewertungsfunktion
	\item Individuum oder Hypothese $\triangleq$ Einzelne Lösung
	\item Population und Generation $\triangleq$ Lösungsmenge
	\item Erzeugung von Nachkommen $\triangleq$ Generierung neuer Lösungen
	\item Genetische Operationen $\triangleq$ Rekombination, Mutation
	\item Veränderter Nachfolger, Kind, Nachkomme $\triangleq$ neue Lösung
	\item Selektion der Besten $\triangleq$ Auswahl der Lösungen, welche die beste Optimierung erzeugen
\end{itemize}

\subsubsection{Grundalgorithmus}
\includegraphics[width=.7\textwidth]{figures/grundalgorithmus.png}

\subsubsection{Generierung von Nachkommen}
\begin{align*}
	\text{Exploration} & \Leftrightarrow \text{Exploitation} \\
	\text{Untersuchen des Raumes} & \Leftrightarrow \text{lokale Optimierung}
\end{align*}

\begin{itemize}
	\item Je stärker und zufälliger Änderungen sind, desto geringer ist die Wahrscheinlichkeit, einen besseren Nachkommen zu erzeugen.
	\item Bei lokalen Verbesserungsmethoden ist die Gefahr der lokalen Minima gegeben.
	\item Explorationsfaktor muss gemäß der aktuellen Fitness der Generation ausgewählt werden.
\end{itemize}
%
\textbf{Kodierung der Individuen}
\begin{itemize}
	\item Wissen/Information muss strukturiert repräsentiert werden
	\begin{itemize}
		\item Wie viel von dieser Strukturinformation soll genutzt werden?
		\begin{itemize}
			\item kein Einsatz, ausschließlich Anwenden der Algorithmen auf den Binärsequenzen
			\item volle Ausnutzung der Strukturinformation $\Rightarrow$ Spezielles Zuschneiden des Optimierungsalgorithmus
		\end{itemize}
	\end{itemize}
	\item Kodieren der Gene
	\begin{itemize}
		\item Binärcodierung $\Leftarrow$ Genetische Algorithmen
		\item Reelle Zahlen, Vektoren $\Leftarrow$ Evolutionäre Strategien (Beispiel: $1010|1110|1111|1010|1000|1010$)
	\end{itemize}
\end{itemize}
%
\textbf{Mutation:} Der Nachkomme stammt von einem Elternteil ab
\begin{itemize}
	\item Mutation einzelner Bits: Bitinversion
	\begin{align*}
		\text{Elternteil: } & 1010110 \\
		\text{Nachkomme: } & 1011110
	\end{align*}
	\item Konzepte:
	\begin{itemize}
		\item Alle Bits einer Sequenz werden unabhängig voneinander mit einer bestimmten Wahrscheinlichkeit invertiert
		\item Für eine bestimmte (oder zufällige) Anzahl von Bits werden die Indizes zufällig ausgewählt
		\item Stochastisch bei kontinuierlicher Repräsentation
		\begin{equation*}
			x_i := x_i + z
		\end{equation*}
		wobei $z$ eine Zufallsvariable  (z.B.\ Nach $N(0,\sigma)$ Normalverteilung um $0$ mit Varianz $\sigma$		
	\end{itemize}
	\item Mutationsoperator bei Sequenzen
	\begin{itemize}
		\item Herausnehmen einer Teilsequenz und Einfügen an anderer Stelle
		\item Invertiertes Einfügen der Teilsequenz
		\item Spezielle Mutationsoperationen  anwendungsspezifisch
	\end{itemize}
\end{itemize}
%
\textbf{Rekombination:} Der Nachkomme stammt von zwei oder mehr Eltern ab
\begin{itemize}
	\item Diskrete Rekombination
	\begin{center}
	\includegraphics[width=.6\textwidth]{figures/diskrete_rekombination.png}
	\end{center}
	\item Intermediäre Rekombination: Sei $x := \text{Elternteil}_1$ und $y := \text{Elternteil}_2$, dann ist Nachkomme $z$ definiert durch
	\begin{equation*}
		z_i := (x_i + y_i) / 2
	\end{equation*}
	\item Crossover
	\begin{center}
	\includegraphics[width=.5\textwidth]{figures/crossover.png}	
	\end{center}
\end{itemize}

\subsubsection{Selektion}
\textbf{Herausforderung einer guten Selektion}
\begin{itemize}
	\item Selektion:
	\begin{itemize}
		\item der Population in jeder Iteration
		\item der Eltern für jeweilige Erzeugung von Nachkommen (Mating)
	\end{itemize}
	\item Problem:
	\begin{itemize}
		\item Genetische Drift: Individuen vermehren sich zufällig mehr als andere
		\item Crowding, Ausreißerproblem: fitte Individuen und ähnliche Nachkommen dominieren die Population
		\item Vielfalt der Population wird eingeschränkt
		\item Entwicklung der Individuen (Konvergenz) wird verlangsamt
	\end{itemize}
	\item Lösung:
	\begin{itemize}
		\item Unterschiedliche Populationsmodelle und Selektionsmethoden
		\item Populationsgröße optimieren
	\end{itemize}
\end{itemize}
%
\textbf{Populationsmodelle}
\begin{itemize}
	\item Eine einfache Menge: die global Besten entwickeln sich rasch weiter, andere Entwicklungslinien werden unterdrückt
	\item Inselmodell: die Evolution läuft weitgehend getrennt, nur manchmal werden Individuen ausgetauscht
	\item Nachbarschaftsmodell: Nachkommen dürfen nur von Individuen erzeugt werden, die in ihrer Nachbarschaft die beste Fitness besitzen
	\item Nomenklatur
	\begin{itemize}
		\item $\lambda$ Anzahl von Nachkommen
		\item $\mu$ Populationsgröße
		\item $f$ Fitness-Funktion
	\end{itemize}
\end{itemize}
%
\textbf{Populationsmitglieder}
\begin{itemize}
	\item Populationsgrößer
	\begin{itemize}
		\item Soll sie konstant bleiben? $\mu$
		\item Wie viele neu erzeugte Nachkommen? $\lambda$
	\end{itemize}
	\item Mitgliederselektion: stochastisch ausgewählt $\Rightarrow$ die besten $\mu$ Individuen
	\begin{itemize}
		\item $(\mu, \lambda)$ Strategie: Auswahl bezieht sich nur auf die Nachkommen (bessere Exploration)
		\item $(\mu + \lambda)$ Strategie: Auswahl zieht auch Eltern mit ein (die Besten werden berücksichtigt, geeignet für gut berechenbare Fitness)
	\end{itemize}
	\item Ersetzungsregel für Mitglieder:
	\begin{itemize}
		\item Nachkommen ersetzen alle Eltern (Generationen-Modus)
		\item Nachkommen ersetzen einen Teil der Eltern
		\item Nachkommen ersetzen die ihnen am ähnlichsten Eltern
		\item Geographische Ersetzung
		\item Bestes Elternindividuum überlebt (Elitist-Modus)
	\end{itemize}
	\item Daumenregel: Das beste Viertel der Population sollte drei Viertel der Nachkommen erzeugen
\end{itemize}
%
\textbf{Selektionsmethoden}
\begin{itemize}
	\item Fitness Based Selection
	\begin{equation*}
		P(x) \approx \frac{f(x)}{\sum_{x' \in Pop} f(x')} \text{ mit}
	\end{equation*}
	\begin{align*}
		P(x): & \text{Wahrscheinlichkeit der Auswahl von Individuum x} \\
		\lambda: & \text{Anzahl von Nachkommen} \\
		\mu: & \text{Populationsgröße} \\
		f: & \text{Fitness-Funktion}
	\end{align*}
	abhängig von der Fitnessfunktion z.B.\ im Laufe der Evolution nur noch geringe Änderungen in $f(x)$ und damit in $P(x)$
	\item Ranking Based Selection
	\begin{equation*}
		P(x) \approx \frac{g(r(x))}{\sum_{x' \in Pop} g(r(x'))} \text{ mit}
	\end{equation*}
	\begin{align*}
		P(x): & \text{Wahrscheinlichkeit der Auswahl von Individuum x} \\
		r(x): & \text{Ranking von x in der aktuellen Population gemäß Fitnessfunktion} \\
		g: & \text{mit der Güte des Ranges monton steigende Funktion größer 0}
	\end{align*}
	\begin{itemize}
		\item Exponentiell: $g(x) = a^{-x}$
		\item Hyperbolisch: $g(x) = x^{-a}$
		\item Die besten k: $g(x) := 
		\begin{cases}
			\frac{1}{k} &, x \leq k \\
			0 &, sonst
		\end{cases}$
	\end{itemize}
	weniger abhängig von dem Betrag der Fitness
	\item Tournament Selection (Tunier)
	\begin{itemize}
		\item wähle für jedes zu erzeugende Individuum $n (=2)$ Individuen
		\item belohne davon, gemäß der Fitness das beste Individuum
		\item wähle Individuen mit höchster Bewertung
	\end{itemize}
	wenig abhängig von dem Betrag der Fitness; Wahl der Selektionsmethode oft anwendungsspezifisch
\end{itemize}
%
\textbf{Evolutionstheorien in der Biologie}
\begin{itemize}
	\item Lamarksche Evolution
	\begin{itemize}
		\item Individuen ändern sich (lernen) nach der Erzeugung
		\item Genotyp wird verändert und anschließend vererbt
	\end{itemize}
	\item Baldwinsche Evolution
	\begin{itemize}
		\item Individuen ändern sich (lernen) nach der Erzeugung
		\item Genotyp bleibt unverändert
	\end{itemize}
	\item Hybride Verfahren
	\begin{itemize}
		\item es gibt veränderbare und fixe Phänotypen
	\end{itemize}
	\item Anwendung: Suche nach optimalen neuronalen Netzen
\end{itemize}

\subsubsection{Anwendungen}
\begin{itemize}
	\item Cybermotten: Wer sich tarnt, wird nicht gefressen
	\begin{itemize}
		\item Konstante Population 200
		\item Fitness -- Zeit in der die Motten vom Vogel entdeckt werden
		\item Rekombination und Mutation (auf dem Graycode der Grauwertpixel)
		\item Eltern erzeugen mehrere Nachkommen
		\item Eltern überleben nur eine Generation
	\end{itemize}
	\begin{center}
	\includegraphics[width=.5\textwidth]{figures/cybermotten.png}
	\end{center}
	\item Optionale Steuerung von Laufmaschinen
	\begin{itemize}
		\item Unterschiedliche Morphologie
		\item Beispiel -- Erzeugung von Beinsteuerung
		\item Steuerungsparameter
		\begin{itemize}
			\item Gelenknummer: integer $[0;3]$
			\item Startzeitpunkt: real $[0;t]$
			\item Dauer: real $[0;t]$
			\item Kräfte: real $[-max;max]$
		\end{itemize}
		\item Individuum = Sequenz von Instruktionen variabler Länge
		\item Mutation
		\begin{itemize}
			\item Micro-Mutation (Veränderung einzelner Werte einer Instruktion)
			\item Macro-Mutation (Austausch ganzer Instruktionsblöcke)
		\end{itemize}
		\item Operatoren (Anpassung)
		\begin{itemize}
			\item Crossover
			\item Micro-Crossover (Änderung eines Parameters einer Instruktion)
			\item Macro-Crossover (Austausch einer Instruktion)
			\item Homologes-Crossover (Austausch von Instruktionen eines speziellen Gelenks)
		\end{itemize}
		\item Fitness = Abweichung von einer vorgegebenen Trajektorie
	\end{itemize}
	\item AIBO -- Autonomous Evolution of Dynamic Gaits
	\begin{itemize}
		\item Ähnlicher Ansatz 
		\item Fitness = Mittelwert aus drei Läufern; Funktion aus Richtung und Geschwindigkeit
	\end{itemize}
	\item Snakebot
	\begin{itemize}
		\item Biologische Inspiration: Seidenwinder Klapperschlange: $5 km/h$, und die schnellste Schlange, die Schwarze Mamba: bis $16 km/h$
		\item Individuum = Steuerung
		\item Fitness = Geschwindigkeit in Fortbewegung und Klettern, Befreien aus Hindernissen etc.\
		\item Adaption: ein oder zwei Gelenke unbeweglich
		\item Hinderniserkennung
	\end{itemize}
\end{itemize}

\subsection{Künstliche Ontogenese: Entwicklung von Agenten durch künstliche Zellen}
\textbf{Ontogenese:}
struktureller Wandel einer Einheit ohne Verlust ihrer Organisation, Entwicklung des einzelnen Lebenswesens von der befruchteten Eizelle zu erwachsenen Lebewesen.\\
\textbf{Ziel:}
Entwicklung von Agenten durch künstliche Zellen\\\\
%
\textbf{Methode}
\begin{itemize}
	\item Repräsentation einer Zelle durch Kugel
	\item Operatoren 
	\begin{itemize}
		\item Wachsen der Kugel im Durchmesser
		\item bei Erreichen der maximalen Größen -- Teilen in zwei Kugeln, die durch Rotationsgelenk oder ein starres Element verbunden sind
	\end{itemize}
	\item Jede Kugel besitzt ein neuronales Netz mit Motoneuronen, Interneuronen und Perzeptionsneuronen
	\item Kodierung als Gleitkomma Genom
\end{itemize}
%
\textbf{Genobots} -- Automatisches Modulares Design
\begin{itemize}
	\item Individuen = Regeln und Algorithmen für die Erzeugung (Generative Repräsentation)
\end{itemize}
%
\textbf{Golem} -- life as it could be
\begin{itemize}
	\item Individuum = Repräsentation der Morphologie
	\item Roboter: Stangen (Länge, D, Steifigkeit), Aktuatoren, Gelenktypen, Neuronen (threshold, synaptische Verbindung)
	\item Beschränkungen 
	\begin{itemize}
		\item Komplexe Strukturen (wie Muskeln) nicht möglich
		\item Design Beschränkung
		\item Evaluation (Fitness)
		\item Übertragbarkeit in die Realität
	\end{itemize}
\end{itemize}

\subsection{Selbstkonfigurierende Systeme -- Mutierende Roboter}
siehe ab Folie 73
\subsubsection{Kurzer Überblick (AIST, Japan) -- Selbstkonfiguration = Evolution? Mutation?}
\subsubsection{Reales 2D System}
\subsubsection{3D Selbstkonfigurierendes Robotersystem}


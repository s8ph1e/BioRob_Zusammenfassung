\emph{Kaum etwas sieht so einfach aus und ist im Detail doch so komplex}
\subsection{Lokomotionsarten}
\emph{Lokomotion beschreibt die aktive Fortbewegung eines Lebewesens}
\begin{itemize}
\item \textbf{Kriechen}: Fortbewegungsart, bei der die Körperunterseite immer Bodenkontakt hat\\
Natürliche Vertreter:
(beinlose) Vertebrate (Schlangen, Reptilien, Amphibien,...)
\item \textbf{Hüpfen}: Fortbewegungsart mit vollständiger Flugphase (keinerlei Bodenkontakt), typischerweise ergibt sich eine ballistische Flugbahn, lange Flugphase\\
Natürliche Vertreter: (eher selten) z.B. Känguru, Frosch
\item \textbf{Laufen}: Fortbewegungsart mit wiederkehrendem Wechsel des Bodenkontakts, Unterscheidung zwischen \glqq Gehen\grqq{} und \glqq Rennen\grqq, Unterschiedliche
Laufmuster (je nach Anzahl der Beine)\\
Natürliche Vertreter: Invertebrate, Vertebrate, eigentlich alle
\end{itemize}
\subsection{Stabilität beim Laufen}
\begin{itemize}
\item Laufroboter müssen in der Regel AKTIV stabilisiert werden
\begin{itemize}
\item Nicht zu viele Beine anheben
\item Körperschwerpunkt ausregeln
\end{itemize}
\item Stabilität wird maßgeblich von Laufmuster beeinflusst
\item Unterscheidung: Statische Stabilität vs. Dynamische Stabilität
\item Faustregel: Je mehr Beine ein Roboter hat, desto einfacher ist es diesen stabil zu halten
\end{itemize}
\subsubsection{Statische Stabilität}
\begin{itemize}
\item Laufbewegung kann zu JEDEM Zeitpunkt angehalten werden $\rightarrow$ System bleibt stehen
(z.B. ein rennender Gepard ist also NICHT statisch stabil)
\item Mindestens 3 Beine haben Bodenkontakt
\item Beine mit Bodenkontakt bilden Aufstandspolygon (stark abhängig von der Geometrie des Roboters, z.B. lange Beine = großes AP)
\item Projizierter Schwerpunkt (CoM = Center of Mass) muss innerhalb des Aufstandspolygons liegen $\rightarrow$ 6-beinige sind statisch stabil
\item Stabilitätsmaße definieren Wert für Stabilität:
\begin{itemize}
\item \textbf{Stability Margin}: Minimaler Abstand zum Support-Polygon Rand (vgl. \autoref{StabMarg})
\item \textbf{Longitudinal Stability Margin}: Minimaler Abstand zum Rand in Laufrichtung (vgl. \autoref{LongStabMarg})
\begin{figure}[h!]
	\centering
	\begin{subfigure}{.3\textwidth}
		\includegraphics[width=\textwidth]{figures/ch06_supppoly.png}
		\caption{Support Polygon}
		\label{SP}
	\end{subfigure}
	\begin{subfigure}{.3\textwidth}
		\includegraphics[width=\textwidth]{figures/ch06_stabmarg.png}
		\caption{Stability Margin}
		\label{StabMarg}
	\end{subfigure}
	\begin{subfigure}{.3\textwidth}
		\includegraphics[width=\textwidth]{figures/ch06_longstabmarg.png}
		\caption{Longitudinal Stability Margin}
		\label{LongStabMarg}
	\end{subfigure}
	\caption{Stabilitätsmaße}
\end{figure}
\end{itemize} 
\end{itemize}
\subsubsection{Dynamische Stabilität}
\begin{itemize}
\item Zyklische Bewegung ohne umzufallen
\item Deutlich komplexere Stabilität, da dynamische Effekte (Geschwindigkeiten, Trägheitsmomente) berücksichtigt werden
\item Bei fortschreitender Bewegung bleibt das System aufrecht/stabil
\item Zahlreiche (komplexe) Stabilitätskriterien:
\begin{itemize}
\item Center of pressure method (EMC)
\item Dynamic stability margin (DSM)
\item Tumble stability margin (TSM)
\item Force-angle stability margin (FASM): \textit{A new measure of tipover stability margin for mobile manipulators}, Papadopoulos1996
\item Zero Moment Point (ZMP): \textit{Zero-moment point -- Thirty five years of its life}, Vukobratovic2005
\item Normalized dynamic energy stability margin (NDESM)
\end{itemize}
\end{itemize}
\subsubsection{Stabilitätskriterien}
\textbf{Vukobratovic - Zero Moment Point (ZMP)}
\begin{itemize}
\item Punkt, in dem sich, wenn er auf den Boden projiziert wird, alle Kräfte auslöschen
\begin{itemize}
\item[$\rightarrow$] anschaulich: ZMP $\approx$ \glqq Drehpunkt\grqq{} auf dem Boden
\item[$\rightarrow$] Wenn der ZMP innerhalb des Aufstandspolygons / Fußfläche liegt ist der Roboter stabil
\item[$\rightarrow$] Fußgröße ist hier wichtig; Bodenkontaktkräfte verteilen sich anteilig über den gesamten Fuß \footnote{Daher ist es so beachtlich, dass \textit{Pacman} von Boston Dynamics normale Menschen-Turnschuhe trägt; viele 2-beinige Roboter haben sehr große Füße.}
\end{itemize}
\item flächiger Bodenkontakt: alle punktuellen Kontakte können zu einem zentralen Kontaktpunkt zusammengefasst werden, gewichtet anhand der Abstände und der Kräfte
\begin{itemize}
\item $\mathbf{p}_i$: Gewichte der $yxz$-Position auf der Ebene
\item $\mathbf{f}_i$: Kraftvektoren (ergeben sich aus Sensorwerten, normal jeweils vier Kraftsensoren vorne und hinten auf Fußfläche)
\end{itemize}
\item Seien $\mathbf{p}_i$ die Fußpunkte, mit den Kräften $f_i = [f_{ix}, f_{iy}, f_{iz}]$, dann ist der ZMP definiert als\\
\begin{center}
$\mathbf{p} := \frac{\sum_{i=1}^N \mathbf{p}_if_{iz}}{\sum_{i=1}^N \mathbf{p}_if_{iz}}$
\end{center}
\item Drehmomente berechenbar durch \\
\begin{center}
$\mathbf{\tau} = \sum_{i=1}^N (\mathbf{p}_i - \mathbf{p}) \times \mathbf{f}_i$
\end{center}
\item Anhand der Definitionen werden die Drehmomente in $x$- und $y$-Achse jeweis Null: $\mathbf{\tau}_x = \mathbf{\tau}_y = 0$ [Zero Moment Point]
\begin{align*}
\mathbf{\tau}_x &= \sum\limits_{i=1}^N(p_{iy}-p_y)f_{iz} - \sum\limits_{i=1}^N(p_{iz}-p_z)f_{iy} &\text{zweiter Term wird 0}\\
\mathbf{\tau}_y &= \sum\limits_{i=1}^N(p_{iz}-p_z)f_{ix} - \sum\limits_{i=1}^N(p_{ix}-p_x)f_{iz} &\text{erster Term wird 0}\\
\mathbf{\tau}_x &= \sum\limits_{i=1}^N(p_{ix}-p_x)f_{iy} - \sum\limits_{i=1}^N(p_{iy}-p_y)f_{ix}\\
\end{align*}
\item Im statischen Fall kann der Schwerpunkt einfach auf den Boden projiziert werden
\item Über Roboterdynamik kann ein virtueller / computed ZMP berechnet werden
\item ZMP bereits 1968 von Vukobratovic in Moskau präsentiert
\item ZMP macht recht starke Annahme: finde den Punkt, an dem sich alle Kräfte  ausgleichen $\rightarrow$ Kriterium funktioniert im Grunde nur auf flachem Boden, in unebenem Gelände nicht so gut anwendbar
\item Dennoch weitverbreitetes dynamisches Stabilitätskriterium insbesondere für zweibeinige Laufroboter und weltweit angewandtes Konzept in der humanoiden Robotik (z.B. laufender \textit{ASIMO} von Honda beweist, dass ZMP ein gutes Kriterium ist)
\end{itemize}
\textbf{Force-Angle Stability Measure (FASM)}
\begin{itemize}
\item Ursprünglich für Baufahrzeuge entwickelt
\item Winkel zur Kippachse als Stabilitätsmaß\\
\begin{align*}
\mathbf{x}_{cog} &= \frac{1}{\sum_{i=1}^N M_i} \sum\limits_{n=0}^N m_i \cdot \mathbf{x}_{cog}\\
\alpha &= min(\theta_i)||\mathbf{f}_r||
\end{align*}
\begin{itemize}
\item Wenn Winkel $\theta = 0$ wird kippt man um
\item Berücksichtigt die Höhe
\item Kann die externe Störkräfte berücksichtigen
\item schnell und effektiv berechenbar
\item Eingangsgrößen:CoM, Gewichtskraft auf CoM, externe Kräfte und Momente (auf CoM), Fußpunkte
\end{itemize}
\end{itemize}
\subsection{Laufmuster}
\textbf{Gait-Diagramme} (siehe z.B. \autoref{symmetry})
\begin{itemize}
\item Zur übersichtlichen Darstellung von Laufmustern
\item Weit verbreitet: sowohl in Biologie wie Robotik
\item $X$-Achse: zeitlicher Verlauf
\item $Y$-Achse (pro Bein): Darstellung der Schwing- bzw. Stemmphase\\
(meist: schwarzer Balken = Bodenkontakt/Stemmphase)
\item Laufmuster aus Gait-Diagramm \glqq ablesbar\grqq
\end{itemize}
\textbf{Phase-Shift}
\begin{itemize}
\item Phase-Shift des $Bein_i$ (Phasenverschiebung):
Laufmuster können über Phasenverschiebung zwischen den Beinen beschrieben/definiert werden\\
\begin{center}
$\phi_i = \frac{Beginn Stemmphase Bein_i}{T}$
\end{center}
\end{itemize}
\textbf{Duty-Factor}
\begin{itemize}
\item Duty-Factor des $Bein_i$ (Belastungsfaktor):
Beschreibt den relativen Anteil der Stemmphase zu der Gesamtzykluszeit ($T = Schwingzeit + Stemmzeit$)\\
\begin{center}
$\beta_i = \frac{Stemmzeit Bein_i}{T}$
\end{center}
\end{itemize}
\begin{figure}[h!]
	\centering
	\includegraphics[width=0.7\textwidth]{figures/ch06_PS_DF.png}
	\caption{Phase-Shift und Duty-Factor}
	\label{PS_DF}
\end{figure}
\subsubsection{Unterscheidungskriterien für Laufmuster}
Laufmuster haben wesentlichen Einfluss auf die Leistungsfähigkeit von Laufrobotern:
\begin{itemize}
\item Stabilität
\item Geschwindigkeit
\item Effizienz
\end{itemize}
Wenn man über Laufmuster spricht sind de folgenden Kriterien in dieser Reihenfolge anzugeben:
\begin{itemize}
\item[1.] \textbf{Anzahl der Beine}: vgl. \autoref{Laufarten}
\begin{figure}[h!]
	\centering
	\includegraphics[width=0.7\textwidth]{figures/ch06_beinanzahl.png}
	\caption{Laufarten}
	\label{Laufarten}
\end{figure}
\item[2.] \textbf{Symmetrie der Beinpaare}: Wie ist die Phasenverschiebung innerhalb eines Beinpaares?\\(vgl. \autoref{symmetry})
\begin{figure}[h!]
	\centering
	\includegraphics[width=0.5\textwidth]{figures/ch06_symmetry.png}
	\caption{Symmetrie der Beinpaare: Beim asymmetrischen Gang schwingen RH und LH in Phase (gleichzeitig), während beim symmetrischen Gang eine klare Phasenverschiebung von 0.5 auftritt.}
	\label{symmetry}
\end{figure}
\item[3.] \textbf{Zyklisches Laufmuster}: Ist ein wiederkehrendes Muster erkennbar? (vgl. \autoref{pattern})
\begin{figure}[h!]
	\centering
	\includegraphics[width=0.5\textwidth]{figures/ch06_pattern.png}
	\caption{Zyklen}
	\label{pattern}
\end{figure}
\item[] Bemerkung: Wie unterscheiden sich Kriterien 2 und 3?
\begin{itemize}
\item Kriterium 2 bezieht sich auf die Phasenverschiebung innerhalb der Beinpaare (z.B. vordere Beine: LF, RF);  symmetrisch: 0.5, ansonsten asymmetrisch
\item Bei Kriterium 3 bleiben die Phasenverschiebung zwischen den Beinen und der Duty-Factor konstant; Muster (kann aus mehreren Schrittfolgen bestehen) bleibt gleich
\end{itemize}
\item[4.] \textbf{Anzahl der Bodenkontakte/Beine auf dem Boden}:
\begin{itemize}
\item Laufmuster wird über die Mindestzahl an Beinen mit Bodenkontakt definiert
\item Mehr Beine im Kontakt mit Boden $\rightarrow$ größere Stabilität
\item Meist handelt es sich um zyklische Laufmuster
\item Typische Beispiele sind: Tripod (immer mind. 3 Beine auf dem Boden), Tetrapod (immer mind. 4 Beine auf dem Boden), Pentapod (immer mind. 5 Beine auf dem Boden)
\end{itemize}
\end{itemize}
\subsubsection{Sechsbeinige, vierbeinige, zweibeinige Laufmuster}
\textbf{Hexapod/6-Beiner}:
\begin{itemize}
\item Schnellstes statisch stabiles Laufmuster: Tripod (abwechselnd 3 Beine auf Boden, zu jedem Zeitpunkt statisch stabil), siehe \autoref{hexapod} A
\item Weitere typische Laufmuster: Tetrapod, Pentapod (4 bzw. 5 Beine auf dem Boden, zu jedem Zeitpunkt statisch stabil), siehe \autoref{hexapod} B
\item Cruse-Regel basiertes Laufmuster (Mix aus den obigen), siehe \autoref{hexapod} Mischung aus A und B
\item Free-Gait -- \glqq Freies Laufmuster\grqq{}: große Flexibilität, besonders für schwieriges Terrain gut geeignet, siehe \autoref{hexapod} C
\begin{figure}[h!]
	\centering
	\includegraphics[width=0.3\textwidth]{figures/ch06_hexapod.png}
	\caption{Gangarten 6-beiniger Insekten}
	\label{hexapod}
\end{figure}
\end{itemize}
\textbf{Free Gait am Beispiel der Stabheuschrecke}
Die Verwendung der Begriffe Tetrapod und Tripod verleitet zu der Annahme der Existenz fester Gangmustern.
Es ist aber bekannt, dass Insekten einen freien Gang verwenden. Die Koordination der Beine erfolgt durch mehrere Koordinationsregeln (vgl. \autoref{LM und CR}). Bei der Stabheuschrecke 
gilt die einfache Regel: \textit{Wenn der linke und rechte Nachbar auf dem Boden sind, dann darf das Bein anfangen zu schwingen} (vgl. \autoref{stabi}). 
\begin{figure}[h!]
	\centering
	\includegraphics[width=0.65\textwidth]{figures/ch06_stabi.png}
	\caption{Gang der Stabheuschrecke. Die Konstellation im dritten Bild ist durch die Cruse-Regeln eigentlich nicht erlaubt.}
	\label{stabi}
\end{figure}
Es ist ein Wechsel/Übergang zwischen verschiedenen Laufmustern möglich:\\
\begin{center}
Tripod: $\beta \approx 0.65$, Tetrapod: $\beta \approx 0.80$, Pentapod: $\beta \approx 0.9$\\
($\beta$ := Schwing- zu Stemmzeit Verhältnis)
\end{center}
\textbf{Quadruped/4-Beiner}:
\begin{itemize}
\item Zahlreiche, sehr unterschiedliche Laufmuster
\item Laufende Muster (Walking Gaits)
\begin{itemize}
\item Mind. 1 Bein immer auf dem Boden
\item Walk, Passgang (Amble)
\item Trott - Walking
\item Creeping Gait (immer nur 1 Bein schwingt)
\item Crawl Gait (Creeping Gait mit max. Stabilität)
\end{itemize}
\item Rennende Muster (Running Gaits)
\begin{itemize}
\item Flugphase vorhanden (kein Bein auf Boden)
\item Passgang (Pace)
\item Leichter Galopp (Canter)
\item Galopp (Gallop)
\item Trab (Trot - Running)
\end{itemize}
\end{itemize}
\textbf{Zweibeinige Laufmuster/Menschliches Laufen}:
\begin{table}[h!]
\centering
\begin{tabular}{|p{5cm}|p{5cm}|}
\hline
\textbf{\glqq Laufen\grqq{} (run)} & \textbf{\glqq Gehen\grqq{} (walk)} \\
\hline
\hline
\begin{itemize}
\item Schneller als \glqq Gehen\grqq{}
\item Mit Flugphase
\item Duty-Factor $<= 0.5$
\end{itemize} &
\begin{itemize}
\item Langsamer \glqq Laufen\grqq{}
\item Keine Flugphase
\item Duty-Factor $> 0.5$
\end{itemize}\\
\hline
\end{tabular}
\caption{Wichtigste Unterscheidung: \glqq Laufen\grqq{} vs. \glqq Gehen\grqq}
\label{tab:2Beiner}
\end{table}
\subsection{Mehrbeiniges Laufen}
\subsubsection{Zentrale Muster Generatoren (CPG)}
\begin{itemize}
\item Bio-inspirierte Mechanik und bio-inspirierte Steuerung (CPGs)
\item Vier gekoppelte Oszillatoren erzeugen unterschiedliche Laufmuster (vgl. \autoref{cpg})
\end{itemize}
\begin{figure}[h!]
	\centering
	\begin{subfigure}{.6\textwidth}
		\includegraphics[width=\textwidth]{figures/ch06_cpg1.png}
		\caption{}
		\label{cpg1}
	\end{subfigure}
	\begin{subfigure}{.6\textwidth}
		\includegraphics[width=\textwidth]{figures/ch06_cpg2.png}
		\caption{}
		\label{cpg2}
	\end{subfigure}
	\caption{Beispiel CPGs}
	\label{cpg}
\end{figure}
%\subsubsection{Reaktive Ansätze}
\subsubsection{Verhaltens- und regelbasierte Steuerung}
\textbf{Beispiel: LAURON 4/5}
\begin{itemize}
\item Modulare Architektur: Wart- und erweiterbar
\item Verschiedene Laufmuster: Tripod, Tetrapod, Pentapod sowie Free-Gait
\item Schrittgrößen und Zykluszeiten werden automatisch angepasst (z.B. bei Gang über Rampe)
\item Hierarchische Verhaltsarchitektur
\begin{itemize}
\item Koordinierte Verhalten (langsam)
\item Reaktive Verhalten (schnell)
\item Hardware Abstraction Layer (schnell)
\end{itemize}
\end{itemize}
\textbf{Passive Walking}:\\
Rein mechanisches Laufen (kein Strom etc.) durch simple Regelungsalgorithmen und intelligente Mechanik (z.B. \textit{Cornell Biped})
\subsection{Zweibeiniges Laufen}
\subsubsection{Biomechanische Ansätze menschliches Laufen zu beschreiben}
\textbf{Selbststabilisierende Systeme beim Laufen}
\begin{itemize}
\item Informationsfluss auf Nervenbahnen deutlich kleiner als Schallgeschwindigkeit\\
$\Rightarrow$ Säugetiere tasten etwa mit $50ms - 100ms$ ab.
\item Trotzdem kann sich z.B. ein Gepard mit über $80 km/h$ bewegen,
das bedeutet, dass er sich pro \glqq Abtastschritt\grqq{} etwa $1-2 m$ bewegt.\\
$\Rightarrow$ Informations-Vorverarbeitung in der \glqq Mechanik\grqq{}
\end{itemize}
Daraus lassen sich biomechanisch folgende Schlüsse ziehen:
\begin{itemize}
\item[a.] Bei schneller Lokomotion kann das Muskel-Skelett-System den Schwerpunkt nicht mehr kontinuierlich stabilisieren.
\item[b.] Die \glqq Bein-Mechanik\grqq{} muss so eingestellt werden, dass ein so genannter \textit{Selbststabilisierender Zyklus} eingestellt wird.
\item[c.] Das Nervensystem stellt die \glqq Mechanik\grqq{} anhand eines \textit{kinematischen Programms} immer wieder neu ein (Reflex-Preflex)
\end{itemize}
Beispiele:
\begin{itemize}
\item detailliertes neuronales Muskel-Skelett-Modell nach Hatze
\item verallgemeinertes mechanisches Modell nach Günther
\item \glqq Virtual Model Control\grqq{} nach Henze bzw. Pratt
\item Feder-Masse-System nach Blickhan
\item Erweiterte Untersuchungen des Feder-Masse-Systems nach Seyfarth/Geyer/Blickhan
\end{itemize}
\noindent
\textbf{Bewegungsbereich und Selbststabilisation}\\
Bewegungsbereich und Selbststabilisation sind beschreibende Attribute der energetischen Topographie (siehe Abbildungen \ref{bew_stab} und \ref{stab}).
\begin{figure}[h!]
	\centering
	\includegraphics[width=0.4\textwidth]{figures/ch06_bew_stab.png}
	\caption{Bewegungsbereich und Selbststabilisation}
	\label{bew_stab}
\end{figure}
\begin{figure}[h!]
	\centering
	\includegraphics[width=0.6\textwidth]{figures/ch06_selbststab.png}
	\caption{Selbststabilität}
	\label{stab}
\end{figure}\\
\bigskip
\textbf{Feder-Masse-System nach Blickhan}\\
\begin{figure}[h!]
	\centering
	\includegraphics[width=0.4\textwidth]{figures/ch06_blickhan1.png}
	\caption{Starke Vereinfachung als Punktmasse an Feder}
	\label{blick}
\end{figure}
Modellierung selbststabilisierender Systeme:  
\begin{itemize}
\item Modellierung menschlichen Laufens durch sehr einfache physikalische Modelle; Modellierung nähert das Feder-Dämpfer Verhalten
der biologischen Muskeln an
\item Abstraktion des gesamten muskulären Apparates auf eine Feder
\item Schwerpunkt bewegt sich auf sinoider Bahn. Bein \glqq federt ein\grqq{} beim Auffußen und \glqq federt aus\grqq{} beim Abfußen
\end{itemize}
\autoref{blick2} zeigt die Modellierung, hierbei ist:
\begin{itemize}
	\item $l_0$: Initiale Beinlänge (Beim Auffußen und Abfußen)
	\item $\alpha_0$: \textit{Angle of Attack} (Aufsetzwinkel)
	\item $m$: Systemmasse
	\item $k_{leg}$: Federkonstante des Beines
	\item $y_i$: Apexhöhe (max. Schulterhöhe)
	\item $v_{x,0}$: Geschwindigkeit in $x$-Richtung
	\end{itemize}
\begin{figure}[h!]
	\centering
	\includegraphics[width=0.4\textwidth]{figures/ch06_blickhan2.png}
	\caption{Modellierung Selbststabilisierender Systeme}
	\label{blick2}
\end{figure}
\autoref{mf} zeigt ein attraktives Masse-Federsystem. Dieses Modell kann 1:1 auf den Menschen übertragen werden. In der rechten Abbildung sieht man den stabilen Zyklus. Hierbei ist der Aufsetzwinkel $\alpha$ als aktiv zu regelnde Stellgröße besonders wichtig. 69° sind zu groß und der stabile Zyklus wird verlassen. 
\begin{figure}[h!]
	\centering
	\includegraphics[width=0.4\textwidth]{figures/ch06_masse-feder.png}
	\caption{Attraktives Masse-Feder-System}
	\label{mf}
\end{figure}
\autoref{al} zeigt den Ablauf. Hierbei ist:
\begin{itemize}
\item Konstante Geschwindigkeit $\omega_R$
\item Start am Apex mit Angriffs-winkel $\alpha_R$
\item Aufsetzwinkel $\alpha_0$
\item Anpassen von $\alpha$ als Funktion über die Flugzeit: $\alpha=\alpha(t)$
\item Konstante Steifheit $k$ im Bein
\item[$\Rightarrow$] Verschiedene \glqq Kinematische Programme\grqq{} im ZNS
\end{itemize}
\begin{figure}[h!]
	\centering
	\includegraphics[width=0.4\textwidth]{figures/ch06_ablauf.png}
	\caption{Selbststabilisierende Systeme -- Ablauf}
	\label{al}
\end{figure}
\noindent
\textbf{Rennen und Gehen}\\
In \autoref{er} ist elastisches Rennen dargestellt. Hierbei zeigt die linke Abbildung das Kraftprofil mit Laufrichtung als $x$-Komponente und Kraft als $z$-Komponente. Stabile Winkel sind 67° und 68°, nicht aber 66° und 69°. Bei letzteren kommt es zum Aufschwingen bzw. Abbruch.
\begin{figure}[h!]
	\centering
	\includegraphics[width=0.8\textwidth]{figures/ch06_rennen.png}
	\caption{Elastisches Rennen}
	\label{er}
\end{figure}\\
\autoref{rg} zeigt Unterschiede zwischen Rennen und Gehen:
\begin{itemize}
\item Verhältnis Schwingen/Stemmen $<0.5 \leftrightarrow >0.5$
\item single-support $\leftrightarrow$ double-support
\item Einhügliches $\leftrightarrow$ zweihügliches Kraftmuster
\item Kurzer $\leftrightarrow$ langer Bodenkontakt
\item hüpfende $\leftrightarrow$ glatte Bewegung
\item Elastische $\leftrightarrow$ steife Beine
\end{itemize}
\begin{figure}[h!]
	\centering
	\includegraphics[width=0.5\textwidth]{figures/ch06_uebergang1.png}
	\caption{Rennen vs. Gehen}
	\label{rg}
\end{figure}
\noindent
\autoref{uebergang} zeigt den Übergang vom Rennen zum Gehen. Auffällig ist die große Lücke zwischen Gehen und Rennen, es gibt keinen kontinuierlichen Übergang sondern der Geschwindigkeitswechsel verändert die Laufmuster grundlegend.
\begin{figure}[h!]
	\centering
	\includegraphics[width=0.4\textwidth]{figures/ch06_uebergang2.png}
	\caption{Übergang vom Rennen zum Gehen}
	\label{uebergang}
\end{figure}\\
Abbildungen \ref{eg} und \ref{erg} geben Informationen über elastisches Gehen.
\begin{figure}[h!]
	\centering
	\includegraphics[width=0.4\textwidth]{figures/ch06_gehen.png}
	\caption{Ablauf des elastischen Gehens}
	\label{eg}
\end{figure}
\begin{figure}[h!]
	\centering
	\includegraphics[width=0.4\textwidth]{figures/ch06_rennen-gehen.png}
	\caption{Elastizität beim Rennen und Gehen}
	\label{erg}
\end{figure}


\subsubsection{Technische Umsetzung dieser Prinzipien auf reale Systeme}
\autoref{umsetzung} stellt die Schritte zur Technischen Umsetzung dar.
\begin{figure}[h!]
	\centering
	\includegraphics[width=0.8\textwidth]{figures/ch06_umsetzung.png}
	\caption{Vorgehen bei der technischen Umsetzung}
	\label{umsetzung}
\end{figure}\\
\textbf{1. Analyse, Übertragung und Adaption der biomechanischen Laufmodelle}\\
Siehe \autoref{s1}.
\begin{figure}[h!]
	\centering
	\includegraphics[width=0.8\textwidth]{figures/ch06_tu1.png}
	\caption{Schritt 1}
	\label{s1}
\end{figure}\\
\textbf{3. Entwurf eines Algorithmus zur elastischen Beinlängen-Schwingungsregelung}\\
Siehe \autoref{s3}.
\begin{itemize}
\item Adaptives Einstellen der Eigenfrequenz über Systemgrößen
\item Kontrolliertes Anregen mit Resonanzfrequenz
\item[$\rightarrow$] Optimale energieeffiziente Bewegung
\item[$\rightarrow$] Unterschiedliche Lauffrequenzen durch Modifikation von Systemgrößen
\end{itemize}
\begin{figure}[h!]
	\centering
	\begin{subfigure}{.5\textwidth}
		\includegraphics[width=\textwidth]{figures/ch06_tu31.png}
		\caption{}
	\end{subfigure}
	\begin{subfigure}{.5\textwidth}
		\includegraphics[width=\textwidth]{figures/ch06_tu32.png}
		\caption{}
	\end{subfigure}
	\caption{Schritt 3}
	\label{s3}
\end{figure}
Ein-Dimensionaler Versuchsstand (EDV): 
\begin{itemize}
\item Entwurf eines Versuchsstandes zur Entwicklung der Beinlängen Regelung
\item Flexibler Aufbau - Tests mit unterschiedlichen Muskellängen, Gewichten, Untergründen
\item Aufzeichnung aller relevanten Größen des Muskels und der Bewegung des Gesamtsystems
\end{itemize}
\textbf{4. Modellierung und Analyse der dynamischen Eigenschaften des gesamten Beins\\
5. Übertragung der elastischen Beinlängen-Regelung auf Mehrsegmentbein;\\
Kopplung und Synchronisation des Hüftantriebs mit der Beinlängenschwingung}
\begin{itemize}
\item 4 DOF pro Bein (2 DOF pro Hüfte) für erste Tests
\item Auslegung für Geschwindigkeiten von bis zu $3m/s$
\item Integrierte Winkel-, Drehmomentsensorik
\item Beinlänge ca. $1m$
\item Gesamtgewicht Unterkörper ca. $20kg$
\item Winkelbereich Hüfte: Vorwärts 135°, Rückwärts 45°, Seitwärts je 30°
\end{itemize}
Regelungsansatz: Virtual Model Control (VMC) – nach Pratt ‘95
\begin{itemize}
\item Kontrolle durch \glqq Virtual Components\grqq{} (VC)
\item Mit Hilfe virtueller externer Kräfte
\item Meist virtuelle Feder-Dämpfer-Systeme
\item Transformation auf Gelenkmomente mittels Jakobi Matrix
\item Keine inverse Kinematik
\item Keine Matrixinversion
\item Intuitive Kontrollmethode
\item Komplex wenn viele VCs von Nöten
\item[$\rightarrow$] Kontrolle ähnelt Marionettenspiel
\item Virtuelle Komponenten für Zweibeiniges System:
\begin{itemize}
\item Ähnelt reaktivem Ansatz
\item Es werden 23 VCs benötigt
\item Abhängig von Ereignissen werden VCs aktiviert, deaktiviert oder neu parametrisiert
\item Drei Grundlegende Zustände der Muskelgetriebenen Gelenke: Aktiv, Passiv, Frei
\end{itemize}
\end{itemize}
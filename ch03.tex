\emph{Muskeln - Natürliche Antagonisten}

\subsection{Der biologische Muskel – das Wirkprinzip}
\subsubsection{Muskelaufbau und unterschiedliche Muskelformen}
\begin{itemize}
	\item Der Muskel ist nur ein Teil des Bewegungsapparates (Sehnen, Muskelfaszien, Sehnenscheiden, Gleitbeutel, Sesambeine und Sehnenknorpel).
	\item Muskeln können nur ziehen, deshalb Gegenspieler (antagonistisches Prinzip).
	\item Viele verschiedene Muskeln für spezielle Bereiche (der Mensch hat ca. 300 Muskeln).
\end{itemize}
\textbf{Muskelgewebe:}
\begin{itemize}
	\item Herzmuskulatur (quergestreifte Muskulatur): das Herz ist ein Sonderfall
	\item Skelettmuskulatur (quergestreifte Muskulatur): für aktive Körperbewegungen; Trizeps, Bizeps
	\item Glatte Muskulatur: vor allem Eingeweidemuskulatur, z.B. Darm; keine willkürliche Kontrolle
\end{itemize}
\textbf{Muskelformen (Skelettmuskulatur):} siehe \autoref{Formen}\\
\begin{figure}[h!]
	\centering
		\includegraphics[width=0.7\linewidth]{figures/ch03_muskeln.png}
		\caption{Formen der Skelettmuskulatur: a. Spindelförmig, b. Ringförmig, c. Ringförmig, glatt, d. Zweiköpfig, e. Doppelt gefiedert, f. Einfach gefiedert, g. Zweibauchig, h. Platter Muskel, i. 					Mehrbauchiger Muskel}
	\label{Formen}
\end{figure}\\
\textbf{Muskelaufbau (Skelettmuskulatur):} siehe Abbildungen \ref{Aufbau} und \ref{Fkt}
\begin{figure}[h!]
	\centering
		\includegraphics[width=0.5\linewidth]{figures/ch03_musclefibres.png}
		\caption{Muskelaufbau des Skelettmuskels}
	\label{Aufbau}
\end{figure}\\
\begin{itemize}
	\item Muskelfasern verlaufen fast parallel in Längsrichtung
	\item Fiederung für Blutversorgung und Raum bei Dickenzunahme
	\item Volumen bleibt während der Kontraktion konstant 
	\item Dickenzunahme durch Vergrößerung des Faserwinkels
\end{itemize}
\begin{figure}[h!]
	\centering
		\includegraphics[width=0.5\linewidth]{figures/ch03_muskelfkt.png}
		\caption{Muskelaufbau des Skelettmuskels}
	\label{Fkt}
\end{figure}
\subsubsection{Prinzipielle Funktionsweise}
\begin{figure}[h!]
	\centering
		\includegraphics[width=0.1\linewidth]{figures/ch03_muskelaufbau.png}
		\caption{Funktionsweise des Skelettmuskels}
	\label{Fkt1}
\end{figure}
\autoref{Fkt1} stellt die Funktionsweise dar, hierbei gilt:
\begin{itemize}
	\item Gedehnter Zustand (schwarz)
	\item Kontrahiert Zustand (gestrichelt)
	\item Hubhöhe $h$, Zugkraft einer Muskelfaser $K$
	\item Zerlegung von $K$ in Komponenten $H$ und $a$: Je größer Fiederungswinkel, desto geringer die erreichte Kraft
	\item Kompensation in der Biologie durch:
	\begin{itemize}
		\item Großer Fiederungswinkel
		\item[$\rightarrow$] Mehr Muskelfasern sind beteiligt
	\end{itemize}
	\item Hubhöhe immer größer als Faserverkürzung
	\item Faustregel: Je größer der Querschnitt, desto größer die Kraft
\end{itemize}
\textbf{Geschwindigkeit - Last Beziehung (Hillsche Beziehung)}
\begin{figure}[h!]
	\centering
		\includegraphics[width=0.2\linewidth]{figures/ch03_hill.png}
		\caption{Hillsche Beziehung}
	\label{hill}
\end{figure}
\begin{itemize}
	\item geringe Last $\rightarrow$ Verkürzungsgeschwindigkeit maximal
	\item mit zunehmender Last $\rightarrow$ nimmt Geschwindigkeit ab
	\item ab gewisser Last $\rightarrow$ Muskel nicht verkürzbar (isometrische Verhältnisse)
\end{itemize} 

\subsubsection{Muskel als Effektor des Nervensystems}
\begin{itemize}
	\item Muskeln werden von einem oder mehreren Nerven versorgt, die \textbf{motorische und sensible} Fasern enthalten.
	\item Muskeln besitzen \textbf{Rezeptoren}, die über den jeweiligen Muskelzustand
informieren:
	\begin{itemize}
		\item Dehnungsrezeptoren (Muskelspindeln)
		\item Spannungsrezeptoren (Sehnenorgane)	
	\end{itemize}
	\item Auch im Ruhezustand laufen ständig Nervenimpulse in den Muskel. Sie bewirken eine Grundspannung (\textbf{Tonus}) des Muskels.
	\item Verstärkte Nervenimpulse verursachen eine Verkürzung (isotonische Kontraktion) oder Spannungszunahme (isometrische Kontraktion).
	\item Bewegungen erfordern \textbf{synergistisches und antagonistisches} Zusammenwirken von Muskeln
	\item Synergisten: Muskeln mit gleicher Wirkrichtung
	\item Antagonisten: Muskeln mit gegensinniger Wirkrichtung
	\item Durch die Kombination beider Wirkprinzipien: Bremsende, gezügelte, harmonische und genau dosierte Bewegung möglich
	\item Einfacher Aufbau von einachsigen Gelenken, aber kompliziertes Zusammenwirken bei komplexeren Gelenken (z.B. Kugelgelenke)!
\end{itemize}
\subsection{Klassische Antriebe in der Robotik}
\begin{itemize}
	\item \textbf{Pneumatische Antriebe}: Einsatz häufig bei Explosionsgefährdung oder Reinraumerfordernissen 
	\item \textbf{Hydraulische Antriebe}: ab gewissem Gewicht besser, aber durch Linearachse Rotationsbewegungen schwierig
	\item \textbf{Elektrische Antriebe}: am weitesten verbreitet  da günstig, hochpräzise Getriebeübersetzung
\end{itemize}
\textbf{Leistungsgewichtsvergleich unterschiedlicher Antriebe in Natur und Technik}:
\begin{figure}[h!]
	\centering
	\begin{subfigure}{.5\textwidth}
		\includegraphics[width=\linewidth]{figures/ch03_klassische-antriebe.png}
		\caption{Technik}
	\end{subfigure}
	\begin{subfigure}{.5\textwidth}
		\includegraphics[width=\linewidth]{figures/ch03_klassische-antriebe1.png}
		\caption{Natur}
	\end{subfigure}
	\caption{}
\end{figure}
\newpage
\subsection{Künstliche Muskelsysteme}
\emph{Von drehender Mechanik zu selbstverformenden Polymeren}
\subsubsection{Mechanischer Muskel}
\begin{figure}[h!]
	\centering
	\includegraphics[width=0.6\linewidth]{figures/ch03_mech-muskel.png}
	\caption{Typen mechanischer Muskeln}
	\label{mechmus}
\end{figure}
\begin{itemize}
	\item Linear to Angle Displacement Device (LADD)
	\item Concentric LADD (CLADD)
\end{itemize}
\subsubsection{Chemische Muskeln (SMP, LAP)}
\begin{itemize}
	\item Light-Activated Polymers (LAP)
	\item Shape Memory Polymers (SMP)
\end{itemize}
\subsubsection{Elektroaktive Polymere (EAP)}
\begin{itemize}
	\item Electronic EAP: Ferroelectric, Dielectric, Electrostrictive
	\item Ionic EAP: Ionic Polymer Gels, Ionometric Polymer-Metal Composites, Conductive Polymers
	\item Anwendungen: Greifer, Lobster
	\item [$\rightarrow$] eher Grundlagenforschung, man kann keine großen Kräfte aufbringen, aber: kann sogar Schwimmbewegungen erzeugen; eher für sehr feine Bewegungen geeignet
\end{itemize}
\subsubsection{Piezoelektrische Antriebe}
\subsection{Fluidische Muskeln}
\emph{Fluidik ist die Wissenschaft, die sich mit flüssiger oder gasförmiger Materie befasst.}
\subsubsection{Spinnenprinzip}
z.B. \glqq Black Spider\grqq : pneumatische mehrbeinige Laufmaschine
\subsubsection{Weitere Muskeln}
z.B. Pleated Pneumatic Artificial Muscle (siehe \autoref{ppam}) 
\begin{figure}[h!]
	\centering
	\includegraphics[width=0.3\linewidth]{figures/ch03_ppam.png}
	\caption{}
	\label{ppam}
\end{figure}
\begin{itemize}
\item Steifigkeit künstlich vorgegeben (Feder-Masse-System, Federkonstante zur Laufzeit veränderbar durch Aufblasen)
\item Vorteile:
\begin{itemize}
\item Elastizität, fließende Bewegungen, Steifigkeit einstellbar
\item Energie im System speicherbar
\item schnelle Bewegung, passive Dämpfung gegenüber Stößen
\item[$\rightarrow$] spannend für Laufbewegungen
\end{itemize}
\item Nachteile:
\begin{itemize}
\item sehr komplex, pro Gelenk mindestens zwei  Antriebe, alle müssen mit Zu- und Ableitungen/Ventilen versorgt  werden
\item Kompressor mit Druckluft nötig, schwierig unterzubringen
\end{itemize}
\end{itemize}
\subsubsection{McKibben-Muskel}
z.B. Shadow-Muskel (schnell konfektionierbar), FESTO-Muskel (Industrienorm)
\subsection{McKibben Prinzip}
\subsubsection{Aufbau und Funktionsweise am Beispiel FESTO-MAS}
\begin{figure}[h!]
	\centering
	\begin{subfigure}{.6\textwidth}
		\includegraphics[width=\linewidth]{figures/ch03_mckibben1.png}
		\caption{Gewebe mit gewissen Winkeln zwischen einzelnen Fasern: Bei Druck möchte sich das System ausbreiten. Durch Faserwinkel entsteht Kontraktion in der Längsrichtung.}
	\end{subfigure}
	\begin{subfigure}{.3\textwidth}
		\includegraphics[width=\linewidth]{figures/ch03_mckibben.png}
		\caption{}
	\end{subfigure}
	\caption{McKibben-Prinzip}
	\label{mck}
\end{figure}
\noindent
Der Innendruck im Muskel bewirkt Längskontraktion, bzw. eine Zugkraft in axialer Richtung, da sich die Länge der einzelnen Gewebefasern nicht verändern kann. Dieses Prinzip funktioniert nur für einen Grundfaserwinkel von weniger als ca. 55°.\\
\autoref{fm} zeigt die Kennkurve für den FESTO Fluidic-Muscle unter verschiedenen Drücken. Auf der $x$-Achse ist die Längenänderung $h$ in \% abgetragen, die $y$-Achse zeigt die Kraft $F$. Es entstehen verschiedene Kurven durch unterschiedliche Kräfte je nach Längenänderung.
\begin{figure}[h!]
	\centering
	\includegraphics[width=0.5\linewidth]{figures/ch03_festo-muskel.png}
	\caption{Zusammenhang zwischen Kontraktion, Druck und Kraft des Muskels}
	\label{fm}
\end{figure}
\newpage
\subsubsection{Verschiedene Muskelmodelle – Simplizität vs. Präzision}
\textbf{Statisches Muskelmodell:} Das Modell basiert auf der Annahme, dass sich
der Muskel wie ein idealer Zylinder verhält (vgl. \autoref{statmod})!\\
\begin{figure}[h!]
	\centering
	\includegraphics[width=0.5\linewidth]{figures/ch03_stat-modell.png}
	\caption{}
	\label{statmod}
\end{figure}
Kraft hängt von Kontraktion und Druck ab:\\
\begin{align*}
	F(\kappa, P) = (\pi \cdot r_0^2) \cdot P \cdot \left[a \cdot (1 - \kappa)^2 - b\right]
\end{align*}
mit
\begin{align*}
	\kappa &= \frac{(l_0 - l}{l_0}, 0 < \kappa \leq \kappa_{max}\\
	a &= \frac{3}{tan^2(\alpha_0}, b = \frac{1}{sin^2(\alpha_0)}\\
\end{align*} mit $\alpha_0$ = Ausgangsfaserwinkel, $\kappa$ = Kontraktion, $P$ = Druck\\
\begin{align*}
	F_{max} &= (\pi \cdot r_0^2) \cdot P \cdot \left[a - b\right] \text{ mit} \kappa = 0\\
	\kappa_{max} &= 1 - \sqrt{\frac{b}{a}} \text{ mit} F = 0
\end{align*}
Der Faktor $\kappa$ ist die relative Längenänderung, für 0 Kontraktion wird Term maximal, für Kraft = 0 maximale Kontraktion in Abhängigkeit vom Faserwinkel.\\
Erweiterung des Modells durch Einführung eines Parameters $\varepsilon$:
\begin{itemize}
	\item Berücksichtigung der Einschnürung an den Enden des Muskels
	\item Modifizierung der Kurve für den Fall eines geringen Drucks
	\item Bestimmung von $c_1$ und $c_2$ durch Least-Square-Schätzer: Vergleich des Modells mit dem realen System
\end{itemize}
\begin{align*}
	F(\kappa, P) = (\pi \cdot r_0^2) \cdot P \cdot \left[a \cdot (1 - \varepsilon \cdot \kappa)^2 - b\right] \text{ mit} \varepsilon = c_1 \cdot e^{-P} + c_2
\end{align*}
\begin{figure}[h!]
	\centering
	\begin{subfigure}{.4\textwidth}
		\includegraphics[width=\linewidth]{figures/ch03_stat-mod0.png}
		\caption{Originalkurven vs. gefilterte Modellkurven}
	\end{subfigure}
	\begin{subfigure}{.4\textwidth}
		\includegraphics[width=\linewidth]{figures/ch03_stat-modell1.png}
		\caption{Besserer Schätzer nach Einführung des Korrekturfaktors: jetzt wird berücksichtigt, dass der Zylinder keine gleichmäßige Ausdehnung, Einschnürungen und Wölbungen aufweist}
	\end{subfigure}
	\caption{Statisches Muskel-Modell}
	\label{mck}
\end{figure}\\
\noindent
\textbf{Dynamisches Muskel-Modell:}
Kraftgleichung:
\begin{align*}
	F_{Mus} = F_{Feder}(P, \kappa) + F_{Daempfer}(P, \dot{\kappa})   
\end{align*}
\begin{figure}[h!]
	\centering
	\includegraphics[width=0.25\linewidth]{figures/ch03_dyn-modell.png}
	\caption{Mechanisches Ersatzschaltbild}
	\label{dm}
\end{figure}
\newpage
\noindent
\textbf{Quick-Release Test Umgebung}
\begin{figure}[h!]
	\centering
	\includegraphics[width=0.3\linewidth]{figures/ch03_testumgebung.png}
	\caption{Test-Umgebung}
	\label{tu}
\end{figure}
\begin{itemize}
	\item Erfassung der dynamischen Eigenschaften des Muskels
	\item Aufzeichnung aller relevanten Muskelkenngrößen
	\item Vergleich des Verhaltens mit biologischem Muskel: Validierung des Muskelmodells z.B. durch Vergleich zwischen aufgezeichneten Daten und Matlab-Simulation
	\item $P_{mech} = F \cdot v$
	\item Validierung des Muskelmodells durch realen mechanischen Aufbau: Bis auf die Anfangsbereiche sehr gute Annäherung an Simulation, daher nutzbar für die Regelung
\end{itemize}

\textbf{Kraft-Geschwindigkeits-Zusammenhang}
\begin{figure}[h!]
	\centering
	\includegraphics[width=0.8\linewidth]{figures/ch03_pv-zshang.png}
	\caption{Kraft-Geschwindigkeits-Zusammenhang: vgl. Hiller'sches Prinzip, genau die gleiche Kurve in technischem System implementiert, biologisches Verhalten schön nachgebildet}
	\label{tu}
\end{figure}

%\subsubsection{Beispiele und Anwendungen}
\subsection{Antagonistische Regelung}
\subsubsection{Antagonistisches Gelenk}
\textbf{Gelenkregelung Stabheuschrecke}:
\begin{figure}[h!]
	\centering
	\begin{subfigure}{.5\textwidth}
		\includegraphics[width=\linewidth]{figures/ch03_Gelenkregelung-Stabheuschrecke.png}
		\caption{}
	\end{subfigure}
	\begin{subfigure}{.5\textwidth}
		\includegraphics[width=\linewidth]{figures/ch03_Gelenkregelung-Stabheuschrecke1.png}
		\caption{Interneurone zur Verstärkung zur Regelung der Tibia-Stellung}
	\end{subfigure}
	\caption{Statisches Muskel-Modell}
	\label{grhs}
\end{figure}\\
\newpage
\noindent
\textbf{Kräfteverhältnis in einem Gelenk} (vgl. \autoref{ar}):\\
$(F_A - F_B)\cdot r = M_S$ mit
\begin{itemize}
\item $F_{A(B)}$: Muskelkraft in Muskel $A(B)$
\item $r$: Radius der Gelenktrommel
\item $M_S$ Störmoment, bzw. äußeres Moment
\end{itemize}
\begin{figure}[h!]
	\centering
	\includegraphics[width=0.4\linewidth]{figures/ch03_antagonistische-regelung.png}
	\caption{Technische Umsetzung als klassisches Rotationsgelenk über Seil/Draht, zwei Muskeln zusammen schalten, in Abhängigkeit von Hebelarm (Durchmesser der Scheibe)
und $F_A, F_B$,  Regelung der Steifigkeit durch Zug auf Seil}
	\label{ar}
\end{figure}
\textbf{Modellbildung -- Aufbau eines Gelenks} (vgl. \autoref{ar1}): gut mathematisch modellierbar\\
Gelenk-Bewegungsgleichung: $J\ddot{\varphi} = -rF_{A}(\kappa_A, \dot{\kappa}_A, p_A)+rF_B(\kappa_B, \dot{\kappa}_B, p_B)-\frac{1}{2}cos(\varphi - \varphi_0)F_g+lF_s$
\begin{figure}[h!]
	\centering
	\includegraphics[width=0.4\linewidth]{figures/ch03_antagonistische-regelung1.png}
	\caption{}
	\label{ar1}
\end{figure}
\newpage
\textbf{Modellbildung -- Festlegung der Zustandsgrößen} (vgl. \autoref{zm}): 
\begin{figure}[h!]
	\centering
	\includegraphics[width=0.4\linewidth]{figures/ch03_zwei-muskeln.png}
	\caption{}
	\label{zm}
\end{figure}
\begin{itemize}
\item Zustandsgrößen $\underline{x}$
\begin{itemize}
\item Druck Muskel $A$
\item Druck Muskel $B$
\item Gelenkwinkel $\varphi$
\item Winkelgeschwindigkeit $\dot{\varphi}$
\item Winkelbeschleunigung $\ddot{\varphi}$
\end{itemize}
\item Eingangsgrößen $\underline{u}$
\begin{itemize}
\item Öffnungsfläche $A_V$ von Ventil $A$
\item Öffnungsfläche $A_V$ von Ventil $B$
\end{itemize}
\item nichtlineare DGL (kann zur Regelung benutzt werden):
\begin{itemize}
\item[] $\underline{\dot{x}} = \underline{f}(\underline{x}, \underline{u}, t)$
\item[] $\underline{y} = \underline{g}(\underline{x}, \underline{u}, t)$
\end{itemize}
\end{itemize}
\subsubsection{Zusammenspiel zweier Fluidischer Muskeln}
\begin{figure}[h!]
	\centering
	\begin{subfigure}{.5\textwidth}
		\includegraphics[width=\linewidth]{figures/ch03_zusammenspiel.png}
		\caption{In neutraler Gelenkposition stark kontrahiert; je nach gewünschter Längenänderung Veränderung der maximalen möglichen Kraft; Schnittstelle: Neutrale Gelenkposition, bereits starke Kontraktion (etwa 12-13\%)}
	\end{subfigure}
	\begin{subfigure}{.5\textwidth}
		\includegraphics[width=\linewidth]{figures/ch03_zusammenspiel1.png}
		\caption{In neutraler Gelenkposition wenig kontrahiert; dichtere Anordnung: höhere Kräfte möglich aber kleinerer Gelenkwinkelbereich, daher viele Möglichkeiten, System einzustellen}
	\end{subfigure}
	\caption{}
	\label{zsspiel}
\end{figure}
\textbf{Pneumatischer Muskel -- Robotergelenke} (vgl. \autoref{zm1}):
\begin{figure}[h!]
	\centering
	\includegraphics[width=0.4\linewidth]{figures/ch03_zwei-muskeln1.png}
	\caption{}
	\label{zm1}
\end{figure}
\begin{itemize}
	\item Einstellbare Steifigkeit
	\item Flexibles Gelenk durch passive Nachgiebigkeit
	\item Schonung und Reduktion der Mechanik
	\item Inhärente Gelenkanschläge
	\item Gelenkeigenschaften über Motorgrößen einstellbar (mit Elektromotoren undenkbar)
\end{itemize}
\subsubsection{Regelungsbeispiel für ein flexibles Gelenk}
\begin{figure}[h!]
	\centering
	\begin{subfigure}{.5\textwidth}
		\includegraphics[width=\linewidth]{figures/ch03_gelenksteuerung.png}
		\caption{Steuerung eines Gelenkes: Nur möglich wenn keine Störungen $\rightarrow$ Regelung}
	\end{subfigure}
	\begin{subfigure}{.5\textwidth}
		\includegraphics[width=\linewidth]{figures/ch03_gelenkregelung.png}
		\caption{Einfache PID-P-Regler: soll Regelfaktor des Sollgelenkwinkels anpasssen; funktioniert nicht wirklich}
	\end{subfigure}
	\begin{subfigure}{.5\textwidth}
		\includegraphics[width=\linewidth]{figures/ch03_gelenkregelung1.png}
		\caption{Regelung mit Muskelmodell}
		\label{rmm}
	\end{subfigure}
	\caption{}
	\label{gbsp}
\end{figure}
Zu \autoref{rmm}:
\begin{itemize}
	\item Regelung eines künstlichen Muskels nach McKibben-Prinzip: hochkomplziert, trotz vieler Vorteile hochgradig nicht-linear auch wenn Grundtechniken nicht so schwer
	\item Vorgabe der Sollposition und der Gelenksteifigkeit
	\item Berücksichtigung des am Gelenk angreifenden Störmomentes
	\item Berücksichtigung der maximal möglichen Kraft in allen möglichen Gelenkstellungen
\end{itemize}
\subsection{Anwendungsbeispiele}
\emph{There is no business like show business}\\ \\
\textbf{Motivation – Muskel als Antrieb von Laufmaschinen:}
\begin{itemize}
	\item schnelle und präzise langsame Bewegungsabläufe möglich
	\item zusammen mit Sehnen sowohl Aktuator als auch Sensor
	\item Nutzung der passiven Dämpfung als Energiespeicher
	\item Anpassung an unebenes Gelände gut möglich (passive Nachgiebigkeit)
	\item nur Zugkräfte, daher immer als antagonistisches Paar eingesetzt
	\item[$\rightarrow$] Eigenschaften, die auch für Laufmaschinenantriebe wünschenswert sind
\end{itemize}
\textbf{AirBug:} sechsbeiniger, durch fluidische Muskeln angetriebener Roboter.\\
Mechanische Parameter:
\begin{itemize}
	\setlength\itemsep{0em}
	\item Gewicht 32 kg (14kg Zentralkörper, 6kg Ventile, 2 kg pro Bein)
	\item Zentralkörper (mm): $850 \times 600 \times 250$
	\item Aufstandsfläche (mm): 1800 (width) $\times$ 1500 (length)
	\item Bein-Länge (mm): $100 \times 472 \times 560$
	\item Muskel-Längen (mm): $\alpha: 160, \beta: 240, \gamma 260$
	\item Winkelbereich: $\alpha: 55°; \beta: 90°; \gamma: 75°$ (sehr ähnlich wie bei der Stabheuschrecke)
	\item Max. Drehmoment: $\alpha: 26.4Nm, \beta: 48.6Nm, \gamma: 31.3Nm$
	\item Winkelgeschwindigkeit: $60°/s$
	\item Max. Geschwindigkeit: $10cm/s$
\end{itemize}
\textbf{PANTER:} \textbf{P}neumatic \textbf{A}ctuated dy\textbf{N}amical s\textbf{T}able quadrup\textbf{E}d \textbf{R}obot\\
Adaptives Elastisches Laufen mit künstlichen Muskeln durch den Entwurf eines Algorithmus zur elastischen Beinlängen-Schwingungsregelung:
\begin{itemize}
	\item Adaptives Einstellen der Eigenfrequenz über Systemgrößen
	\item Kontrolliertes Anregen mit Resonanzfrequenz
	\item[$\rightarrow$] Optimale energieeffiziente Bewegung
	\item[$\rightarrow$] Unterschiedliche Lauffrequenzen durch Modifikation von
Systemgrößen
\end{itemize}

\textbf{FESTO-Muskel}

\subsubsection{Vergleich biologischer vs. fluidischer Muskel}
\begin{table}[hbt]
\centering
\begin{tabular}{|c||c|c|}
\hline
 & \textbf{biologisch} & \textbf{fluidisch} \\
\hline
Kontraktion &  50 – 70\%  & max. 35\%\\
Kraft/cm2  & 20 – 60N  & bis 700N\\
Leistung/kg  & 40 – 250W/kg  & 0,5 – 2kW/kg\\
Wirkungsgrad  & 20 – 30\%  & 22 – 50\%\\
Kontraktionsgeschwindigkeit  & bis 8m/s  & bis 2m/s\\
Bandweite  & 20-100Hz (Zuckungen) & 5Hz\\
Regelung & gut bis sehr gut & befriedigend bis gut\\
Betrieb in Wasser & ja & ja\\
Temperaturbereich & 0 – 40°C & -30 bis +80°C\\
Widerstandsfähigkeit & sehr gut & befriedigend bis gut\\
Regeneration & ja & nein\\
Energiequelle & chemisch & pneumatisch\\
Umweltschmutz & produziert $CO_2$ & je nach Herstellung der Druckluft\\
Skalierbarkeit & $\mu m - m$ & mm – m\\
Linearbetrieb & ja & ja\\
\hline
\end{tabular}
\caption{Biologischer vs. fluidischer Muskel}
\label{tab:Vergleich}
\end{table}
\newpage
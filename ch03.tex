\emph{Muskeln - Natürliche Antagonisten}

\subsection{Der biologische Muskel – das Wirkprinzip}
\subsubsection{Muskelaufbau und unterschiedliche Muskelformen}
\textbf{Allgemeines:}
\begin{itemize}
	\item Der Muskel ist nur ein Teil des Bewegungsapparates (Sehnen, Muskelfaszien, Sehnenscheiden, Gleitbeutel, Sesambeine und Sehnenknorpel).
	\item Muskeln können nur ziehen, deshalb Gegenspieler (antagonistisches Prinzip).
	\item Viele verschiedene Muskeln für spezielle Bereiche (der Mensch hat ca. 300 Muskeln).
\end{itemize}
\textbf{Muskelgewebe:}
\begin{itemize}
	\item Herzmuskulatur (quergestreifte Muskulatur): das Herz ist ein Sonderfall
	\item Skelettmuskulatur (quergestreifte Muskulatur): für aktive Körperbewegungen; Trizeps, Bizeps
	\item Glatte Muskulatur: vor allem Eingeweidemuskulatur, z.B. Darm; keine willkürliche Kontrolle
\end{itemize}
\textbf{Muskelformen (Skelettmuskulatur):}
\begin{enumerate}
	\item Spindelförmig
	\item Ringförmig
	\item Ringförmig, glatt
	\item Zweiköpfig
	\item Doppelt gefiedert
	\item Einfach gefiedert
	\item Zweibauchig
	\item Platter Muskel
	\item Mehrbauchiger Muskel
\end{enumerate}
\textbf{Muskelaufbau (Skelettmuskulatur):}
\begin{itemize}
	\item Muskelfasern verlaufen fast parallel in Längsrichtung
	\item Fiederung für Blutversorgung und Raum bei Dickenzunahme
	\item Volumen bleibt während der Kontraktion konstant 
	\item Dickenzunahme durch Vergrößerung des Faserwinkels
\end{itemize}
\subsubsection{Prinzipielle Funktionsweise}
vgl. Abbildung
\begin{itemize}
	\item Gedehnter Zustand (schwarz)
	\item Kontrahiert Zustand (gestrichelt)
	\item Hubhöhe $h$, Zugkraft einer Muskelfaser $K$
	\item Zerlegung von $K$ in Komponenten $H$ und $a$: Je größer Fiederungswinkel, desto geringer die erreichte Kraft
	\item Kompensation in der Biologie durch:
	\begin{itemize}
		\item Großer Fiederungswinkel
		\item[$\rightarrow$] Mehr Muskelfasern sind beteiligt
	\end{itemize}
	\item Hubhöhe immer größer als Faserverkürzung
	\item Faustregel: Je größer der Querschnitt, desto größer die Kraft
\end{itemize}
\textbf{Geschwindigkeit - Last Beziehung (Hillsche Beziehung)}
\begin{itemize}
	\item geringe Last $\rightarrow$ Verkürzungsgeschwindigkeit maximal
	\item mit zunehmender Last $\rightarrow$ nimmt Geschwindigkeit ab
	\item Ab gewisser Last $\rightarrow$ Muskel nicht verkürzbar (isometrische Verhältnisse)
\end{itemize} 

\subsubsection{Muskel als Effektor des Nervensystems}
\begin{itemize}
	\item Muskeln werden von einem oder mehreren Nerven versorgt, die \textbf{motorische und sensible} Fasern enthalten.
	\item Muskeln besitzen \textbf{Rezeptoren}, die über den jeweiligen Muskelzustand
informieren:
	\begin{itemize}
		\item Dehnungsrezeptoren (Muskelspindeln)
		\item Spannungsrezeptoren (Sehnenorgane)	
	\end{itemize}
	\item Auch im Ruhezustand laufen ständig Nervenimpulse in den Muskel. Sie bewirken eine Grundspannung (\textbf{Tonus}) des Muskels.
	\item Verstärkte Nervenimpulse verursachen eine Verkürzung (isotonische Kontraktion) oder Spannungszunahme (isometrische Kontraktion).
	\item Bewegungen erfordern \textbf{synergistisches und antagonistisches} Zusammenwirken von Muskeln
	\item Synergisten: Muskeln mit gleicher Wirkrichtung
	\item Antagonisten: Muskeln mit gegensinniger Wirkrichtung
	\item Durch die Kombination beider Wirkprinzipien: Bremsende, gezügelte, harmonische und genau dosierte Bewegung möglich
	\item Einfacher Aufbau von einachsigen Gelenken, aber kompliziertes Zusammenwirken bei komplexeren Gelenken!
\end{itemize}
\subsection{Klassische Antriebe in der Robotik}
\subsubsection{Pneumatische Antriebe}
\subsubsection{Hydraulische Antriebe}
\subsubsection{Elektrische Antriebe}
\subsubsection{Leistungsgewichtsvergleich unterschiedlicher Antriebe}
\subsection{Künstliche Muskelsysteme}
\emph{Von drehender Mechanik zu selbstverformenden Polymeren}
\subsubsection{Mechanischer Muskel}
\begin{itemize}
	\item Linear to Angle Displacement Device (LADD)
	\item Concentric LADD (CLADD)
\end{itemize}
\subsubsection{Chemische Muskeln (SMP, LAP)}
\begin{itemize}
	\item Light-Activated Polymers (LAP)
	\item Shape Memory Polymers (SMP)
\end{itemize}
\subsubsection{Elektroaktive Polymere (EAP)}
\begin{itemize}
	\item Electronic EAP: Ferroelectric, Dielectric, Electrostrictive
	\item Ionic EAP: Ionic Polymer Gels, Ionometric Polymer-Metal Composites, Conductive Polymers
	\item Anwendungen: Greifer, Lobster
\end{itemize}
\subsubsection{Piezoelektrische Antriebe}
\subsection{Fluidische Muskeln}
\emph{Fluidik ist die Wissenschaft, die sich mit
flüssiger oder gasförmiger Materie befasst.}
\subsubsection{Spinnenprinzip}
z.B. \glqq Black Spider\grqq : pneumatische mehrbeinige Laufmaschine
\subsubsection{Weitere Muskeln}
z.B. Pleated Pneumatic Artificial Muscle
\subsubsection{McKibben-Muskel}
z.B. Shadow-Muskel, FESTO-Muskel
\subsection{McKibben Prinzip}
\subsubsection{Aufbau und Funktionsweise am Beispiel FESTO-MAS}
Innendruck im Muskel bewirkt Längskontraktion, bzw. eine Zugkraft in axialer Richtung, da sich die Länge der einzelnen Gewebefasern nicht verändern kann. Dieses Prinzip funktioniert nur für einen Grundfaserwinkel von weniger als ca. 55°.
Abb: Zusammenhang zwischen Kontraktion, Druck und Kraft des Muskels
\subsubsection{Verschiedene Muskelmodelle – Simplizität vs. Präzision}
\textbf{Statisches Muskelmodell:} Das Modell basiert auf der Annahme, dass sich
der Muskel wie ein idealer Zylinder verhält!\\
Kraft hängt von Kontraktion und Druck ab:\\
\begin{align*}
	F(\kappa, P) = (\pi \cdot r_0^2) \cdot P \cdot \left[a \cdot (1 - \kappa)^2 - b\right]
\end{align*}
mit
\begin{align*}
	\kappa &= \frac{(l_0 - l}{l_0}, 0 < \kappa \leq \kappa_{max}\\
	a &= \frac{3}{tan^2(\alpha_0}, b = \frac{1}{sin^2(\alpha_0)}\\
\end{align*} mit $\alpha_0$ = Ausgangsfaserwinkel, $\kappa$ = Kontraktion, $P$ = Druck\\
\begin{align*}
	F_{max} &= (\pi \cdot r_0^2) \cdot P \cdot \left[a - b\right] \text{ mit} \kappa = 0\\
	\kappa_{max} &= 1 - \sqrt{\frac{b}{a}} \text{ mit} F = 0
\end{align*}
Erweiterung des Modells durch Einführung eines Parameters $\varepsilon$:
\begin{itemize}
	\item Berücksichtigung der Einschnürung an den Enden des Muskels
	\item Modifizierung der Kurve für den Fall eines geringen Drucks
	\item Bestimmung von $c_1$ und $c_2$ durch Least-Square-Schätzer: Vergleich des Modells mit dem realen System
\end{itemize}
\begin{align*}
	F(\kappa, P) = (\pi \cdot r_0^2) \cdot P \cdot \left[a \cdot (1 - \varepsilon \cdot \kappa)^2 - b\right] \text{ mit} \varepsilon = c_1 \cdot e^{-P} + c_2
\end{align*}
\noindent
\textbf{Dynamisches Muskelmodell:}
Kraftgleichung:
\begin{align*}
	F_{Mus} = F_{Feder}(P, \kappa) + F_{Daempfer}(P, \dot{\kappa})   
\end{align*}
Mechanisches Ersatzschaltbild siehe Abbildung\\
\noindent
\textbf{Quick-Release Test Umgebung}
\begin{itemize}
	\item Erfassung der dynamischen Eigenschaften des Muskels
	\item Aufzeichnung aller relevanten Muskelkenngrößen
	\item Vergleich des Verhaltens mit biologischem Muskel: Validierung des Muskelmodells z.B. durch Vergleich zwischen aufgezeichneten Daten und Matlab-Simulation
\end{itemize}

\textbf{Kraft-Geschwindigkeits-Zusammenhang}
vgl. Abbildung
\subsubsection{Beispiele und Anwendungen}
\subsection{Antagonistische Regelung}
\subsubsection{Antagonistisches Gelenk}
\subsubsection{Zusammenspiel zweier Fluidischer Muskeln}
\subsubsection{Regelungsbeispiel für ein flexibles Gelenk}
\subsection{Anwendungsbeispiele}
\emph{There is no business like show business}\\ \\
\textbf{Motivation – Muskel als Antrieb von Laufmaschinen:}
\begin{itemize}
	\item schnelle und präzise langsame Bewegungsabläufe möglich
	\item zusammen mit Sehnen sowohl Aktuator als auch Sensor
	\item Nutzung der passiven Dämpfung als Energiespeicher
	\item Anpassung an unebenes Gelände gut möglich (passive Nachgiebigkeit)
	\item nur Zugkräfte, daher immer als antagonistisches Paar eingesetzt
	\item[$\rightarrow$] Eigenschaften, die auch für Laufmaschinenantriebe wünschenswert sind
\end{itemize}
\textbf{AirBug:} sechsbeiniger, durch fluidische Muskeln angetriebener Roboter.\\
Mechanische Parameter:
\begin{itemize}
	\setlength\itemsep{0em}
	\item Gewicht 32 kg (14kg Zentralkörper, 6kg Ventile, 2 kg pro Bein)
	\item Zentralkörper (mm): $850 \times 600 \times 250$
	\item Aufstandsfläche (mm): 1800 (width) $\times$ 1500 (length)
	\item Bein-Länge (mm): $100 \times 472 \times 560$
	\item Muskel-Längen (mm): $\alpha: 160, \beta: 240, \gamma 260$
	\item Winkelbereich: $\alpha: 55°; \beta: 90°; \gamma: 75°$
	\item Max. Drehmoment: $\alpha: 26.4Nm, \beta: 48.6Nm, \gamma: 31.3Nm$
	\item Winkelgeschwindigkeit: $60°/s$
	\item Max. Geschwindigkeit: $10cm/s$
\end{itemize}
\textbf{PANTER:} \textbf{P}neumatic \textbf{A}ctuated dy\textbf{N}amical s\textbf{T}able quadrup\textbf{E}d \textbf{R}obot\\
Adaptives Elastisches Laufen mit künstlichen Muskeln durch den Entwurf eines Algorithmus zur elastischen Beinlängen-Schwingungsregelung:
\begin{itemize}
	\item Adaptives Einstellen der Eigenfrequenz über Systemgrößen
	\item Kontrolliertes Anregen mit Resonanzfrequenz
	\item[$\rightarrow$] Optimale energieeffiziente Bewegung
	\item[$\rightarrow$] Unterschiedliche Lauffrequenzen durch Modifikation von
Systemgrößen
\end{itemize}

\textbf{FESTO-Muskel}

\subsubsection{Vergleich biologischer vs. fluidischer Muskel}
\begin{table}[hbt]
\centering
\begin{tabular}{|c||c|c|}
\hline
 & \textbf{biologisch} & \textbf{fluidisch} \\
\hline
Kontraktion &  50 – 70\%  & max. 35\%\\
Kraft/cm2  & 20 – 60N  & bis 700N\\
Leistung/kg  & 40 – 250W/kg  & 0,5 – 2kW/kg\\
Wirkungsgrad  & 20 – 30\%  & 22 – 50\%\\
Kontraktionsgeschwindigkeit  & bis 8m/s  & bis 2m/s\\
Bandweite  & 20-100Hz (Zuckungen) & 5Hz\\
Regelung & gut bis sehr gut & befriedigend bis gut\\
Betrieb in Wasser & ja & ja\\
Temperaturbereich & 0 – 40°C & -30 bis +80°C\\
Widerstandsfähigkeit & sehr gut & befriedigend bis gut\\
Regeneration & ja & nein\\
Energiequelle & chemisch & pneumatisch\\
Umweltschmutz & produziert $CO_2$ & je nach Herstellung der Druckluft\\
Skalierbarkeit & $\mu m - m$ & mm – m\\
Linearbetrieb & ja & ja\\
\hline
\end{tabular}
\caption{Biologischer vs. fluidischer Muskel}
\label{tab:Vergleich}
\end{table}

\emph{Vieles, was Ingenieure bei technischen Sensoren noch für utopisch halten,
hat die Natur mit größter Raffinesse schon realisiert}

\subsection{Sinneseindrücke und Sensoren – Übersicht und Klassifikation}
\subsubsection{Klassifikation von Sinneseindrücken}
\begin{table}[hbt]
\centering
\begin{tabular}{|p{4cm}|p{5cm}|p{5cm}|}
\hline
\textbf{Sinne \& Rezeptortyp} & \textbf{Adäquater Reiz} & \textbf{Sinnesorgane beim
Menschen/ $\Rightarrow$ (Bsp. aus dem Tierreich)} \\
\hline
\hline
\textbf{Mechanische Sinne} \newline Mechanorezeptoren &  &\\
\hline
Tastsinn & Zug, Druck & Tastkörperchen\\
\hline
Strömungssinn & Strömungen in Wasser \& Gasen & $\Rightarrow$ (z.B. Seitenlinienorgan bei Amphibien und Fischen)\\
\hline
Drehsinn & Winkelbeschleunigung & Bogengänge im Innenohr\\
\hline
Lagesinn & Erdbeschleunigung = Schwerkraft & 3 Bogengänge im Innenohr $\Rightarrow$ (z.B. Borstenfeld bei Ameise)\\
\hline
Hörsinn & Schallwellen(= Druckwellen) 20 – 20.000 Hz beim Menschen Ultraschall (über 20 kHz) & Schnecke im Innenohr $\Rightarrow$ (z.B. Hörorgan bei Fledermaus)\\
\hline
\textbf{Chemische Sinne} \newline Chemorezeptoren & & \\
\hline
Geruchssinn & Chemische Stoffe in Gasphase & Riechschleimhaut der Nase\\
\hline
Geschmackssinn & Gelöster Stoff & Geschmacksknospen der Zunge\\
\hline
\textbf{Elektrischer Sinn} \newline Elektrorezeptoren & Elektrisches Feld & $\Rightarrow$(bei elektrischen Fischen)\\
\hline
\textbf{Magnetischer Sinn} \newline Magnetorezeptoren & Magnetfeld & $\Rightarrow$(bei Zugvögeln, Bienen)\\
\hline
\textbf{Lichtsinn} \newline Photorezeptoren & Elektromagnetische Wellen \glqq sichtbares Licht\grqq{} beim Menschen 380-780 nm, UV (unter 380 nm) Polarisiertes Licht & Augen $\Rightarrow$(z.B. Facettenauge der Insekten)\\
\hline
\end{tabular}
\caption{Übersicht über Sinne, Reize und Sinnesorgane}
\label{tab:Sinnesübersicht}
\end{table}

\subsubsection{Rezeptoren und Reizverarbeitung}
\subsubsection{Sensorik in der Technik}
Der Begriff des Sensors (lateinisch \glqq sensus\grqq{} der Sinn) leitet sich ab von den menschlichen Sinnen.
\begin{itemize}
\setlength\itemsep{0em}
\item Auge $\Rightarrow$ Optik, Kameras
\item Nase $\Rightarrow$ Gassensor
\item Tastsinn $\Rightarrow$ Druck, Temperatur
\item Geschmack $\Rightarrow$ chemische Sensoren
\end{itemize}
IEC-Definition für technische Sensoren:
\glqq Ein Sensor ist das primäre Element in einer Messkette, das eine variable Eingangsgröße in ein geeignetes Messsignal umsetzt.\grqq \\
\textbf{Anforderungen an technische Sensorik:}
\begin{itemize}
\setlength\itemsep{0em}
\item Genauigkeit 
\item Präzision 
\item Betriebsbereich 
\item Antwortgeschwindigkeit 
\item Kalibrierung 
\item Zuverlässigkeit 
\item Kosten
\item einfache Installation
\item Linearität
\item Auflösung
\item Reproduzierbarkeit
\item Querempfindlichkeit
\item Alterung
\item Ansprechverhalten
\item[$\Rightarrow$] Ausgehend von der Aufgabenstellung und des Ortes der Integration wird der Sensor ausgewählt
\end{itemize}

\textbf{Bionik in der Sensorik:}
\begin{itemize}
\setlength\itemsep{0em}
\item Biologische Sensoren sind...
\begin{itemize}
\setlength\itemsep{0em}
\item[...] hoch spezialisiert auf ihren Anwendungsbereich,
\item[...] hoch miniaturisiert,
\item[...] in großen \glqq Stückzahlen\grqq{} in Sensorfeldern angeordnet.
\end{itemize}
\item Schwerpunkte für eine technische Umsetzung:
\begin{itemize}
\setlength\itemsep{0em}
\item Sensorprinzip
\item Datenauswertung
\end{itemize}
\item Kopieren der Natur an dieser Stelle (noch) unmöglich!
\end{itemize}
\subsection{Sensorsysteme in Biologie und Technik – Eine vergleichende Einführung}
\subsubsection{Hören – Von der Fledermaus zum Ultraschallsensor}
\textbf{Ultraschall – Die Fledermaus als Vorbild:}
\begin{itemize}
\setlength\itemsep{0em}
\item Hoch entwickeltes Echoorientierungssystem
\item Kehlkopf erzeugt Ultraschalltöne im Bereich von 20 - 200 kHz
\item Echos werden von den Ohren empfangen und ausgewertet
\item Fähigkeit sich bei absoluter Dunkelheit zu orientieren:
\begin{itemize}
\setlength\itemsep{0em}
\item Distanz, Richtung, Form
\item Größe, Struktur und
\item Eigenbewegung der reflektierenden Ziele
\end{itemize}
\item Zum Teil auch Klassifikation der Beute
\end{itemize}
\textbf{Sonar:}
\begin{itemize}
\setlength\itemsep{0em}
\item Echoorientierung durch Serien kurzer Klicklauten (bis zu 130 kHz)
\item können verschiedene Fische, Pflanzen und Gegenstände unterscheiden (auch Minen unter dem Sand)
\item Echoorientierung nicht angeboren
\end{itemize}

\textbf{Ultraschall – Technische Umsetzung}
\begin{itemize}
\setlength\itemsep{0em}
\item Einsatz: 
\begin{itemize}
\setlength\itemsep{0em}
\item Abstandsmessung
\item Hinderniserkennung (Nahbereich)
\item Kartographieren der Umgebung (Fernbereich)
\end{itemize}
\item Eigenschaften:
\begin{itemize}
\setlength\itemsep{0em}
\item akustische Wellen im Frequenzbereich oberhalb 20 kHz
\item Ausbreitung ist nur in Materie möglich
\item beim Auftreffen einer Schallwelle auf eine Grenzfläche zwischen zwei Medien teilt sich die Welle in eine reflektierten und transmittierten Anteil auf 
\item Ausbreitungsgeschwindigkeit von Schallwellen in Luft (20°C): 343,8 m/s
\item Distanzbestimmung über Laufzeitmessung eines Schallsignals
\end{itemize}
\item Nachteile:
\begin{itemize}
\setlength\itemsep{0em}
\item Schallgeschwindigkeit ist abhängig von Temperatur, Luftfeuchtigkeit und Druck
\item Reflexion und Streuung an verschiedenen Oberflächen kann Messergebnisse verfälschen
\item es existiert eine kürzeste und längste messbare Entfernung
\end{itemize}
\end{itemize}
\subsubsection{Gelenkwinkel – Wie ist die Extremität orientiert?}
\textbf{Gelenkwinkelsensoren}
\begin{enumerate}
\setlength\itemsep{0em}
\item Messprinzipien (Ohmisch, Induktiv, Magnetisch, Photoelektrisch )
\item Interpolation
\item Absolute Positionsmessung (Anschlag, Indexkanal, Potentiometer, Graycodierung, Quasiabsolutcode)
\end{enumerate}

\textbf{Interpolation von Zwischenwerten:}
\begin{itemize}
\setlength\itemsep{0em}
\item Erhöhung der Auflösung
\item Aus analogem Signal (egal bei welchem Sensor) können Zwischenstufen interpoliert werden
\end{itemize}

\textbf{Absolute Positionsmessung}
\begin{itemize}
\setlength\itemsep{0em}
\item relative optische Encoder:
\begin{itemize}
\setlength\itemsep{0em}
\item nur relative Ticks
\item Absoluten Winkel nur durch Anschlag, Indexkanal
\item Pro: einfacher \& robuster Sensor, kostengünstig
\item Contra: Komplexere Reset-Routine, nur einmalige Referenzierung
\end{itemize}
\item absolute optische Encoder: 
\begin{itemize}
\setlength\itemsep{0em}
\item Absolute Position in jedem Schritt
\item Spezielle Codierung: z.B. Graycode
\item Pro: Kein extra Reset notwendig, kein Encoder-Drift möglich
\item Contra: Viele optische Spuren notwendig, komplexere Sensorauswertung, teurer \& empfindlicher
\end{itemize}
\item Gelenkwinkelerfassung mit Quasiabsolutwertgeber:
\begin{itemize}
\setlength\itemsep{0em}
\item Problematik: Absolute Gelenkposition notwendig; Bewegung zu einem Referenzpunkt nicht möglich; Gray codierte Sensoren schwer integrierbar und teuer
\item Vorteil: Nur 2 optische Spuren; Inkrementalcode + einfacher Absolutecode; Leicht integrierbar (Größe skalierbar, Runder / Längscode, Reflex / Gabellichtschranke)
\item Nachteil: Bewegen des Gelenks um Absolutcode auszulesen; Software zur Positionsberechung
\item Positionscode durch zwei Spuren abschnittsweise eindeutig
\end{itemize}
\end{itemize}
\subsubsection{Gravitation bzw. Beschleunigung – Innenohr gegen Inertialsystem}
\textbf{Inertialsysteme (Kreiselsysteme)}
\begin{itemize}
\setlength\itemsep{0em}
\item Realisierbar mit:
\begin{itemize}
\setlength\itemsep{0em}
\item Mechanischen Kreiseln
\item 3 Beschleunigungssensoren, 3 Piezo- oder Laser-Kreisel
\end{itemize}
\item Messwerte:
\begin{itemize}
\setlength\itemsep{0em}
\item Beschleunigung in 3 Raumachsen
\item Winkelgeschwindigkeit um 3 Raumachsen
\end{itemize}
\item Messergebnis:
\begin{itemize}
\setlength\itemsep{0em}
\item Bestimmung der Gravitationsrichtung
\item Bestimmung der relativen Beschleunigungen und Drehungen relativ zum Raumkoordinatensystem
\item  Trajektorienberechnung
\item Genauigkeit hängt vom verwendeten Kreiselsystem ab (Laserkreisel: 1cm Abweichung pro Stunde)
\end{itemize}
\end{itemize}
\subsubsection{Temperatur – Wo ist es warm?}
\begin{itemize}
\setlength\itemsep{0em}
\item Die Haut hat spezifische Kalt- und Warmpunkte
\item Pro cm2 Haut 1-4 Kaltpunkte und 0,4 Warmpunkte
\item Kalt- und Warmsinn haben unterschiedliche statische und dynamische Empfindlichkeiten
\item Empfindlichkeit steigt mit der Geschwindigkeit von Temperaturänderungen
\end{itemize}
\subsubsection{Tasten – Schreibtischplatte oder Kaktus...}
\subsubsection{Riechen – Chemie der Stoffe}
\subsubsection{Sehen – Facetten, Kameras und der optische Fluss}

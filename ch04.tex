\emph{Vieles, was Ingenieure bei technischen Sensoren noch für utopisch halten,
hat die Natur mit größter Raffinesse schon realisiert}

\subsection{Sinneseindrücke und Sensoren – Übersicht und Klassifikation}
\subsubsection{Klassifikation von Sinneseindrücken}
\begin{table}[hbt]
\centering
\begin{tabular}{|p{4cm}|p{5cm}|p{5cm}|}
\hline
\textbf{Sinne \& Rezeptortyp} & \textbf{Adäquater Reiz} & \textbf{Sinnesorgane beim
Menschen/ $\Rightarrow$ (Bsp. aus dem Tierreich)} \\
\hline
\hline
\textbf{Mechanische Sinne} \newline Mechanorezeptoren &  &\\
\hline
Tastsinn & Zug, Druck & Tastkörperchen\\
\hline
Strömungssinn & Strömungen in Wasser \& Gasen & $\Rightarrow$ (z.B. Seitenlinienorgan bei Amphibien und Fischen)\\
\hline
Drehsinn & Winkelbeschleunigung & Bogengänge im Innenohr\\
\hline
Lagesinn & Erdbeschleunigung = Schwerkraft & 3 Bogengänge im Innenohr $\Rightarrow$ (z.B. Borstenfeld bei Ameise)\\
\hline
Hörsinn & Schallwellen(= Druckwellen) 20 – 20.000 Hz beim Menschen Ultraschall (über 20 kHz) & Schnecke im Innenohr $\Rightarrow$ (z.B. Hörorgan bei Fledermaus)\\
\hline
\textbf{Chemische Sinne} \newline Chemorezeptoren & & \\
\hline
Geruchssinn & Chemische Stoffe in Gasphase & Riechschleimhaut der Nase\\
\hline
Geschmackssinn & Gelöster Stoff & Geschmacksknospen der Zunge\\
\hline
\textbf{Elektrischer Sinn} \newline Elektrorezeptoren & Elektrisches Feld & $\Rightarrow$(bei elektrischen Fischen)\\
\hline
\textbf{Magnetischer Sinn} \newline Magnetorezeptoren & Magnetfeld & $\Rightarrow$(bei Zugvögeln, Bienen)\\
\hline
\textbf{Lichtsinn} \newline Photorezeptoren & Elektromagnetische Wellen \glqq sichtbares Licht\grqq{} beim Menschen 380-780 nm, UV (unter 380 nm) Polarisiertes Licht & Augen $\Rightarrow$(z.B. Facettenauge der Insekten)\\
\hline
\end{tabular}
\caption{\"{U}bersicht \"{u}ber Sinne, Reize und Sinnesorgane}
\label{tab:Sinnesuebersicht}
\end{table}

\subsubsection{Rezeptoren und Reizverarbeitung}
Siehe Abbildungen \ref{sensinput} bis \ref{reizverarbeitung1}.
\begin{figure}[h!]
	\centering
	\includegraphics[width=0.8\textwidth]{figures/ch07_sensinput.png}
	\caption{Aufnahme, Bewusstmachung und Abgabe von Information}
	\label{sensinput}
\end{figure}
\begin{figure}[h!]
	\centering
	\includegraphics[width=0.6\textwidth]{figures/ch07_reizverarbeitung.png}
	\caption{Reizverarbeitung}
	\label{reizverarbeitung}
\end{figure}
\begin{figure}[h!]
	\centering
	\includegraphics[width=0.6\textwidth]{figures/ch04_reizverarbeitung1.png}
	\caption{Beziehung/Kontrastierung zwischen Reiz, Rezeptor und Aktionspotential}
	\label{reizverarbeitung1}
\end{figure}
\subsubsection{Sensorik in der Technik}
Der Begriff des Sensors (lateinisch \glqq sensus\grqq{} der Sinn) leitet sich ab von den menschlichen Sinnen.
\begin{itemize}
\setlength\itemsep{0em}
\item Auge $\Rightarrow$ Optik, Kameras
\item Nase $\Rightarrow$ Gassensor
\item Tastsinn $\Rightarrow$ Druck, Temperatur
\item Geschmack $\Rightarrow$ chemische Sensoren
\end{itemize}
IEC-Definition für technische Sensoren:
\glqq Ein Sensor ist das primäre Element in einer Messkette, das eine variable Eingangsgröße in ein geeignetes Messsignal umsetzt.\grqq \\ \\
\textbf{Sensoren in der Robotik}:\\
Siehe Abbildungen \ref{klassfkt} bis \ref{intextbsp}.
\begin{figure}[h!]
	\centering
	\includegraphics[width=0.5\linewidth]{figures/ch04_sensrob.png}
	\caption{Klassifikation von Sensoren nach ihrer Funktionsweise}
	\label{klassfkt}
\end{figure}
\begin{figure}[h!]
	\begin{subfigure}{.5\textwidth}
		\includegraphics[width=\linewidth]{figures/ch04_intern.png}
		\caption{Interne Sensoren}
	\end{subfigure}
	\begin{subfigure}{.5\textwidth}
		\includegraphics[width=\linewidth]{figures/ch04_extern.png}
		\caption{Externe Sensoren}
	\end{subfigure}
	\caption{}
	\label{intext}
\end{figure}\\
\begin{figure}[h!]
	\centering
	\includegraphics[width=0.5\linewidth]{figures/ch04_intext.png}
	\caption{Beispiel Interne vs. Externe Sensoren}
	\label{intextbsp}
\end{figure}\\
\textbf{Sensoren und Perzeption}:\\
Siehe \autoref{sensperz}.
\begin{figure}[h!]
	\centering
	\includegraphics[width=0.6\textwidth]{figures/ch04_sensperz.png}
	\caption{}
	\label{sensperz}
\end{figure}
\newpage
\textbf{Anforderungen an technische Sensorik:}
\begin{itemize}
\setlength\itemsep{0em}
\item Genauigkeit 
\item Präzision 
\item Betriebsbereich 
\item Antwortgeschwindigkeit 
\item Kalibrierung 
\item Zuverlässigkeit 
\item Kosten
\item einfache Installation
\item Linearität
\item Auflösung
\item Reproduzierbarkeit
\item Querempfindlichkeit
\item Alterung
\item Ansprechverhalten
\item[$\Rightarrow$] Ausgehend von der Aufgabenstellung und des Ortes der Integration wird der Sensor ausgewählt; jedoch können meistens nur 2-3 Eigenschaften auf einmal erfüllt werden da viele sich gegenseitig ausschließen; man muss auf den angestrebten Anwendungsfall optimieren
\end{itemize}

\textbf{Bionik in der Sensorik:}
\begin{itemize}
\setlength\itemsep{0em}
\item Biologische Sensoren sind...
\begin{itemize}
\setlength\itemsep{0em}
\item[...] hoch spezialisiert auf ihren Anwendungsbereich,
\item[...] hoch miniaturisiert \& parallel,
\item[...] in großen \glqq Stückzahlen\grqq{} in Sensorfeldern angeordnet.
\end{itemize}
\item Schwerpunkte für eine technische Umsetzung:
\begin{itemize}
\setlength\itemsep{0em}
\item Sensorprinzip
\item Datenauswertung
\end{itemize}
\item Kopieren der Natur an dieser Stelle (noch) unmöglich! (Parallelität, Datenrate wäre zu hoch für Verarbeitung)
\end{itemize}
\subsection{Sensorsysteme in Biologie und Technik – Eine vergleichende Einführung}
\subsubsection{Hören – Von der Fledermaus zum Ultraschallsensor}
\textbf{Menschliches Gehör}:
hat sehr guten Frequenzbereich, aber spezialisiert:
\begin{itemize}
\item Ohrmuschel: unterstützt räumliche Ortung
\item Mittelohr: Impedanzwandlung
\item Innenohr: Umwandlung von Schallsignalen in Neuronenimpulse; med. Konstruktion von Knöchelchen
\end{itemize}
\begin{figure}[h!]
	\begin{subfigure}{.5\textwidth}
		\includegraphics[width=\linewidth]{figures/ch04_schall.png}
		\caption{Schallaufnahme und -weiterleitung}
	\end{subfigure}
	\begin{subfigure}{.5\textwidth}
		\includegraphics[width=\linewidth]{figures/ch04_cochlea.png}
		\caption{Cochlea (H\"{o}rschnecke) Details}
	\end{subfigure}
	\caption{}
	\label{hoeren}
\end{figure}
\textbf{Ultraschall – Die Fledermaus als Vorbild:}
\begin{itemize}
\setlength\itemsep{0em}
\item Hoch entwickeltes Echoorientierungssystem
\item Kehlkopf erzeugt Ultraschalltöne im Bereich von 20 - 200 kHz
\item Echos werden von den Ohren empfangen und ausgewertet
\item Fähigkeit sich bei absoluter Dunkelheit zu orientieren:
\begin{itemize}
\setlength\itemsep{0em}
\item Distanz, Richtung, Form
\item Größe, Struktur und
\item Eigenbewegung der reflektierenden Ziele
\end{itemize}
\item Zum Teil auch Klassifikation der Beute
\end{itemize}
\textbf{Sonar:}
\begin{itemize}
\setlength\itemsep{0em}
\item Echoorientierung durch Serien kurzer Klicklauten (bis zu 130 kHz)
\item können verschiedene Fische, Pflanzen und Gegenstände unterscheiden (auch Minen unter dem Sand)
\item Echoorientierung nicht angeboren
\end{itemize}

\textbf{Ultraschall – Technische Umsetzung}
\begin{itemize}
\setlength\itemsep{0em}
\item Einsatz: 
\begin{itemize}
\setlength\itemsep{0em}
\item Abstandsmessung
\item Hinderniserkennung (Nahbereich)
\item Kartographieren der Umgebung (Fernbereich)
\end{itemize}
\item Eigenschaften:
\begin{itemize}
\setlength\itemsep{0em}
\item akustische Wellen im Frequenzbereich oberhalb 20 kHz
\item Ausbreitung ist nur in Materie möglich
\item beim Auftreffen einer Schallwelle auf eine Grenzfläche zwischen zwei Medien teilt sich die Welle in eine reflektierten und transmittierten Anteil auf 
\item Ausbreitungsgeschwindigkeit von Schallwellen in Luft (20°C): 343,8 m/s
\item Distanzbestimmung über Laufzeitmessung eines Schallsignals
\end{itemize}
\item Nachteile:
\begin{itemize}
\setlength\itemsep{0em}
\item Schallgeschwindigkeit ist abhängig von Temperatur, Luftfeuchtigkeit und Druck
\item Reflexion und Streuung an verschiedenen Oberflächen kann Messergebnisse verfälschen
\item es existiert eine kürzeste und längste messbare Entfernung
\end{itemize}
\end{itemize}
\subsubsection{Gelenkwinkel – Wie ist die Extremität orientiert?}
\begin{figure}[h!]
	\centering
	\includegraphics[width=0.6\textwidth]{figures/ch04_gelenkwink.png}
	\caption{Geschwindigkeit/Winkel der Gelenkbeugung beim Menschen}
	\label{gelwink}
\end{figure}
\textbf{Gelenkwinkelsensoren}
\begin{enumerate}
\setlength\itemsep{0em}
\item Messprinzipien (Ohmisch, Induktiv, Magnetisch, Photoelektrisch )
\item Interpolation
\item Absolute Positionsmessung (Anschlag, Indexkanal, Potentiometer, Graycodierung, Quasiabsolutcode)
\end{enumerate}

\textbf{Interpolation von Zwischenwerten:}
\begin{itemize}
\setlength\itemsep{0em}
\item Erhöhung der Auflösung
\item Aus analogem Signal (egal bei welchem Sensor) können Zwischenstufen interpoliert werden
\end{itemize}

\textbf{Absolute Positionsmessung}
\begin{itemize}
\setlength\itemsep{0em}
\item Relative optische Encoder:
\begin{itemize}
\setlength\itemsep{0em}
\item Nur relative Ticks
\item Absoluten Winkel nur durch Anschlag, Indexkanal
\item Pro: einfacher \& robuster Sensor, kostengünstig
\item Contra: Komplexere Reset-Routine, nur einmalige Referenzierung
\end{itemize}
\item Absolute optische Encoder: 
\begin{itemize}
\setlength\itemsep{0em}
\item Absolute Position in jedem Schritt
\item Spezielle Codierung: z.B. Graycode
\item Pro: Kein extra Reset notwendig, kein Encoder-Drift möglich
\item Contra: Viele optische Spuren notwendig, komplexere Sensorauswertung, teurer \& empfindlicher
\end{itemize}
\item Gelenkwinkelerfassung mit Quasiabsolutwertgeber:
\begin{itemize}
\setlength\itemsep{0em}
\item Problematik: Absolute Gelenkposition notwendig; Bewegung zu einem Referenzpunkt nicht möglich; Gray codierte Sensoren schwer integrierbar und teuer
\item Vorteil: Nur 2 optische Spuren; Inkrementalcode + einfacher Absolutcode; Leicht integrierbar (Größe skalierbar, Runder / Längscode, Reflex / Gabellichtschranke)
\item Nachteil: Bewegen des Gelenks um Absolutcode auszulesen; Software zur Positionsberechung
\item Positionscode durch zwei Spuren abschnittsweise eindeutig
\end{itemize}
\end{itemize}
\subsubsection{Gravitation bzw. Beschleunigung – Innenohr gegen Inertialsystem}
Das Gleichgewichts-(Vestibular-)organ des Menschen enthält Bogengänge (siehe \autoref{vest}): drei Kreisel im Ohr, welche jeweils um 90° gegeneinander verschoben sind. So ist Neigung in allen drei Raumrichtungen messbar. Ein weiteres Beispiel aus der Natur sind Borstenfelder zur Messung des Schwersinns bei der Ameise. Die Borstenverformung ist messbar, so können Insekten Gravitation kompensieren.
\begin{figure}[h!]
	\centering
	\includegraphics[width=0.4\textwidth]{figures/ch04_vestibular.png}
	\caption{Gleichgewichts-(Vestibular-)organ des Menschen}
	\label{vest}
\end{figure}
\noindent
\textbf{Inertialsysteme (Kreiselsysteme)}
\begin{itemize}
\setlength\itemsep{0em}
\item Realisierbar mit:
\begin{itemize}
\setlength\itemsep{0em}
\item Mechanischen Kreiseln
\item 3 Beschleunigungssensoren, 3 Piezo- oder Laser-Kreisel
\end{itemize}
\item Messwerte:
\begin{itemize}
\setlength\itemsep{0em}
\item Beschleunigung in 3 Raumachsen
\item Winkelgeschwindigkeit um 3 Raumachsen
\end{itemize}
\item Messergebnis:
\begin{itemize}
\setlength\itemsep{0em}
\item Bestimmung der Gravitationsrichtung
\item Bestimmung der relativen Beschleunigungen und Drehungen relativ zum Raumkoordinatensystem
\item  Trajektorienberechnung
\item Genauigkeit hängt vom verwendeten Kreiselsystem ab (Laserkreisel: 1cm Abweichung pro Stunde)
\end{itemize}
\end{itemize}
\autoref{inertsys}  zeigt ein Inertialsystem. Es enthält drei Rotationssensoren, drei Beschleunigungssensoren sowie Magnetkompass. Die Messung (und somit auch ihr
Fehler) wird hierbei zweimal aufintegriert, was bei der Positionsberechnung nicht zu vernachlässigen ist.
\begin{figure}[h!]
	\centering
	\includegraphics[width=0.8\textwidth]{figures/ch04_inertsys.png}
	\caption{Inertialsystem}
	\label{inertsys}
\end{figure}
\subsubsection{Temperatur – Wo ist es warm?}
\textbf{Temperaturmessung beim Menschen} (vgl. \autoref{temp})
\begin{figure}[h!]
	\centering
	\includegraphics[width=0.4\textwidth]{figures/ch04_temp.png}
	\caption{Thermorezeptoren}
	\label{temp}
\end{figure}
\begin{itemize}
\setlength\itemsep{0em}
\item Die Haut hat spezifische Kalt- und Warmpunkte
\item Pro $\text{cm}^2$ Haut 1-4 Kaltpunkte und 0,4 Warmpunkte
\item Kalt- und Warmsinn haben unterschiedliche statische und dynamische Empfindlichkeiten
\item Empfindlichkeit steigt mit der Geschwindigkeit von Temperaturänderungen; sehr starke Gewöhnung an Signale
\end{itemize}
\noindent
\textbf{Der Feuerkäfer}
\begin{itemize}
\setlength\itemsep{0em}
\item wird vom Feuer angezogen und legt Eier in verbrannte Bäume welche nach Brand \glqq wehrlos\grqq{} sind; kann beim Anflug heiße von kühlen Stellen unterscheiden
\item Besitzt \textbf{photomechanische Rezeptoren}
\begin{itemize}
\item winzige Sinnesborsten auf dem Chitinpanzer
\item Auftreffen von Wärmestrahlung bewirkt Bewegung von Kügelchen durch physikalische Resonanz-Effekt $\Rightarrow$ Nervenimpuls
\item Sinnesorgane reagieren über viele km auf kleinste Temperaturunterschiede
\end{itemize}
\end{itemize}
\noindent
\textbf{Temperaturmessung in der Technik}
\begin{itemize}
\item Thermoresistive Sensoren: Materialien, die ihren Widerstand in Abhängigkeit von Temperatur verändern oder sich verformen; Einsatz vor allem bei sicherheitskritischen Schaltkreisen (z.B. Feuermelder)
\begin{itemize}
\item Metalle
\begin{itemize}
\item Temperaturabhängigkeit des Widerstands zur Messung der Temperatur
\item bereichsweise linear
\item häufig wird Platin als Messwiderstandsmaterial verwendet
\item Platin wird im Bereich -200°C bis +500°C eingesetzt
\item als Drähte oder als Metallschichtwiderstände in Dünnfilmtechnik ausgeführt
\item Messstrom nimmt Einfluss auf die Genauigkeit
\end{itemize}
\item Keramikwerkstoffe
\begin{itemize}
\item Nichtlinear
\item PTC-Widerstände (Kaltleiter, dotierter polykristalliner Titanatkeramik)
\item NTC-Widerstände (Heißleiter, polykristalline Mischoxidkeramik)
\end{itemize}
\end{itemize}
\item Thermoelektrische Sensoren:
\begin{itemize}
\item zwei verschiedene Metalle in innigen Kontakt
\item zwischen Enden kann eine Spannung (Thermospannung) gemessen werden
\item Thermospannung ist proportional zur Temperatur des Kontaktes
\end{itemize}
\end{itemize}
\subsubsection{Tasten – Schreibtischplatte oder Kaktus...}
\textbf{Menschliche Haut}:\\
Die menschliche Haut weist spezialisierte Tastrezeptoren auf. Die Hauptrezeptoren beim Menschen (in \autoref{tast} dargestellt) sind
\begin{enumerate}
\setlength\itemsep{0em}
\item Berührung
\item Druck
\item Vibration
\item Berührung
\item Druck
\item Dehnung
\end{enumerate}
\begin{figure}[h!]
	\centering
	\includegraphics[width=0.4\textwidth]{figures/ch04_tast.png}
	\caption{Tastende Sensorik}
	\label{tast}
\end{figure}
\begin{figure}[h!]
	\centering
	\includegraphics[width=0.4\textwidth]{figures/ch04_tast1.png}
	\caption{Reaktion der Hautrezeptoren für Druck (1), Berührung (2) und Vibration (3)}
	\label{tast1}
\end{figure}
\textbf{Fühler von Insekten}:
\begin{itemize}
\item Erfassung der Umgebung mittels Fühlern (können einen torusartigen Bereich abdecken; häufig dort auch Geruchssinn)
\item[$\rightarrow$] bei LAURON: wenn kein Bodenkontakt wird erst gesucht
\item Anpassung der Laufbewegung anhand der Fühler; Beine folgen Fühlern
\item Zyklische Fühlerbewegung synchron mit dem Laufmuster
\end{itemize}
\textbf{Künstlicher Schnurrbart}:
\begin{itemize}
\item Schnurrbart bei Mäusen zum Ertasten der Umgebung
\item Technische Umsetzung: Feines Haar (menschliches Haar) auf beweglichem Mikrofon
\end{itemize}
\textbf{Dehnmessstreifen als Kraftsensoren}
\begin{itemize}
\item Dehnmessstreifen verformt sich bei Krafteinwirkung
\item[$\rightarrow$] Veränderung des ohmschen Widerstands
\item Zusammenhang: $\frac{\Delta R}{R} = \kappa \cdot \varepsilon$ mit
\begin{itemize}
\item $\varepsilon = \frac{\Delta l}{l}$ \textit{Längenänderung}
\item $\kappa$ \textit{Dehnungsempfindlichkeitskonstante}
\end{itemize}
\item bei Lauron:  3D Kraftvektor als Messwerte, in $Z$- sowie $X-$ und $Y-$Richtung
\begin{itemize}
\item hochpräzise, empfindlich, aufwändige Auswerteelektronik, anfällige Kontaktierung
\end{itemize}
\end{itemize}
\textbf{Taktile Sensoren}: Kraftmessung über die Nutzung physikalischer Effekte von
Materialien. Man unterscheidet folgende Klassen taktiler Sensoren:
\begin{itemize}
\item Piezoresistiv
\item Piezoelektrisch
\item Kapazitiv
\item Induktiv
\item Optisch (Reflektion, Transduktion, Polarisation)
\end{itemize}
\textbf{Touchscreens} können mittels zahlreicher, unterschiedlicher Technologien umgesetzt werden:
\begin{itemize}
\item Resistive Touchscreens
\begin{itemize}
\item Zwei leitfähige, transparente Schichten
\item Gleichspannung an jede Schicht anlegen
\item An Berührpunkt ist Spannung in Schichten gleich
\item Über Spannungsteiler kann Berührpunkt ermittelt werden
\end{itemize}
\item Kapazitive Touchscreens
\begin{itemize}
\item Beschichtete Folie mit transparentem Metalloxid
\item Über Wechselspannung entsteht elektrisches Feld
\item Bei Berührung fließt kleiner Entladestrom
\item Dominierende Technologie: Smartphones (Apple, Samsung, HTC,...)
\end{itemize}
\item Induktion, Akustische Oberflächenwelle, Infrarote Gitter, Hybride
\end{itemize}
\subsubsection{Riechen und Geschmack – Chemie der Stoffe}
Das menschliche Riechorgan ist in \autoref{riech} dargestellt.
\begin{figure}[h!]
	\centering
	\includegraphics[width=0.7\textwidth]{figures/ch04_riech.png}
	\caption{Nasenhöhle mit Riechorgan/ Riechepithel: Mehrere Dutzend bis mehrere Tausend Rezeptortypen}
	\label{riech}
\end{figure}\\
\textbf{Feuchtigkeitssensoren} (vgl. \autoref{feuchtsens}):
\begin{figure}[h!]
	\centering
	\includegraphics[width=0.6\textwidth]{figures/ch04_feuchtsens.png}
	\caption{Prinzip Feuchtigkeitssensoren}
	\label{feuchtsens}
\end{figure}
\begin{itemize}
\item Messprinzip erfolgt über Kapazitätsmessung des Dielektrikums
\item Schaltung ist aufwendig, wegen benötigtem Wechselspannung
\end{itemize}
\newpage
\textbf{Gassensor-Arrays}:
\begin{itemize}
\item KArlsruher MIkroNAse -- KAMINA (siehe \autoref{kamina}): 
\begin{itemize}
\item Entwicklung eines speziellen Gassensor-Mikroarrays
\item Chip, monolithische Fertigung
\item Schnelles Probeentnahmemodul sowie spezielle Aufbereitung
\item Lernalgorithmus, Aufzeichnung von Feature-Vektorem für verschiedene Gerüche $\rightarrow$ Matching der charakteristischen Kurven
\item Software zur Online-Analyse
\item Optimierung der Baugröße
\end{itemize}
\item Spezielle Gassensor-Mikroarrays als \glqq electronic noses\grqq{} zur Weinunterscheidung \\(siehe \autoref{wein}):
\begin{itemize}
\item fruity, floral, herbaceous, vegetative, spicy, smoky
\end{itemize}
\end{itemize}
\begin{figure}[h!]
	\begin{subfigure}{.3\textwidth}
		\includegraphics[width=\linewidth]{figures/ch04_kamina.png}
		\caption{KAMINA}
		\label{kamina}
	\end{subfigure}
	\begin{subfigure}{.6\textwidth}
		\includegraphics[width=\linewidth]{figures/ch04_wein.png}
		\caption{Gassensor-Arrays zur Weinunterscheidung}
		\label{wein}
	\end{subfigure}
	\caption{}
\end{figure}
\subsubsection{Sehen – Facetten, Kameras und der optische Fluss}
\begin{figure}[h!]
	\begin{subfigure}{.31\textwidth}
		\includegraphics[width=\linewidth]{figures/ch04_auge.png}
		\caption{Menschliches Auge}
		\label{auge}
	\end{subfigure}
	\begin{subfigure}{.5\textwidth}
		\includegraphics[width=\linewidth]{figures/ch04_netzhaut.png}
		\caption{Gassensor-Arrays zur Weinunterscheidung}
		\label{netzhaut}
	\end{subfigure}
	\caption{Menschlicher Sehsinn}
\end{figure}
Die \textbf{Netzhaut} enthält zwei Rezeptortypen:
\begin{itemize}
\item Stäbchen: Hell – Dunkel empfindlich (gestrichelte Linie)
\item Zapfen: 3 Arten mit unterschiedlichen Empfindlichkeiten (vgl. \autoref{zapfen})
\end{itemize}
\begin{figure}[h!]
	\centering
	\includegraphics[width=0.4\textwidth]{figures/ch04_zapfen.png}
	\caption{Trichromatisches Farbsehen}
	\label{zapfen}
\end{figure}
\textbf{Facettenauge bei Insekten}:
\begin{figure}[h!]
	\centering
	\includegraphics[width=0.25\textwidth]{figures/ch04_facette.png}
	\caption{Facettenauge}
	\label{facette}
\end{figure}
Die Navigation von Insekten erfordert \glqq \textbf{Sensorfusion}\grqq: 
\begin{itemize}
\item Biene orientiert sich am Erdmagnetfeld
\item Berücksichtigung von Sonnenstand und Sonnenweg
\item Polarisation des Himmellichtes
\item Geschwindigkeit der Bilder des linken + rechten Auges
\item ermöglicht Flug mit geringstem Energieverbrauch und kürzestem Weg
\end{itemize}
\textbf{Technische Facettenaugen} können folgendermaßen realisiert werden:
\begin{itemize}
\item Kameraarrays mit unterschiedlichen opt. Eigenschaften:
\begin{itemize}
\item Unterschiedliche Spektralfilter, Brennweiten, Fokus
\item Multi-Kamera Fusion, verbesserte Bild- Segmentierung, größere Tiefenschärfe, 3D-Rekonstruktion
\end{itemize}
\item Kameraarrays mit gleichen optischen Eigenschaften:
\begin{itemize}
\item Einsatz von Mikrolinsen zum Zusammensetzen von Bildern
\item Prinzip der Lichtfeld-Fotografie
\item Schärfe und Fokus nachträglich veränderbar
\item weniger Platzbedarf
\item 3D-Rekonstruktion möglich
\end{itemize}
\end{itemize}
Ein Beispiel für die Verwendung \textbf{insektenartiger Odometrie} (Methode zur Lageschätzung) ist \glqq \textbf{Melissa}\grqq{} (siehe \autoref{melissa}).
\begin{figure}[h!]
	\centering
	\includegraphics[width=0.4\textwidth]{figures/ch04_melissa.png}
	\caption{Melissa}
	\label{melissa}
\end{figure}\\
\textbf{Optischer Fluss für Roboter \& Flugdrohnen} (vgl.\autoref{optflow}):
\begin{itemize}
\item Optischer Fluss basiert auf Featuredetection und Tracking
\begin{itemize}
\item Markante Features werden im Bild erkannt (SIFT, SURF, ...) \footnote{siehe z.B. Nourani-Vatan2012, \textit{A Study of Feature Extraction Algorithms for Optical Flow Tracking}}
\item Bewegung des Features von Frame zu Frame erzeugt Flussvektor
\end{itemize}
\item Optischer Fluss für mobile Robotern, Autos und Flugdrohnen
\begin{itemize}
\item Hardware (Kamera) sehr günstig + Optischen Flusses algorithmisch einfach
\item Nur relative Lokalisierungmethode (mit Drift),
\item Keine Lokalisierung in featurelosen, homogenen Umgebungen
\end{itemize}
\item Optical Flow Sensoren als kompakte, günstige \glqq Fertig\grqq- Sensoren verfügbar
\end{itemize}
\begin{figure}[h!]
	\centering
	\includegraphics[width=0.7\textwidth]{figures/ch04_optflow.png}
	\caption{Optischer Fluss}
	\label{optflow}
\end{figure}
\newpage
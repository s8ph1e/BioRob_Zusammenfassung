\emph{The last Frontier of the biological science: Understanding the biological basis of consciousness!}
\subsection{Neuronen, Synapsen und Nervenfasern}
\subsubsection{Das Neuon als Basisrecheneinheit}
\begin{itemize}
\item Zellkörper: enthält Zellkern und biochemischen Apparat (Versorgung der Zelle), von Zellhaut/Membran umgeben
\item Dendriten: (altgr. Dendron = \glqq Baum\grqq) kurze, rohrförmige, stark verzweigte Fortsätze des Zellkörpers zum Empfang von Signalen
\item \textbf{Nervenfaser}: langer, dünner Nerven-
fortsatz, meist nur eine Nervenfaser pro
Nervenzelle (bis zu 1m lang sein beim
Menschen), Weiterleiten der Signale
\item \textbf{Myelin}: besteht aus fettartiger Substanz und Proteinen,
dient zur Beschleunigung des Signal-Transports
\item \textbf{Mitochondrien}: erzeugen die benötigte Energie
\item \textbf{Endoplasmatische Retikulum}: synthetisiert die benötigten Eiweißstoffe
\end{itemize}
\textbf{Rezeptoren und Reizverarbeitung}
\begin{figure}[h!]
	\centering
	\includegraphics[width=0.4\textwidth]{figures/ch07_sensinput.png}
	\caption{Aufnahme, Bewusstmachung und Abgabe von Information}
	\label{sensinput}
\end{figure}
\textbf{Reizverarbeitung und Informationskodierung}
\begin{figure}[h!]
	\centering
	\includegraphics[width=0.4\textwidth]{figures/ch07_reizverarbeitung.png}
	\caption{Reizverarbeitung}
	\label{reizverarbeitung}
\end{figure}
\textbf{Signale auf Nervenfasern}
\subsection{Hierarchien der Informationsverarbeitung im Gehirn}
\subsubsection{Das Zentrale Nervensystem}
\begin{figure}[h!]
	\centering
	\includegraphics[width=0.4\textwidth]{figures/ch07_zns.png}
	\caption{ZNS}
	\label{zns}
\end{figure}
\subsubsection{Aufteilung der Rindenfeldes}
\begin{figure}[h!]
	\centering
	\includegraphics[width=0.4\textwidth]{figures/ch07_rindenfeld.png}
	\caption{Rindenfeld}
	\label{rindenfeld}
\end{figure}
\subsubsection{Aufteilung der Großhirnhälften}
\begin{figure}[h!]
	\centering
	\includegraphics[width=0.4\textwidth]{figures/ch07_grosshirn.png}
	\caption{Grosshirn}
	\label{grosshirn}
\end{figure}
\subsubsection{Stabheuschrecke – Abstraktion des Steuerungskonzeptes}
\begin{itemize}
\item Intelligente Einheiten
\item Definition der Einheiten über gefordertes Steuerungsverhalten
\item Adaptivität auf den Verhaltensebenen
\item Kommunikation der Einheiten auf einer Verhaltensebene
\item Hierarchische, verteilte Steuerung
\end{itemize}
\subsection{Rechnerarchitektur für biologisch motivierte Roboter}
\subsubsection{Anforderungen an die Rechnerarchitektur}
\begin{figure}[h!]
	\centering
	\includegraphics[width=0.4\textwidth]{figures/ch07_arch.png}
	\caption{Anforderungen an die Rechnerarchitektur}
	\label{arch}
\end{figure}
Probleme bei der Umsetzung:
\begin{itemize}
\item Platzmangel
\item Gewicht der Komponenten
\item Energieversorgung
\item Energieeffizienz
\item Resistenz gegen elektromagnetische Störungen
\end{itemize}
Energieversorgung:\\
Energieversorgung im Falle von Rechnerbauteilen durch elektrische Energie\\
$\rightarrow$ Problem: elektrische Energie lässt sich schlecht kompakt speichern
\begin{itemize}
\item Batterien:
\begin{itemize}
\item  Schwer, Groß
\item Kurze Laufzeit, lange Ladezeit
\item Verbesserung durch LiPo-Akkus, Grundproblem bleibt bestehen
\end{itemize}
\item Brennstoffzellen:
\begin{itemize}
\item werden zunehmend interessanter
\item noch kein breiter kommerzieller Einsatz
\item noch in der Erprobung
\end{itemize}
\end{itemize}
Vergleich – Energiebedarf:
\begin{table}[hbt]
\centering
\begin{tabular}{|p{5cm}|p{11cm}|}
\hline
\textbf{Herkömmlicher PC} & \textbf{Im Vergleich dazu} \\
\hline
\hline
\begin{itemize}
\item PC-System Intel Core i3, 3 GHz mit Monitor ca. 300 Watt
\item Pentium M ohne Monitor ca. 150 Watt
\item Intel Core i5-3450 (3,1 GHz) ca. 77 Watt
\end{itemize} &
\begin{itemize}
\item Mikrocontroller ca. 0,4 Watt (C167 Infineon)
\item DSP ca. 0,55 Watt (DSP56F803 Motorola)
\item CPLD / FPGA ca. 0,7 Watt (Altera Flex 10k10)
\item PC/104-System (Pentium M @1 GHz) ca. 20 Watt (DigitalLogic)
\item PC/104-System (Core 2 Duo @ 2×1,5 GHz) ca. 25 Watt
\item Raspberry Pi (ARM @700 Mhz) 3,5 Watt
\item BeagleBoard (ARM Cortex A8 @1 GHz) 6–10 Watt
\end{itemize}\\
\hline
\end{tabular}
\caption{Vergleich – Energiebedarf}
\label{tab:energie}
\end{table}
\subsubsection{Systemkomponenten – die einzelnen Bausteine}
\subsubsection{Kopplung – Bus-Systeme}
\subsection{Beispielarchitekturen}






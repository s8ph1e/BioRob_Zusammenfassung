\emph{Nichts ist so komplex wie die Bewegungssteuerung in der Biologie... und doch auch so einfach.}
\subsection{Neuronale Kontrolle in der Natur}
\subsubsection{Struktur vs. Parametrisierung}
Grundsätzlich zwei Teile bei einer Steuerung:
\begin{itemize}
\item[1.] Der Aufbau, bzw. das Steuerungsparadigma
z.B. Verhaltensbasiertes System, KNN
\item[2.] Die Methode zum Parametrisieren des Systems
z.B. Empirisch, Reinforcement Learning,
\item[$\rightarrow$] Leider nicht immer eindeutig trennbar!
\end{itemize}
Beispiel: PID - proportional–integral–derivative controller
\begin{itemize}
\item Reglerstruktur:
\begin{itemize}
\item Differentialgleichung: $u(t) = K_P\left[e(t) + \frac{1}{T_N}\int_0^te(\tau)d\tau + T_V\frac{d}{dt}e(t)\right]$
\item Übertragungsfunktion: $\frac{U}{E}(s) = K_P + \frac{K_P}{T_N \cdot s}+K_P\cdot T_V\cdot s = K_P(1 + \frac{1}{T_N \cdot s} + T_V) \cdot s$
\end{itemize}
Parametrisierung: Bestimmen der Werte $K_P, T_V, T_N$
\begin{itemize}
\item Empirisch / Einstellen durch Ausprobieren
\item Einstellregel nach Ziegler/Nichols
\item Einstellregel nach Chien, Hrones und Reswick
\end{itemize}
\end{itemize}
Elementare Bewegungsteuerung in der Natur: Bewegungsabläufe
\begin{itemize}
\item Willkürmotorik
\item Rhythmische Bewegungsabläufe
\item Reflexe
\end{itemize}
Elementare Bewegungsteuerung in der Robotik: Verhaltensbasierte Steuerung der sechsbeinigen Laufmaschine LAURON
\begin{itemize}
\item Höhere Verhalten
\item Haltungskontrolle
\item Reflexe
\end{itemize}
\subsubsection{Zentraler Mustergenerator vs. Reflexe}
Als \textbf{zentraler Mustergenerator} (Central Pattern Generator - CPG) wird
ein neuronales Netzwerk verstanden, welches bei isoliertem zentralen
Netzwerk (also ohne jede Art Sensorfeedback) die zeitlichen Impulse der
rhythmischen Motoraktionen liefert (nach Baessler86). Der Begriff des
(rhythmischen) Mustergenerators schließt darauf aufbauend den zentralen
Mustergenerator sowie die modulierende Sensorinformation ein.\\
\textbf{Reflexe} sind zielgerichtete, sensorgekoppelte Verhaltensweisen, bei
denen motorische sowie vegetative Prozesse ablaufen können. Die
Reflexantwort ist dabei hinsichtlich Latenz, Stärke und Muster eng an die
Intensität der Sensorerregung gekoppelt (nach Schmidt95).\\ 
Beispiele für Mustergeneratoren und Reflexe:
\begin{itemize}
\item Rhythmogenese der Atmung (Respiratory CPG)
\begin{itemize}
\item respiratorische Neuronen in verlängerten Rückenmark
\item 3 rhythmische Aktivitäten
\item Inspirationsphase: Einatmungsmuskulatur wird durch Nervenimpulse zur Kontraktion veranlasst
\item Postinspirationsphase: Neuronale Aktivierung der Einatmungsmuskulatur lässt nach. Somit wird eine passive Ausatmung ermöglicht.
\item Aktive Exspirationsphase: Die Ausatmungsmuskulatur aktiviert.
\end{itemize}
\item Pupillenlichtreflex 
\begin{itemize}
\item Veränderung der Weite der Iris in Abhängigkeit vom Lichteinfall
\item Beide Augen reagieren auch auf einseitigen Reiz
\end{itemize}
\end{itemize}
\subsubsection{Das Neuron: Funktionale Approximation, KNN}
\begin{itemize}
\item Eingabe durch eine Propagierungsfunktion mit Gewichten verknüpft
\item Aktivierungsfunktion definiert Reaktion auf Eingabe
\end{itemize}
Künstliche Neuronale Netze - Historischer Abriss
\begin{itemize}
\item 1943: McCulloch und Pitts beschreiben erstes neurologisches Netz
\item 1949: Hebb'sche Lernregel: aktivierende oder hemmende Wirkung einer Synapse lässt sich als Produkt der prä-
und postsynaptischen Aktivität berechnen
\item 1958: Frank Rosenblatt präsentiert das Perzeptron, welches die Grundlage der heutigen KNNs bildet
\item Heute: Große Vielfalt an Topologien (Strukturen der Netze) und Verbindungsarten (z.B. \glqq spiking neural networks\grqq), spezialisierte Netze
\end{itemize}
Anwendungsgebiete von KNNs in biologisch motivierter Robotik:
\begin{itemize}
\item Robotersteuerungen
\item Regelung komplexer Bewegungsabläufe
\item Sprachgenerierung (Beispiel: NETtalk)
\item Bildverarbeitung und Mustererkennung
\item Spracherkennung
\item virtuelle Agenten
\end{itemize}
KNNs -- Prinzipien:
\begin{itemize}
\item Einfaches Einzelneuron – komplexes Gesamtsystem
\item Hohe Fehlertoleranz
\item Integration sensorischer Massendaten
\item Lernvorgang (bestärkendes, (un)-überwachtes, stochastisches Lernen)
\item Direkte Generierung kontinuierlicher Steuersignale
\item Komplizierte nichtlineare Funktionen erlernbar
\end{itemize}
Netztopologien:
\begin{itemize}
\item Einschichtiges Feed-Forward-Netzwerk (azyklischer gerichteter Graph): eine Eingabeschicht, eine Ausgabeschicht (azyklischer gerichteter Graph)
\item Mehrschichtiges Feed-Forward Netzwerk: 1-$n$ verdeckte Schichten, verbesserte die Abstraktionsfähigkeit, z.B. XOR Problem lösbar
\item Rekurrentes (rückgekoppeltes Netz):
\begin{itemize}
\item direkte Rückkopplungen (\textit{direct feedback})
\item indirekte Rückkopplungen (\textit{indirect feedback})
\item seitliche Rückkoppelungen (\textit{lateral feedback})
\item vollständige Verbindungen (jedes Neuron ist mit jedem anderen Neuron verbunden)
\end{itemize}
\end{itemize}
Neuro-Oszillatoren:
\begin{itemize}
\item Paare aus sich gegenseitig hemmenden und erregenden Neuronen
\item Der Antagonistische Aufbau der Muskeln setzt sich in der Steuerung fort
\item Dienen auf höheren Ebenen als Mustergeneratoren
\end{itemize}
\begin{figure}[h!]
	\centering
	\includegraphics[width=\textwidth]{figures/ch05_NeuroOs.png}
	\caption{Neuro-Oszillator}
\end{figure}
Verhaltensbasierte Systeme
\begin{itemize}
\item Komplexe Verhalten entstehen nicht notwendigerweise aus komplexen Steuerungssystemen
\item Kein Modell sondern direkte Interaktion mit der Umwelt
\item Charakteristika
\begin{itemize}
\item Situatedness: Der Roboter ist als Einheit in seiner Umgebung eingebunden
\item Embodiment: Der Roboter hat eine physikalische Präsenz
\item Emergence: Intelligenz entsteht durch Interaktion mit der Umwelt
\end{itemize}
\end{itemize}
Hierarchie biologisch motivierter Steuerstrukturen (zunehmende Abstraktion und Komplexität)
\begin{itemize}
\item[0.] Mustergeneratoren $\rightarrow$ CPGs, zyklische Bewegungen
\item[1.] Reflexe $\rightarrow$ reaktive Systeme in Robotern
\item[2.] Neuronen $\rightarrow$ KNNs
\item[3.] Verhalten $\rightarrow$ verhaltensbasierte Systeme (vgl. Kap. 9)
\item[$\rightarrow$] es werden aber auch mehr und mehr Parameter!
\end{itemize}
Steuerungsstrukturen parametrieren: Parameteridentifikation und Tuning stellt bei komplexen
Robotern und anspruchsvollen Aufgaben eine zentrale Herausforderung der Entwicklung dar. 
Es gibt verschiedene Verfahren und Methoden:
\begin{itemize}
\item Modellbasierte Parametrierung – \glqq Regler Einstellregeln\grqq
\item Biologische Untersuchungen (z.B. KNN bei Insekten)
\item Empirisches Parametrieren anhand von umfangreichen Versuchen und Expertenwissen
\item Systematische, lernende Ansätze: 
\begin{itemize}
\item Maschinelles Lernen: finde iterativ Optimum im Parameterraum über mehrdimensionalen Gradientenabstieg; verschiedene Verfahren zum Beschleunigen des Lernvorgangs
\item Genetische Algorithmen: Problemrepräsentation durch \glqq Genom\grqq, Mutation und Rekombination
\end{itemize}
\end{itemize}
\subsection{Neuronale Steuerung der Stabheuschrecke}
\subsubsection{Warum Stabheuschrecke als Vorbild?}
Stabheuschrecke (carausius morosus):
\begin{itemize}
\item großer 6-Beiner $\rightarrow$ Statisch stabiles Laufen, Überwinden komplexer Hindernisse, läuft auch Wände hoch, sehr robust
\item Langsam $\rightarrow$ Realisierbar mit verfügbarer Aktorik (z.B. Elektromotoren)
\item Einfacher kinematischer Aufbau $\rightarrow$ Einfache, gut beherrschbare mechanische Konstruktion
\item Trotzdem sehr flexibel $\rightarrow$ Viele Forschungsbereiche und Herausforderungen
\item Weitreichende biologische Untersuchungen vorhanden (z.B. Prof. H. Cruse, Prof. A. Büschges) $\rightarrow$ Koordinationsmechanismen (\glqq Cruse Regeln\grqq)
\item Verteilte dezentrale Steuerung $\rightarrow$ Verteile Hardware
\end{itemize}
\subsubsection{Laufmuster und Cruse-Regeln}
Mehrbeinige Laufmuster werden oft durch die Anzahl der
am Boden verbleibenden (tragenden) Beine klassifiziert:
\begin{itemize}
\item Tripod: zu jedem Zeitpunkt mind. 3 Bein auf Boden
\item Tetrapod: zu jedem Zeitpunkt mind. 4 Beine auf Boden
\item Pentapod: zu jedem Zeitpunkt mind. 5 Beine auf Boden
\item[$\rightarrow$] Verwendung der Begriffe Tetrapod und Tripod
verleitet zu der Annahme der Existenz \textit{fester Gangmustern}.
\item[$\rightarrow$] Es ist aber bekannt, dass Insekten einen \textit{freien
Gang} verwenden.
\end{itemize}
Koordinationsregen nach Cruse:
\begin{itemize}
\item[1.] Hintere Schwingphase hemmt Start vorderer Schwingphase: wenn hinteres Bein in der Luft ist soll Vorderes nicht auch in der Luft sein (vgl. Abb. \ref{CR1})
\item[2.] Beginn der Stemmphase löst vorderes Schwingen aus (vgl. Abb. \ref{CR2})
\item[3.] Beinposition während der Stemmphase reizt Start der Schwingphase: je weiter das Bein sich am Ende seines Arbeitsraumes befindet, desto stärker wird der Start der Schwingphase des Vorderbeins motiviert (vgl. Abb. \ref{CR3})
\item[4.] Position beeinflusst die Position am Ende der Schwingphase (\glqq targeting\grqq)
\item[5.] Erhöhter Widerstand erfordert erhöhte Kraft (Stemmen)
\item[6.] Tread-On-Tarsus Reflex: wenn man der Heuschrecke über den Fuß streift hebt sie ihn, was verhindern soll, dass sie sich selbst auf den Fuß tritt
\end{itemize}
\begin{figure}[h!]
	\centering
	\begin{subfigure}{.3\textwidth}
		\includegraphics[width=\textwidth]{figures/ch05_CR1.png}
		\caption{Regel 1}
		\label{CR1}
	\end{subfigure}
	\begin{subfigure}{.3\textwidth}
		\includegraphics[width=\textwidth]{figures/ch05_CR2.png}
		\caption{Regel 2}
		\label{CR2}
	\end{subfigure}
	\begin{subfigure}{.3\textwidth}
		\includegraphics[width=\textwidth]{figures/ch05_CR3.png}
		\caption{Regel 3}
		\label{CR3}
	\end{subfigure}
	\caption{Cruse-Regeln 1 - 3}
\end{figure}
\begin{figure}[h!]
	\centering
	\includegraphics[width=0.5\textwidth]{figures/ch05_Cruse.png}
	\caption{Cruse-Regeln}
	\label{Cruse}
\end{figure}
\subsection{Oszillatoren als Grundbausteine für natürliche Bewegungssteuerung}
\subsubsection{Rhythmusgeneratoren, Erweiterung des Neuronen Modells}
Als \textbf{zentraler Mustergenerator} (Central Pattern Generator - CPG)
wird ein neuronales Netzwerk verstanden, welches bei isoliertem
zentralen Netzwerk (also ohne jede Art Sensorfeedback) die zeitlichen
Impulse der rhythmischen Motoraktionen liefert (nach Baessler86). Der
Begriff des (rhythmischen) Mustergenerators schließt darauf
aufbauend den zentralen Mustergenerator sowie die modulierende
Sensorinformation ein.
\begin{itemize}
\item neuronales Netz ohne externe Sensorinformation
\item in der Lage rhythmische Muster zu generieren
\item genutzt um rhythmische Motoransteuerungen zu erzeugen
\end{itemize}
\textbf{Reflexe} sind schnelle, automatische und unbewusste Reaktionen,
die von einem sensorischen Stimulus ausgelöst werden. Es sind zielgerichtete, sensorgekoppelte Verhaltensweisen,
bei denen motorische sowie vegetative Prozesse ablaufen können.
Die Reflexantwort ist dabei hinsichtlich Latenz, Stärke und Muster
eng an die Intensität der Sensorerregung gekoppelt (nach Schmidt95).
\begin{itemize}
\item Beispiel: Muskelkontraktion bei Hitze (taktil), Ausweichen vor einem Ball (visuell)
\end{itemize}
\textbf{Taxen} (\textit{sg. Taxis}) sind Verhalten, die ein Tier in Richtung (attraktive) oder
entgegen (aversive) eines Stimulus orientieren.
\begin{itemize}
\item Beispiel: Orientierung von Bienen nach Sonnenstand, Bewegung von Lichtquelle weg bei Asseln
\end{itemize}
\textbf{Feste Muster} sind Reaktionen auf Stimuli, die länger anhalten als die Stimuli selbst.
\begin{itemize}
\item Fluchtreflex hält länger an als Stimulus selbst (\glqq Reflex\grqq{} ist hier eine irreführende Bezeichnung)
\end{itemize}

\subsubsection{Oszillatormodelle: Leaky Integrator, Matsuoka}
\subsection{Anwendungsbeispiele für Oszillatoren}
\subsubsection{Neunauge, Salamander – Schwimmen und Laufen, Tekken – Vierbeiniges Laufen}
\subsubsection{Zusammenfassung \& Diskussion}
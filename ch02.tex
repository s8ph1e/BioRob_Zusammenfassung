\textbf{Anforderung an die Mechanik -- Was hat die Natur an Lösungen zu bieten}

\subsection{Leichtbau (Knochen/Skelett)}
\emph{Biologische Materialien werden für einen bestimmten Zweck und möglichst massearm hergestellt}

\subsubsection{Biologische Grundprinzipien und typische Merkmale}
10 Grundprinzipien natürlicher Konstruktion:
\begin{enumerate}
	\item Hohe Funktionsintegration statt Funktionstrennung (z.B. zahlreiche Funktionen menschlicher Haut)
	\item Ganzheitliche Optimierung
	\item Multifunktionalität statt Monofunktionalität 
	\item Minimaler Einsatz von Energie (Ökologische Nische muss durch bestmöglichen Energieeinsatz ausgefüllt werden)
	\item Nutzung von Fremdenergie (z.B. Nutzung von Sonnenlicht durch Poikilotherme)
	\item Zeitliche Limitierung statt Langlebigkeit (z.B. Eintagsfliegen; ständig nachwachsende Hautschichten)
	\item Hohe Vernetzung / Wechselwirkung
	\item Veränderungen in kleinen Schritten (z.B. Wachstum bei Kindern; inkrementelle Änderung von Generation zu Generation $\rightarrow$ ganz anders in der Robotik)
	\item Feinabstimmung gegenüber der Umwelt
	\item Totales Recycling (z.B. Häutung von Reptilien)
\end{enumerate}
Typische Merkmale biologischer Materialien:
\begin{enumerate}
	\item Materialschichtung während des Entstehens (z.B. Chitinpanzer bei Insekten: zwischen Elementen elastisch/weniger Schichten, außen hart/mehr Schichten)
	\item Streng funktionell (genaue Abstimmung auf Umweltanforderungen)
	\item Funktionelle Kompartimente
	\item Häufig Ultraleicht (Verhältnis Kraft/Gewicht)
	\item Sandwich-Bauweise
	\item Selbstreparabel (z.B. nachwachsende Beine der Stabheuschrecke; wäre tolles Feature für LAURON)
	\item Terminierte Lebensdauer
\end{enumerate}

\subsubsection{Materialauswahl und Reduzierung}
\begin{itemize}
	\item Technische Möglichkeiten zur Gewichtsoptimierung
	\begin{itemize}
		\item Menge (Volumen) des eingesetzten Materials minimieren:
		\begin{itemize}
			\item Materialdicken reduzieren
			\item Aussparungen einfügen (Materialmenge solange reduzieren, dass Funktion gerade noch erhalten wird; z.B. Spaichen beim Fahrrad)
		\end{itemize}
		\item Gewichte (Dichte) des eingesetzten Materials reduzieren
		\begin{itemize}
			\item Austausch des verwendeten Materials (z.B.\ Aluminium statt Stahl, Karbon statt Aluminium, ...)
		\end{itemize}
	\end{itemize}
	\item Stabilität und Belastbarkeit muss erhalten bleiben!
\end{itemize}

\subsubsection{Konstruktionsprinzipien}
\begin{itemize}
	\item Hohe Stabilität bei geringem Gewicht durch Netzwerk aus Knochenlammellen
	\item Hohlbauweise (Rohr mechanisch viel stabiler als ein Volleisenstab)
	\item Extrem stabil (ein menschlicher Oberschenkelknochen kann ca. 1$t$ Längsbelastung aushalten)
\end{itemize}
\textbf{Innenskelett}:
\begin{itemize}
	\item CFE -- Rohre als Knochen
	\begin{itemize}
		\item Kohlenstoffaser--Gewebe--Epoxy Verbundstoff
		\item Leicht maschinell zu fertigen, aber schlecht nachbearbeitbar
		\item Hohe Biege- und Torsionsfestigkeit
		\item Sehr leicht ($1,5 g/cm^3$)
	\end{itemize}
\end{itemize}
\textbf{Exoskelett}:
\begin{itemize}
	\item Chitin
	\begin{itemize}
		\item Exoskelett ist Stützstruktur für Organismen mit stabiler äußerer Hülle
		\item Stickstoffhaltiges Polysaccharid
		\item Primärer Baustoff der Insekten
		\item Unterschiedliche Einlagerungen führen zu unterschiedlichen Härtegraden
		\item Schichtbauweise mit unterschiedlichen mechanischen Eigenschaften
	\end{itemize}
	\item Glasfaser Verbundstoffe
	\begin{itemize}
		\item Verbundwerkstoff aus Gewebematten (Glasfaser, Kohlefaser) und Harz (Epoxyd, Aramid)
		\item Freie Formen möglich
		\item Dicke lässt sich an kritischen Stellen erhöhen an unkritischen senken
		\item Relativ leicht ($2,52 g/cm^3$)
		\item Genutzt z.B. zur Gewichtsoptimierung bei Formel1; aber: viel Handarbeit/teuer
	\end{itemize}
\end{itemize}

\subsubsection{Technische Materialien -- Werkstoffe im Roboterbau}
Begriffe und Definitionen:
\begin{itemize}
	\item Dichte $\rho = \frac{m}{V} = \frac{\text{Masse}}{\text{Volumen}}$
	\item Zugfestigkeit $R = \frac{F_{MAX}}{A} = \frac{Kraft_{MAX}}{Flaeche}$
	\item mechanische Spannung $\sigma = \frac{F}{A} = \frac{Kraft}{Flaeche}$
	\item Dehnung $\epsilon = \frac{\Delta l}{l_o} = \frac{Laengenaenderung}{urspruengliche Laenge}$
	\textit{relative Längenänderung}
	\item Elastizitätsmodul/E-Modul $E = \frac{\sigma}{\epsilon} = \frac{Spannung}{Dehnung} = const$\\
	\textit{Materialkennwert beschreibt Zusammenhang zwischen Spannung und Dehnung}
	\item Bruchfestigkeit: \textit{mechanische Spannung, die unter gleichmäßiger Steigerung der Belastung bei einem Bauteil zum Bruch führt}
\end{itemize}
Unterschiedliche Anwendungen benötigen unterschiedliche Werkstoffe. Häufig ist auch ein Materialmix notwendig.

\begin{figure}[!htb]
    \centering
    \begin{minipage}{.5\textwidth}
        \centering
        \includegraphics[width=\textwidth]{figures/werkstoffe.png}
    \end{minipage}%
    \begin{minipage}{0.5\textwidth}
        \centering
        \includegraphics[width=\textwidth]{figures/werkstoffpruefung.png}
    \end{minipage}
\end{figure}

\subsubsection{Verhältnis: Kraft zu Gewicht}
\begin{itemize}
	\item Klassischer Industrie-Roboter (KUKA KR 1000 TITAN von 2007):
	\begin{itemize}
		\item Traglast $1000$kg
		\item Gewicht $4950$kg
		\item Kraft/Gewicht $0.2$
	\end{itemize}
	\item Mariusz Pudzianowski (\glqq stärkster Mensch der Welt\grqq)
	\begin{itemize}
		\item Traglast $415$kg
		\item Gewicht $138$kg
		\item Kraft/Gewicht ca. $3.0$ (nur kurzfristig!)
	\end{itemize}
	\item Ameise (hier: \textit{Formica polyctena - Waldameise}
	\begin{itemize}
		\item Traglast $350$mg
		\item Gewicht $9$ mg
		\item Kraft/Gewicht $40$
	\end{itemize}
	\item Hauspferd
	\begin{itemize}
		\item Leistung $0.4 - 1.2$ PS (kurzfristige Spitzenleistung bis zu $20$PS)
		\item Gewicht $200-800$kg
		\item Kraft/Gewicht $0.16$
	\end{itemize}
\end{itemize}

\subsection{Kinematische Struktur (Morphologie)}

\subsubsection{Gelenke}
\textbf{In der Natur}
\begin{itemize}
	\item Arten:
	\begin{itemize}
		\item Kugelgelenk
		\item Sattelgelenk
		\item Scharniergelenk
	\end{itemize}
	\item Eigenschaften:
	\begin{itemize}
		\item Selbstschmierend (Knorpel, Gelenkflüssigkeit)
		\item Gekapselt (Kapsel, Muskelgewebe, Haut, Chitinpanzer)
		\item Optimierte Geometrie für die Anwendung (z.B.\ menschliches Knie: geringer Reibwert aber hoher Wirkungsgrad; aber Kugelgelenke in Robotik schwer anzutreiben)
	\end{itemize}
\end{itemize}
\textbf{Kugelgelenk -- Technische Approximation}:
Kugelgelenk der Stabheuschrecke wird durch drei Rotationsgelenke approximiert \\
\textbf{Gelenkskapselung}: Klassische mechanische Kapselung durch Simmer- oder Wellendichtringe meist zu schwer; Gummimembrane nur begrenzt haltbar, daher regelmäßige Wartung erforderlich

\subsubsection{Multisegmentsysteme}
Die \textbf{Wirbelsäule} als Beispiel für ein biologisches Multisegmentsystem ist ein flexibles System mit vielen gekoppelten Einzelelementen.
Ein Beispiel für einen Multisegmentarm ist der \textbf{Festo Rüssel}, ein bionischer Handling Assistent mit 13 pneumatischen Aktuatorn, 11 Freiheitsgraden, $1.8$kg Gewicht und $0.5$kg Traglast.
Er enthält Elastizitäten und durch den Luftdruck zusätzliche Nicht-Linearitäten.
%TODO Folie 43-45?

\subsubsection{Laufmaschine (Invertebrate, Vertebrate)}
Warum die Stabheuschrecke als Vorbild?
\begin{itemize}
	\item 6-Beiner
	\item Langsames Insekt
	\item Einfacher kinematischer Aufbau und trotzdem sehr flexibel
	\item Weitreichende biologische Untersuchungen vorhanden
\end{itemize}
\autoref{ch:02:fig:stabheuschrecke_mechanisch} zeigt ein mechanisches Modell der Stabheuschrecke. 
%TODO Folie 49,51
\begin{figure}
	\centering
	\includegraphics[width=\textwidth]{figures/stabheuschrecke_mechanisch.png}
	\caption{Stabheuschrecke -- Mechanisches Modell}
	\label{ch:02:fig:stabheuschrecke_mechanisch}
\end{figure}\\ \\
\textbf{Sechsbeinige Laufmaschine -- Optimierung}
\begin{itemize}
	\item Berechnung der Dynamik \\
	$M \stackrel{..}{q} + h = J^T_R t + J_T^T f$
	\item Optimierung nach folgenden Kriterien:
	\begin{itemize}
		\item Minimales Biegepotential
		\item Verspannungsfreiheit Beine
		\item Minimale Gelenkzwangsmomente
		\item Optimale Aufstandskraft"=Richtungen
		\item Minimale Antriebsleistung
	\end{itemize}
	\item[$\Rightarrow$] Optimal sind $40\%$ Biegepotential und $60\%$ Verspannung. Das entspricht genau der Stabheuschrecke!
\end{itemize}
\textbf{Vorteil eines langen Zentralkörpers}\\
Je nach Beinanordnung verändert sich die Stabilitätsreserve (vgl. \autoref{sr}).
\begin{figure}
	\centering	
	\includegraphics[width=\textwidth]{figures/zentralkoerper.png}
	\caption{}
	\label{sr}
\end{figure}
\subsubsection{Lokomotionsarten} %TODO Beispiele 55-58
\begin{itemize}
	\item Zweibeiniges Laufen: Menschen, Vögel
	\item Vierbeiniges Laufen: Sehr viele unterschiedliche Morphologien/Kinematiken vierbeiniger Läufer
	\begin{itemize}
		\item Reptilienartiges Laufen: Schildkröten, Eidechsen, Geckos (Tier trägt viel von seiner Gewichtskraft)
		\item Amphibienartiges Laufen: Salamander, Frösche
		\item Säugetierartiges Laufen:
		\begin{itemize}
			\item Zehengänger: z.B. Hund, Katze; Stand ohne Energieverbrauch bei komplett arretierten Beinen
			\item Spitzengänger: z.B. Pferde, Elefanten
			\item Sohlengänger: z.B. Bären, Kängurus; niederigerer Hebelarm aber immer noch angespanntes System im Stehen; führt zu hohem Energieverbrauch, aber dafür hoch agil/dynamisch; für 						  Elefanten wäre z.B. Sohlengang nicht möglich
		\end{itemize}
		\item[$\rightarrow$] Anordung der Beine am Zentralkörper ist eine wichtige Designentscheidung
	\end{itemize}
	\item Sechsbeiniges Laufen: Stabheuschrecke, Ameisen, Kakerlaken, Gottesanbeterinnen
	\item Achtbeiniges Laufen: Spinnen, Skorpione, Krebse
	\item Mehrbeiniges Laufen: Hundertfüßler, Tausendfüßler
\end{itemize}

Bei Vertebraten (Wirbeltieren) hat die Wirbelsäule eine Einfluss beim Auf- und Abfußen.

\begin{figure}
	\centering
	\begin{subfigure}{.8\textwidth}
		\centering
		\includegraphics[width=\textwidth]{figures/befestigungswinkel.png}
	\end{subfigure}\par\medskip
	\begin{subfigure}{.5\textwidth}
		\centering
		\includegraphics[width=\textwidth]{figures/befestigungswinkel_1.png}
	\end{subfigure}
	\caption{Vergleich der Befestigungswinkel im Beinansatzpunkt für technische und biologisch-inspirierte Roboter, hier LAURON}
\end{figure}
\newpage
\subsubsection{Arbeitsraumvergrößerung durch Anstellwinkel}
\begin{enumerate}
	\item Analyse des Arbeitsraums für einen Laufroboter
	\begin{itemize}
		\item DLR-Crawler -- Sechsbeiniger Laufroboter
		\item Finger der DLR-Hand II als Beine
		\item Kinematisch eingeschränkter Arbeitsraum
		\item Achsensymmetrische Beinanordnung
		\item \textbf{Arbeitsraumoptimierung durch Anstellwinkel der Beine}
	\end{itemize}
	\item Durch den Anstellwinkel $\alpha = 30°$ vergrößert sich der nutzbare Arbeitsbereich
	\begin{itemize}
		\item $+60\%$ nutzbarer Arbeitsbereich
		\item Größere Flexibilität (Schritthöhe, Schrittweite)
		\item Ähnlichkeit zum Anstellwinkel der Stabheuschrecke
		\item Komplexere mechanische Anbringung
	\end{itemize}
\end{enumerate}
%TODO 67-69


\subsection{Energie--Effizienz}
\emph{Billig bedeutet in der Biologie immer \glqq mit geringem Energieaufwand erreichbar\grqq}
\subsubsection{Reduktion des Strömungswiderstandes}
Strömungswiderstandskoeffizient $c_w = \frac{F_w}{q \cdot A} = \frac{\text{Widerstandskraft}}{\text{Staudruck der Anstroemung} \cdot \text{Referenzflaecheninhalt}}$
\begin{itemize}
	\item Strömungswiderstand und Form am Beispiel Pinguin
	\begin{itemize}
		\item er kann bis zu einer Geschwindikeit von $7$m/s schwimmen, das entspricht $100$m/s in der Luft
		\item er taucht täglich bis zu $100$km weit und $400$m tief, mit einer Magenfüllung Plankton pro Tag (ca. $1500$ bis $2000$km mit $1$l Benzin)
		\item ermöglicht durch einen $c_w$-Wert von $0.07$ und strömungsgünstigen Flossen
	\end{itemize}
	\item Strömungsoptimierte Fahrzeugkarosserie: Bionic Car von Mercedes Benz
	\begin{itemize}
		\item $140$PS, $4.3$ Liter/$100$km, $c_w$-Wert $0.19$
		\item Pate für die Entwicklung war der Kofferfisch, der trotz seiner kantigen Form hervorragende Strömungseigenschaften hat
	\end{itemize}
	\item Strömungswiderstand und Oberfläche
	\begin{itemize}
		\item Haifischhaut und Schmetterlingsschuppen bilden Streichlinien
		\item Durch die Wasser- bzw. Luftverwirbelung entsteht ein sehr strömungsgünstiges Profil
		\item Der Energieverbrauch von Verkehrsflugzeugen ließe sich damit um $4$\% reduzieren
	\end{itemize}
\end{itemize}

\subsubsection{Energieeffizienz beim Laufen}
\textbf{Specific Resistance} beschreibt die Effizienz der Lokomotion:
\begin{equation}
 \varepsilon = \frac{P_{lok}(t)}{G \cdot v(t)}
\end{equation}
mit Leistung der Fortbewegung $P_{lok}(t)$, Laufgeschwindigkeit $v(t)$ und Gewichtskraft $G$. 
Sie ist dimensionslos, unabhängig von der Art des Fahrzeugs oder des Roboters und gut vergleichbar.
Die \emph{Metabolic Cost of Transport} ist formelmäßig ähnlich, was einen Vergleich zur Natur ermöglicht.\\
\noindent
In der Robotik gebräuchliche Form (integriert über die Zeit): $\varepsilon = \frac{E}{G \cdot s}$.

\begin{figure}
	\includegraphics[width=\textwidth]{figures/specific_resistance.png}
	\caption{Specific Resistance: Für Hund, Vogel, etc. wurde der $CO_2$-Gehalt und somit indirekt der Energieverbrauch gemessen. Kein System konnte sich unterhalb der G\&K-Linie bewegen.}
\end{figure}

Effizienzsteigerung durch Gewichtsreduktion und/oder Geschwindigkeitssteigerung

\subsubsection{Beispiele für Energiebilanzen in der Natur}
\begin{itemize}
	\item Ein \textbf{Gepard} läuft beim Erlegen seiner Beute anaerob. Nach dem schlagen ist er so erschöpft, dass er $50\%$ seiner Beute an Aasfresser verliert.
	\item Das Fell von \textbf{Eisbären} absorbiert Sonnenlicht und hält damit einen hohen Anteil der Sonnenenergie als Wärme zurück.
	\item Ein biologische \textbf{Muskel} wächst nur dort wo er benötigt wird; damit wird sowohl der Energieverbrauch beim Wachsen als auch bei der Bewegung optimiert.
\end{itemize}
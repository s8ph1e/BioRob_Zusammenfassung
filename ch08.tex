\emph{Planen bedeutet zu Vermeiden herauszufinden, was man als nächstes zu tun hat}
\subsection{Verhalten im Tierreich: Neurologisch, Psychologisch, Ethologisch}
\emph{It is said that the limbic system of the brain controls the four Fs: Feeding, Fighting, Fleeing and Reproduction}
\subsubsection{Neurologisch -- Vektorfeldaddition im Frosch-Rückenmark}
\begin{itemize}
	\item Neurobiologische Hypothese der motorischen Kontrolle durch Vektorfelder
	\item Bizzi (MIT) hat gezeigt, dass Gliedmaßenbewegung in Fröschen im Rückenmark codiert abgelegt sind
	\item Mikrostimulation im Rückenmark generiert eine Bewegung zu einem spezifischen Punkt
	\item Planung im ZNS wird auf solche Reize umgesetzt
\end{itemize}

\subsubsection{Psychologisch -- Verhaltens-, Gestalt- und Kognitive "-Psychologie}
\begin{itemize}
	\item \textbf{Verhaltens-Psychologie:}
	Verhalten wird durch Beobachtung definiert.
	Alles wird auf ein Stimulus-Antwort Schema zurückgeführt.
	\item \textbf{Gestalt-Psychologie:}
	Einbeziehung der Physik.
	Verhalten entsteht als direkte Konsequenz aus der Struktur der physischen Umgebung.
	\item \textbf{Kognitive Psychologie:}
	Einbeziehung des Wissens.
	Vereinheitlichte Methode, um die Beziehung von Aktion und Perzeption zu erklären.
\end{itemize}

\subsubsection{Ethologisch -- Reflexe, Taxe und feste Muster}
\begin{itemize}
	\item Ethologie bezeichnet die Beobachtung von tierischem Verhalten in seinem natürlichen Lebensraum.
	\item Das Tier wird als Teil des Gesamtsystems angesehen, das auch die Umgebung mit einbeziehen muss.
	\item Klassifizierung in \textbf{Reflexe}, \textbf{Taxe} und \textbf{Feste  Muster}.
	\item Motivierte Verhalten als Antwort auf interne Stimuli.
	\item Schematheorie (Lorenz): Zusammenfassung von komplizierten Kombinationen aus Reflexen, Taxen und Mustern zu Schemata.
	\item Prinzip der \glqq Ökologische Nische \grqq (McFarland). Ein Tier überlebt, weil es eine Nische gefunden hat.
\end{itemize}

Drei Klassen von Verhalten im Tierreich:
\begin{itemize}
	\item \textbf{Reflexe}: schnelle, automatische und unbewusste Reaktionen, die von einem sensorischen Stimulus ausgelöst werden.
	\item \textbf{Taxe}: Verhalten, die ein Tier in Richtung (attractive) oder entgegen (aversive) eines Stimulus orientieren.
	\item \textbf{Feste Muster}: Reaktionen auf Stimuli, die länger anhalten als die Stimuli selbst.
\end{itemize}

\subsection{Verhaltensbasierte Robotik -- Übersicht  und Einführung}
\emph{Es gibt zwei Möglichkeiten beim Entwurf von Robotersteuerungen: Erstens, so simpel, dass es keine offensichtlichen Mängel gibt; Zweitens, so komplex, dass es keine offensichtlichen Mängel gibt.}
\subsubsection{Motivation -- Warum verhaltensbasierte Architekturen}
\begin{itemize}
	\item Komplexe Verhalten entstehen nicht notwendigerweise aus komplexen Steuerungssystemen
	\item Die Natur ist das beste Vorbild
	\item Simplizität ist eine Tugend
	\item Robustheit bei verrauschten Sensorwerten ist ein Designziel
	\item Systeme sollten inkrementell aufgebaut werden können
	\item Alle Berechnungen Onboard
	\item Roboter sollten billig sein
\end{itemize}

\subsubsection{Roboter Steuerungs"=Spektrum -- Von reaktiv bis deliberativ}
\textbf{Roboter Steuerungs-Spektrum:}
\begin{center}
	\includegraphics[width=0.5\textwidth]{figures/steuerungs_spektrum.png}
\end{center}

\textbf{Hierarchische Steuerung:}
\begin{itemize}
	\item Hierarchische Anordnung von Steuerungselemente, wobei höhere Ebenen Unterziele für niedrigere erzeugen
	\item Eigenschaften
	\begin{itemize}
		\item Es wird vorausgesetzt, dass die zu erfüllende Aufgabe zerlegbar ist
		\item Ein Weltmodell auf jedem Level garantiert die korrekte Ausführung
	\end{itemize}
\end{itemize}

\textbf{Reaktive Steuerung:}
\begin{itemize}
	\item Bedeutet vereinfacht eine enge Kopplung zwischen Perzeption und Aktion
	\item Definierte Schlüsselelemente:
	\begin{itemize}
		\item \textbf{Embodiment:} Der Roboter hat eine physikalische Präsenz
		\item \textbf{Situatedness:} Der Roboter ist als Einheit in seiner Umgebung eingebunden
		\item \textbf{Emergence:} Intelligenz entsteht durch Interaktion mit der Umwelt
	\end{itemize}
\end{itemize}

\subsubsection{Die Anfänge}
\begin{itemize}
	\item Walters Machina Speculatrix 1953
	\item Braitenberg Vehicles 1984
\end{itemize}

\subsection{Beispiele für verhaltensbasierte Architekturen}

\subsubsection{Anforderung an die Architektur / Bewertungsschema}
\begin{itemize}
	\item Unterstützung der \textbf{Parallelisierung}: Verhalten auf verschiedenen Prozessoren berechnen
	\item Abstimmung auf die Hardware / \textbf{Hardwareabhängigkeit}: Verhalten speziell auf Hardware abgestimmt
	\item \textbf{Modularität}: Einfache klare Module, einfache System"=Erweiterbarkeit
	\item \textbf{Robustheit}: Gesamtsystem sollte stets stabil bleiben
	\item Fortschritt der \textbf{Umsetzung}: Gibt es eine funktionierende Implementierung?
	\item \textbf{Flexibilität zur Laufzeit}: Kann sich das System anpassen?
	\item \textbf{Effektivität} in Hinsicht der Erfüllung der Aufgabe: System Overhead?
\end{itemize}

\subsubsection{Subsumption (MIT)}
\begin{itemize}
	\item Anordnung der Verhalten auf horizontalen Schichten
	\item Alle Verhalten greifen auf alle Sensordaten zu und generieren Aktionen für alle Aktoren
	\item Interaktion der einzelnen Verhalten durch Inhibieren von Eingängen und Überstimmen von Ausgängen
	\item Rückkopplung primär über die Umgebung
	\item Eigenes C Derivat: Interactive C
\end{itemize}
Siehe \autoref{subsum}.
\begin{figure}[h!]
	\begin{subfigure}{.5\textwidth}
		\centering
		\includegraphics[width=\textwidth]{figures/aufbau_verhaltensebenen.png}
		\caption{Aufbau der Verhaltensebenen}
	\end{subfigure}
	\begin{subfigure}{.5\textwidth}
		\centering
		\includegraphics[width=\textwidth]{figures/aufbau_verhaltensebenen_1.png}
		\caption{Aufbau der Verhaltensebenen}
	\end{subfigure}\par\medskip
	\begin{subfigure}{.5\textwidth}
		\centering
		\includegraphics[width=\textwidth]{figures/subsumption_basiselement.png}
		\caption{Subsumption-Basiselement: Blockiere verhindert das ein Signal ausgegeben wird, Unterdrücker unterdrückt ursprüngliches Signal und ersetzt es durch das Unterdrückungssignal, Reset stellt den Ursprungszustand wieder her}
	\end{subfigure}
	\begin{subfigure}{.5\textwidth}
		\centering
		\includegraphics[width=\textwidth]{figures/verschaltung_basiselement.png}
		\caption{Verschaltung von Basiselementen}
	\end{subfigure}
	\caption{Subsumptionsarchitektur}
	\label{subsum}
\end{figure}\\
Ein Beispiel für diese Architektur zeigt \autoref{subbsp}.
\begin{figure}[h!]
	\includegraphics[width=.8\textwidth]{figures/subsumption_beispiel.png}
	\caption{Subsumption: Beispiel der MIT Laufroboter Ghengis/Attila -- Ebenen}
	\label{subbsp}
\end{figure}

\textbf{Subsumption Bewertung}
\begin{itemize}
	\item Parallelisieren möglich
	\item Hardwareabhängig
	\item Modulare erweiterbare Struktur teilweise vorhanden
	\item Robustheit von Modulen abhängig
	\item Prototypisch umgesetztes System
	\item Keine Flexibilität zur Laufzeit
	\item Teilweise effektive Erfüllung der Aufgabe
\end{itemize}

\subsubsection{Verhaltensnetzwerke/ Verhaltenssteuerung}
\textbf{Vorteile}
\begin{itemize}
	\item Flexibel, adaptiv
	\item Robust, Ausfall eines Verhaltens führt nicht zu Totalausfall
	\item Können auf unvorhergesehene Situationen besser reagieren
	\item Asynchron, jedes Verhalten kann eigene Zykluszeiten verwenden
	\item Viele kleine Module, die separat testbar sind
	\item Unterstützt Parallelisierung
\end{itemize}
\textbf{Nachteile}
\begin{itemize}
	\item Verhalten des gesamten Systems schwer zu überschauen 
	\item Viel Aufwand durch Tuning des Systems, speziell der Fusion
	\item[$\rightarrow$] schwer zu debuggen, jedes einzelne Verhalten ja, aber wenn zu großem Gesamtverhalten zusammengeschaltet \textbf{unvorhersehbar}!!!
	\end{itemize}
Die Vor- und Nachteile hängen von konkreten Architektur"=Ausprägungen ab. Mögliche Varianten und deren Charakteristika sind:
\begin{itemize}
	\item Fusionsprinzip/Coordinator: von Kooperativ bis Konkurrierend
	\item Abstaktionsniveau/Granularity: von Reaktiv bis Deliberativ
	\item Datenformat/Action"=encoding: von Diskret bis Kontinuierlich
\end{itemize}
\noindent
\textbf{Ausgabe"=Fusion}:\\
Die Fusion oder Koordination der Ausgabe der einzelnen Verhalten:
\begin{itemize}
	\item \textbf{Cooperative}
	\begin{itemize}
		\item Alle Verhalten bestimmen zusammen eine Aktion oder 
		\item Alle Verhalten bestimmen die Stärke vieler Aktionen 
		\item[$\Rightarrow$] Sprungfreie Aktionen
	\end{itemize}
	\item \textbf{Competitive}
	\begin{itemize}
		\item Ein Verhalten bestimmt die zu ausführende Aktion
		\item[$\Rightarrow$] Einfacher zu überschauen und zu testen
	\end{itemize}
\end{itemize}
Problem: beste Wahl hängt stark von Situation und konkreten Verhalten ab. Dies ist in \autoref{bsp3} an einem Beispiel verdeutlicht.
\autoref{fus} stellt die unterschiedlichen Methoden zur Fusion gegenüber.\\
\begin{figure}
	\centering
	\begin{subfigure}{.8\textwidth}
		\centering
		\includegraphics[width=\textwidth]{figures/fusion_beispiel.png}
		\caption{Beispiel: Drei Verhalten: weiche nach links/rechts aus, fahre gerade zum Ziel}
		\label{bsp3}
	\end{subfigure}\par\medskip
	\begin{subfigure}{.8\textwidth}
		\centering
		\includegraphics[width=\textwidth]{figures/fusion_methoden.png}
		\caption{Unterschiedliche Methoden zur Fusion}
		\label{fus}
	\end{subfigure}
	\caption{Verhaltensbasierte Steuerungen: Ausgabe-Fusion}
\end{figure}
\textbf{Priorität"=basierte Fusion}
\begin{itemize}
	\item Jedem Verhalten wird eine Priorität zugewiesen
	\item Prioritäten sind (meist) dynamisch
	\item Zuweisung erfolgt durch höhere Verhalten oder Arbiter, Automaten, Planer, etc.
	\item Konkurrierend: Verhalten mit höchster Prioriät bestimmt Aktion
\end{itemize}

\textbf{Aktivität"=basierte Fusion}
\begin{itemize}
	\item Jedes Verhalten bestimmt selbst seine Aktivität
	\begin{itemize}
		\item Normierte Stellgröße
		\item Normierte Soll"=Ist Differenz ...
	\end{itemize}
	\item Entsprechend der Aktivität wird gewählt
	\begin{itemize}
		\item Konkurrierend: Das Verhalten mit der höchsten Aktivität wählt eine Aktion
		\item Kooperative: Jede Aktion wird mit der Summe der Aktivitäten gewichtet ausgeführt
	\end{itemize}
\end{itemize}

\textbf{Fuzzy"=basierte Fusion:}\\
\begin{center}
\includegraphics[width=.8\textwidth]{figures/fuzzy_based.png}
\end{center}
\textbf{Wahl"=basierte Fusion}
\begin{itemize}
	\item Endliche Menge definierter Aktionen
	\item Jedes Verhalten stimmt für eine Aktion
	\begin{itemize}
		\item Priority Based: Jedes Verhalten hat ein Stimmrecht gleich seiner Priorität
		\item Activity Based: Jedes Verhalten hat ein Stimmrecht gleich seiner Aktivität
	\end{itemize}
	\item Entsprechend der Stimmen wird ausgewählt
	\begin{itemize}
		\item Die Aktivität mit den meisten Stimmen wird ausgeführt
		\item Alle Aktivitäten werden mit ihren Stimmen gewichtet ausgeführt
	\end{itemize}
	\item Variante: Fusion besitzt ein Gedächtnis. Verhalten kann sowohl für als auch gegen eine Aktion stimmen. Die stimmen werden über die Zyklen akkumuliert.\\
	 Vor- und Nachteil: Schnelle Aktionswechsel können verhindert werden.
\end{itemize}
\noindent
\textbf{Status"=basierte Fusion}
\begin{itemize}
	\item Ein endlicher Automat kontrolliert die Verhalten
	\item Je nach Zustand bestimmt der Automat, welches Verhalten die Aktion auswählen kann
	\item Wird verwendet um Verhalten Prioritäten zuweisen zu können
	\item Wird verwendet um zwischen verschiedenen Fusionsmethoden zu wechseln
\end{itemize}

\subsubsection{Verhaltensnetzwerke (FZI)}
\begin{itemize}
	\item Einheitliche Schnittstelle der Reflexeinheiten, bestehend aus Sensor- und Steuerungsdaten sowie Aktivierung, Aktivität und Bewertung
	\item Modularer Aufbau
	\item Hierarchische Anordnung innerhalb der Architektur abhängig von der Roboter"=Kinematik
	\item Trennung von Daten- und Kontrollfluss
	\item Implementierung in MCA2
\end{itemize}

\textbf{BISAM}:\\
\begin{center}
\includegraphics[width=.3\textwidth]{figures/bisam.png}
\end{center}

\begin{figure}
	\begin{subfigure}{.5\textwidth}
		\centering
		\includegraphics[width=\textwidth]{figures/verhalten_aufbau.png}
		\caption{Verhalten -- Aufbau}
	\end{subfigure}
	\begin{subfigure}{.5\textwidth}
		\centering
		\includegraphics[width=\textwidth]{figures/verhalten_ausgabefusion.png}
		\caption{Verhalten -- Ausgabefusion}
	\end{subfigure}\par\medskip
	\begin{subfigure}{.5\textwidth}
		\centering
		\includegraphics[width=\textwidth]{figures/verhalten_kopplung.png}
		\caption{Verhalten -- Kopplung}
	\end{subfigure}
	\begin{subfigure}{.5\textwidth}		
		\centering
		\includegraphics[width=\textwidth]{figures/verhalten_netzwerke.png}
		\caption{Verhalten -- Netzwerke}
	\end{subfigure}
	\caption{}
	\label{fig:verhalten_bisam}
\end{figure}
\autoref{fig:verhalten_bisam} veranschaulicht BISAMs Verhalten.
Zu \autoref{fig:verhalten_bisam}d:
\begin{itemize}
	\item Verteilung von reaktiv nach deliberativ
	\item Zustandsbewertung und Aktivität a als virtuelle Sensoren
	\item Motivation t als virtueller Aktor
	\item Regionen durch virtuelle Aktoren
	\item Stress und Angst einer Region
	\begin{align}	
		Angst_T(A) &= \int_{t = t_n - T}^{t_n} \left( \sum_{A_t \in R(A)} r_i\left(t\right)\right) dt \\
		Stress_T(a) &= \int_{t = t_n - T}^{t_n} \left( \sum_{A_t \in R(A)} a_i\left(t\right)\right) dt
	\end{align}
\end{itemize}

\textbf{Verhalten -- Entwurfsverfahren}:
\begin{enumerate}
	\item Definition der Verhalten und Reflexe
	\item Finden von kinematischen Gruppen
	\item Verteilen der Verhalten und Reflexe auf die Gruppen
	\item Top-Down Sortierung deliberativ nach reaktiv (Regionen)
	\item Bottom-Up Implementierung und schrittweise Validierung
\end{enumerate}
\newpage
\textbf{Statisch stabiles Laufen}:
\begin{itemize}
	\item Variabler Untergrund
	\item Adaption an Hindernisse
	\item Entlastung des nächsten Schwungbeines
	\item Stabilität während des Schwungzyklus
\end{itemize}

BISAM Bewertung:
\begin{itemize}
	\item Parallelisieren möglich
	\item Hardwareunabhängig
	\item Modulare, erweiterbare Strukturen sind vorhanden
	\item Robustheit ist abhängig von Einzelverhalten
	\item System ist prototypisch umgesetzt
	\item Beschränkte Flexibilität der Laufzeit
	\item Effektive Erfüllung der Aufgaben
\end{itemize}
\noindent
\textbf{InBOT} (selbstfahrender Einkaufswagen, der mit Menschen interagiert) als Beispiel für verhaltesbasierte Steuerung für Fahrzeuge:
\begin{figure}[h!]
	\includegraphics[width=\textwidth]{figures/inbot.png}
	\caption{Verhaltensbasierte Steuerung für Fahrzeuge Beispiel InBOT}
	\label{ch:08:fig:inbot}
\end{figure}
Hierarchisches Modell von Fußgängerbewegungen:
\begin{itemize}
	\item Strategische Verhalten:
	\begin{itemize}
		\item Grobe Routenplanung
		\item Basiert auf Vorlieben
		\item Erzeugt Knotenpunkte (KP)
	\end{itemize}
	\item Taktische Verhalten
	\begin{itemize}
		\item Erzeugt Unter-KP
		\item Geometrie der Umgebung
	\end{itemize}
	\item Ausführende Verhalten
	\begin{itemize}
		\item Bewegungserzeugung
		\item Pfad wird an Hindernisse und Menschen angepasst
	\end{itemize}
\end{itemize}

Vorteile:
\begin{itemize}
	\item Nutzen erfolgreicher Navigationsstrategien
	\item Geringer A-Priori Informationsbedarf (keine detaillierte Karten)
\end{itemize}
Nachteile:
\begin{itemize}
	\item Das Verhalten des Roboters ist unscharf
	\item D.h. schwieriger zu kontrollieren
\end{itemize}
%TODO f59,60,62
\noindent
\textbf{Welt Modell} -- Drei Ebenen:
\begin{itemize}
	\item Topologische Karte
	\begin{itemize}
		\item Invariant
		\item Semantische Erweiterung
	\end{itemize}
	\item Objekt Datenbank
	\begin{itemize}
		\item Segmentierte Hindernisse
		\item Tracking Informationen
	\end{itemize}
	\item Occupancy Maps
	\begin{itemize}
		\item Basislevel
		\item Eignet sich gut für Datenfusion
		\item Diskretes Welt Modell
	\end{itemize}
\end{itemize}

\textbf{Aufgaben-Orientiertes Level}:
Vgl. \autoref{ch:08:fig:aufgaben-orientiert}.
\begin{figure}[h!]
	\begin{subfigure}{.5\textwidth}
		\centering
		\includegraphics[width=\textwidth]{figures/aufgaben_level.png}
	\end{subfigure}
	\begin{subfigure}{.5\textwidth}
		\centering
		\includegraphics[width=\textwidth]{figures/aufgaben_level_1.png}
	\end{subfigure}
	\caption{Aufgaben-Orientiertes Level}
	\label{ch:08:fig:aufgaben-orientiert}
\end{figure}
\begin{itemize}
	\item Fusion der Task-Orientierten Verhalten
	\item Auswahl eines Verhaltens durch eine Maximum Fusion (konkurrierend)
	\item Vorausschauende Hindernis"=Behandlung
	\item Sub-Ziele werden generiert, wenn der direkte Weg zum Ziel blockiert ist
	\item Ecken der Hindernisse dienen als Sub-Ziele
	\item Der Roboter bewegt sich ähnlich zu einem Sichtgraph
	\item Alle sichtbaren Eckpunkte werden nach ihrer Qualität gewichtet
	\item Ecke mit bester Qualität wird ausgewählt
	\item Bei Sackgassen/Lokalen Minima:
	\begin{itemize}
		\item Günstigste Ecke wird als Sub-Ziel gewählt
		\item Befindet sich der Roboter bereits in der Sackgasse ist das Sub-Ziel hinter ihm
		\item[$\Rightarrow$] Der Roboter kann um Sackgassen herumfahren oder aus ihnen entkommen
		\item[$\Rightarrow$] Klassisches Problem der Lokalen Minima der Potentialfeldermethoden wird vermieden
	\end{itemize}
\end{itemize}
\noindent
\textbf{Objekt-Orientiertes Level}:
Vgl. \autoref{ch:08:fig:objekt-orientiert}.
\begin{figure}[h!]
	\begin{subfigure}{.5\textwidth}
		\centering
		\includegraphics[width=\textwidth]{figures/objekt_level.png}
		\caption{Objekt-Orientiertes Level}
	\end{subfigure}
	\begin{subfigure}{.5\textwidth}
		\centering
		\includegraphics[width=\textwidth]{figures/objekt_level_1.png}
		\caption{Behandlung statischer Hindernisse}
	\end{subfigure}
	\caption{}
	\label{ch:08:fig:objekt-orientiert}
\end{figure}
\begin{itemize}
	\item Fusion der Aktion aus dem Aufgaben-Orientierten Level mit den Ausgaben der Verhaltensgruppen zur Hindernisbehandlung
	\item Kooperative Fusion (alle drei Aktionen werden gewichtet und gleichzeitig ausgeführt)
\end{itemize}
Objekt-Orientiertes Level: Avoid Obstacles (vgl. \autoref{ch:08:fig:objekt-orientiert-hindernis}).
\begin{figure}[h!]
	\centering
	\begin{subfigure}{.4\textwidth}
		\centering
		\includegraphics[width=\textwidth]{figures/objekt_level_hindernis.png}
	\end{subfigure}
	\begin{subfigure}{.4\textwidth}
		\centering
		\includegraphics[width=\textwidth]{figures/objekt_level_hindernis_1.png}
	\end{subfigure}
	\caption{Aufgaben-Orientiertes Avoid Obstacles}
	\label{ch:08:fig:objekt-orientiert-hindernis}
\end{figure}
\begin{itemize}
	\item Dynamische Vektorfeld Methode
	\item Erzeugt Vektor, der vom Hindernis"=Schwerpunkt weg zeigt
	\item Jeweils ein abstoßender Vektor für Links- und Rechtsseitige Hindernisse
	\item Fusion (kooperativ) dieser beiden Vektoren mit einem anziehenden Vektor Richtung Ziel
\end{itemize}

Bewertung InBOT:
\begin{itemize}
	\item Parallelisieren möglich
	\item Hardwareunabhängig
	\item Modulare, erweiterbare Struktur vorhanden
	\item Robustheit ist abhängig von Einzelverhalten
	\item System ist sehr weit umgesetzt
	\item Flexibilität zur Laufzeit
	\item Effektive Erfüllung der Aufgabe
\end{itemize}
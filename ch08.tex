\section{Verhalten im Tierreich: Neurologisch, Psychologisch, Ethologisch}

\subsubsection{Neurologisch -- Vektorfeldaddition im Frosch-Rückenmark}
\begin{itemize}
	\item Neurobiologische Hypothese der motorischen Kontrolle durch Vektorfelder
	\item Bizzi (MIT) hat gezeigt, dass Gliedmaßenbewegung in Fröschen im Rückenmark codiert abgelegt sind
	\item Mikrostimulation im Rückenmark generiert eine Bewegung zu einem spezifischen Punkt
	\item Planung im ZNS wird auf solche Reize umgesetzt
\end{itemize}

\subsubsection{Psychologisch -- Verhaltens-, Gestalt- und Kognitive"-Psychologie}
\begin{itemize}
	\item \textbf{Verhaltens-Psychologie:}
	Verhalten wird durch Beobachtung definiert.
	Alles wird auf ein Stimulus-Antwort Schema zurückgeführt.
	\item \textbf{Gestalt-Psychologie:}
	Einbeziehung der Physik.
	Verhalten entsteht als direkte Konsequenz aus der Struktur der physischen Umgebung.
	\item \textbf{Kognitive Psychologie:}
	Einbeziehung des Wissens.
	Vereinheitlichte Methode, um die Beziehung von Aktion und Perzeption zu erklären.
\end{itemize}

\subsubsection{Ethologisch -- Reflexe, Taxe und feste Muster}
\begin{itemize}
	\item Ethologie bezeichnet die Beobachtung von tierischem Verhalten in seinem natürlichen Lebensraum.
	\item Das Tier wird als Teil des Gesamtsystems angesehen, das auch die Umgebung mit einbeziehen muss.
	\item Klassifizierung in \textbf{Reflexe}, \textbf{Taxe} und \textbf{Feste  Muster}.
	\item Motivierte Verhalten als Antwort auf interne Stimuli.
	\item Schematheorie (Lorenz): Zusammenfassung von komplizierten Kombinationen aus Reflexen, Taxen und Mustern zu Schemata.
	\item Prinzip der \glqq Ökologische Nische \grqq (McFarland). Ein Tier überlebt, weil es eine Nische gefunden hat.
\end{itemize}

Drei Klassen von Verhalten im Tierreich:
\begin{itemize}
	\item \textbf{Reflexe}: schnelle, automatische und unbewusste Reaktionen, die von einem sensorischen Stimulus ausgelöst werden.
	\item \textbf{Taxe}: Verhalten, die ein Tier in Richtung (attractive) oder entgegen (aversive) eines Stimulus orientieren.
	\item \textbf{Feste Muster}: Reaktionen auf Stimuli, die länger anhalten als die Stimuli selbst.
\end{itemize}

\section{Verhaltensbasierte Robotik -- Übersicht  und Einführung}

\subsubsection{Motivation -- Warum verhaltensbasierte Architekturen}
\begin{itemize}
	\item Komplexe Verhalten entstehen nicht notwendigerweise aus komplexen Steuerungssystemen
	\item Die Natur ist das beste Vorbild
	\item Simplizität ist eine Tugend
	\item Robustheit bei verrauschten Sensorwerten ist ein Designziel
	\item Systeme sollten inkrementell aufgebaut werden können
	\item Alle Berechnungen Onboard
	\item Roboter sollten billig sein
\end{itemize}

\subsubsection{Roboter Steuerungs"=Spektrum -- Von reaktiv bis deliberativ}
\textbf{Roboter Steuerungs-Spektrum:}
\begin{center}
	\includegraphics[width=0.5\textwidth]{figures/steuerungs_spektrum.png}
\end{center}

\textbf{Hierarchische Steuerung:}
\begin{itemize}
	\item Hierarchische Anordnung von Steuerungselemente, wobei höhere Ebenen Unterziele für niedrigere erzeugen
	\item Eigenschaften
	\begin{itemize}
		\item Es wird vorausgesetzt, dass die zu erfüllende Aufgabe zerlegbar ist
		\item Ein Weltmodell auf jedem Level garantiert die korrekte Ausführung
	\end{itemize}
\end{itemize}

\textbf{Reaktive Steuerung:}
\begin{itemize}
	\item Bedeutet vereinfacht eine enge Kopplung zwischen Perzeption und Aktion
	\item Definierte Schlüsselelemente:
	\begin{itemize}
		\item \textbf{Embodiment:} Der Roboter hat eine physikalische Präsenz
		\item \textbf{Situatedness:} Der Roboter ist als Einheit in seiner Umgebung eingebunden
		\item \textbf{Emergence:} Intelligenz entsteht durch Interaktion mit der Umwelt
	\end{itemize}
\end{itemize}

\subsubsection{Die Anfänge}
\begin{itemize}
	\item Walters Machina Speculatrix 1953
	\item Braitenberg Vehicles 1984
\end{itemize}

\section{Beispiele für verhaltensbasierte Architekturen}

\subsubsection{Anforderung an die Architektur / Bewertungsschema}
\begin{itemize}
	\item Unterstützung der \textbf{Parallelisierung}: Verhalten auf verschiedenen Prozessoren berechnen
	\item Abstimmung auf die Hardware / \textbf{Hardwareabhängigkeit}: Verhalten speziell auf Hardware abgestimmt
	\item \textbf{Modularität}: Einfache klare Module, einfache System"=Erweiterbarkeit
	\item \textbf{Robustheit}: Gesamtsystem sollte stets stabil bleiben
	\item Fortschritt der \textbf{Umsetzung}: Gibt es eine funktionierende Implementierung?
	\item \textbf{Flexibilität zur Laufzeit}: Kann sich das System anpassen?
	\item \textbf{Effektivität} in Hinsicht der Erfüllung der Aufgabe: System Overhead?
\end{itemize}

\subsubsection{Subsumption (MIT)}
\begin{itemize}
	\item Anordnung der Verhalten auf horizontalen Schichten
	\item Alle Verhalten greifen auf alle Sensordaten zu und generieren Aktionen für alle Aktoren
	\item Interaktion der einzelnen Verhalten durch Inhibieren von Eingängen und Überstimmen von Ausgängen
	\item Rückkopplung primär über die Umgebung
	\item Eigenes C Derivat: Interactive C
\end{itemize}

\begin{figure}
\centering
\subfloat[]{\includegraphics[width=0.5\textwidth]{figures/aufbau_verhaltensebenen.png}}
\subfloat[]{\includegraphics[width=0.5\textwidth]{figures/aufbau_verhaltensebenen_1.png}}\par\medskip
\subfloat[]{\includegraphics[width=0.5\textwidth]{figures/subsumption_basiselement.png}}
\subfloat[]{\includegraphics[width=0.5\textwidth]{figures/verschaltung_basiselement.png}}
	\caption{\textbf{Subsumption}: a) und b) Aufbau der Verhaltensebenen; c) Subsumption-Basiselement: Blockiere verhindert das ein Signal ausgegeben wird, Unterdrücker unterdrückt ursprüngliches Signal und ersetzt es durch das Unterdrückungssignal, Reset stellt den Ursprungszustand wieder her; d) Verschaltung von Basiselementen}
\end{figure}

\begin{figure}
	\includegraphics[width=\textwidth]{figures/subsumption_beispiel.png}
	\caption{Subsumption Beispiel der MIT Laufroboter Ghengis/Attila -- Ebenen}
\end{figure}

\textbf{Subsumption Bewertung}
\begin{itemize}
	\item Parallelisieren möglich
	\item Hadwareabhängig
	\item Modulare erweiterbare Struktur teilweise vorhanden
	\item Robustheit von Modulen abhängig
	\item Prototypisch umgesetztes System
	\item Keine Flexibilität zur Laufzeit
	\item Teilweise effektive Erfüllung der Aufgabe
\end{itemize}

\subsubsection{Verhaltensnetzwerke/ Verhaltenssteuerung}
\textbf{Vorteile}
\begin{itemize}
	\item Flexibel, adaptiv
	\item Robust, Ausfall eines Verhaltens führt nicht zu Totalausfall
	\item Können auf unvorhergesehene Situationen besser reagieren
	\item Asynchron, jedes Verhalten kann eigene Zykluszeiten verwenden
	\item Viele kleine Module, die separat testbar sind
	\item Unterstützt Parallelisierung
\end{itemize}
\textbf{Nachteile}
\begin{itemize}
	\item Verhalten des gesamten Systems schwer zu überschauen 
	\item Viel Aufwand durch Tuning des Systems, speziell der Fusion
\end{itemize}
Die Vor- und Nachteile hängen von konkreten Architektur"=Ausprägungen ab. Mögliche Varianten und deren Charakteristika sind:
\begin{itemize}
	\item Fusionsprinzip/Coordinator: von Kooperativ bis Konkurrierend
	\item Abstaktionsniveau/Granularity: von Reaktiv bis Deliberativ
	\item Datenformat/Action"=encoding: von Diskret bis Kontinuierlich
\end{itemize}

\textbf{Ausgabe"=Fusion}: die Fusion oder Koordination der Ausgabe der einzelnen Verhalten:
\begin{itemize}
	\item \textbf{Cooperative}
	\begin{itemize}
		\item Alle Verhalten bestimmen zusammen eine Aktion oder 
		\item Alle Verhalten bestimmen die Stärke vieler Aktionen 
		\item[$\Rightarrow$] Sprungfreie Aktionen
	\end{itemize}
	\item \textbf{Competitive}
	\begin{itemize}
		\item Ein Verhalten bestimmt die zu ausführende Aktion
		\item[$\Rightarrow$] Einfacher zu überschauen und zu testen
	\end{itemize}
\end{itemize}
Problem: beste Wahl hängt stark von Situation und konkreten Verhalten ab siehe \autoref{ch:08:fig:fusion_beispiel}.

\begin{figure}
	\centering
	\subfloat[]{\includegraphics[width=0.8\textwidth]{figures/fusion_beispiel.png}}\par\medskip
	\subfloat[]{\includegraphics[width=0.8\textwidth]{figures/fusion_methoden.png}}
	\caption{a) Beispiel: 3 Verhalten: weiche nach links/rechts aus, fahre gerade zum Ziel; b) unterschiedliche Methoden zur Fusion.}
	\label{ch:08:fig:fusion_beispiel}
\end{figure}

\textbf{Priorität"=basierte Fusion}
\begin{itemize}
	\item Jedem Verhalten wird eine Priorität zugewiesen
	\item Prioritäten sind (meist) dynamisch
	\item Zuweisung erfolgt durch höhere Verhalten oder Arbiter, Automaten, Planer, etc.
	\item Konkurrierend: Verhalten mit höchster Prioriät bestimmt Aktion
\end{itemize}

\textbf{Aktivität"=basierte Fusion}
\begin{itemize}
	\item Jedes Verhalten bestimmt selbst seine Aktivität
	\begin{itemize}
		\item Normierte Stellgröße
		\item Normierte Soll"=Ist Differenz ...
	\end{itemize}
	\item Entsprechend der Aktivität wird gewählt
	\begin{itemize}
		\item Konkurrierend: Das Verhalten mit der höchsten Aktivität wählt eine Aktion
		\item Kooperative: Jede Aktion wird mit der Summe der Aktivitäten gewichtet ausgeführt
	\end{itemize}
\end{itemize}

\textbf{Fuzzy"=basierte Fusion:}\\
\includegraphics[width=.8\textwidth]{figures/fuzzy_based.png}

\textbf{Wahl"=basierte Fusion}
\begin{itemize}
	\item Endliche Menge definierter Aktionen
	\item Jedes Verhalten stimmt für eine Aktion
	\begin{itemize}
		\item Priority Based: Jedes Verhalten hat ein Stimmrecht gleich seiner Priorität
		\item Activity Based: Jedes Verhalten hat ein Stimmrecht gleich seiner Aktivität
	\end{itemize}
	\item Entsprechend der Stimmen wird ausgewählt
	\begin{itemize}
		\item Die Aktivität mit den meisten Stimmen wird ausgeführt
		\item Alle Aktivitäten werden mit ihren Stimmen gewichtet ausgeführt
	\end{itemize}
	\item Variante: Fusion besitzt ein Gedächtnis. Verhalten kann sowohl für als auch gegen eine Aktion stimmen. Die stimmen werden über die Zyklen akkumuliert. Vor- und Nachteil: Schnelle Aktionswechsel können verhindert werden.
\end{itemize}

\textbf{Status"=basierte Fusion}
\begin{itemize}
	\item Ein endlicher Automat kontrolliert die Verhalten
	\item Je nach Zustand bestimmt der Automat, welches Verhalten die Aktion auswählen kann
	\item Wird verwendet um Verhalten Prioritäten zuweisen zu können
	\item Wird verwendet um zwischen verschiedenen Fusionsmethoden zu wechseln
\end{itemize}

\subsubsection{Verhaltensnetzwerke (FZI)}
\begin{itemize}
	\item Einheitliche Schnittstelle der Reflexeinheiten, bestehend aus Sensor- und Steuerungsdaten sowie Aktivierung, Aktivität und Bewertung
	\item Modularer Aufbau
	\item Hierarchische Anordnung innerhalb der Architektur abhängig von der Roboter"=Kinematik
	\item Trennung von Daten- und Kontrollfluss
	\item Implementierung in MCA2
\end{itemize}

\textbf{BISAM}:\\
\begin{center}
\includegraphics[width=.5\textwidth]{figures/bisam.png}
\end{center}